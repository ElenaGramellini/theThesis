\chapter{Introduction}\label{ch:-1}
The discovery of the Higgs boson in 2012 marked the triumph of the Standard Model of Particle Physics; exploring what lays beyond is the real challenge in our field today. 
Since their formulation in 1930, neutrinos have been a source of surprises (and Nobel Prizes) for particle physicists, tiny cracks in our understanding of Nature. In particular, the discovery of neutrino oscillation represents the first evidence of physics Beyond the Standard Model (BSM).  From a theoretical point of view, the field is developing new theories to account for the small but non-zero mass of neutrinos, while trying to remain consistent with the rest of the Standard Model.  From an experimental point of view, we are developing technologies and huge collaborations to probe these theories. As we enter the era of high statistics, precision measurements of neutrino interactions, neutrinos might hold the key to the next generation of discoveries in particle physics.


This thesis work describes the first measurement of the  ($\pi^-$-Ar)  total hadronic cross section  in the 100-1000 MeV  kinetic energy range and the first measurement of the ($K^+$-Ar) total hadronic cross section  in the 100-650 MeV  kinetic energy range. These measurements were performed with the LArIAT experiment,  a small (0.25 ton)  Liquid Argon Time Projection Chamber (LArTPC) in a beam of charged particles at the Fermilab Test Beam Facility.   Particle and nuclear physics have a long history of hadronic cross section measurements; what makes these measurements unique is both the target (argon) and the methodology used -- the ``thin slice method" -- which takes advantage of the detection capabilities of the LArTPC technology. The combination of fine-grained tracking and excellent calorimetric information provided by the LArTPC technology  enables the measurement of unprecedented details of particle interactions in argon and, in LArIAT, to measure the kinetic energy of a hadron at each step along the particle traces. A renewed interest for precision measurements of hadronic cross sections, particularly in argon, arises from the current  panorama of experimental  particle physics at the intensity frontier, in particular neutrino physics in LArTPCs.


Experimentally, precision measurements can be achieved only if the detector technology is able to resolve the fine details of a neutrino interaction and to record a statistically relevant number of neutrinos.  With ``fine details" here we mean the ability to distinguish the many products of the neutrino interaction, such as protons, pions, muons and electrons, and to measure their energy.
Historically,  bubble chamber neutrino detectors were the first revolution in neutrino detection: for example, the spatial resolution of Gargamelle allowed the discovery of neutrino neutral current interaction\cite{HASERT1973138}. Despite the high precision of bubble chambers images, this technology is hard to scale to massive size, making statistical analyses on neutrino interactions almost impossible to perform. To make up for the small neutrino interaction cross section, neutrino experiments moved to very large size, at the expenses of spatial precision. This is the case for the detectors which discovered neutrino oscillation:  both Super-Kamiokande and SNO are massive Cherenkov detectors \cite{PDGOsc}. With LArTPCs, the field is gaining again bubble-chamber like precision but at massive scales. Following the recommendations of  the latest Particle Physics Project Prioritization Panel  \cite{P5}, the US particle physics panorama is directing a substantial effort towards the exploration of the intensity frontier through the construction of massive LArTPCs. In particular, we are seeing the development of a Short Baseline Neutrino Program (SBN) and long baseline neutrino program  (DUNE), both based on the LArTPC detector technology. The US liquid argon program has the potential to answer many of the fundamental open questions in particle physics today, such as: is there a fourth generation neutrino? is CP violated in the lepton sector? are there any additional symmetries? and, can we find an indication of Grand Unified Theories? 

The SBN program at Fermilab is tasked with conclusively addressing the existence of a fourth neutrino generation in the  $\Delta m^2= \Delta m^2_{14} \sim [0.1 - 10]$ eV$^2$ parameter space. The SBN program entails three surface LArTPCs positioned on the Booster Neutrino Beam at different distances from the neutrino production in oder to fully exploit  the L/E dependence of the oscillation pattern:  SBND (110 m from the decay pipe), MicroBooNE (470 m), and ICARUS (600 m). SBN will also perform an extensive 
program of neutrino cross section measurements, fundamental to abate systematics in the oscillation analyses in both SBN and DUNE.

DUNE has a vast neutrino and non-accelerator physics reach. Within neutrino physics, oscillation analyses in DUNE have the capability of solving the mass hierarchy and octant problem,  and discovering CP violation in the neutrino sector. Besides its neutrino program, DUNE can open an experimental window on Grand Unified Theories (GUTs). GUTs could potentially answer fundamental questions such as the existence of non-zero neutrino masses and matter-antimatter asymmetry, explaining some ``accidents" in the Standard Models, such as the exact cancellation of the  proton and the electron charge.   Directly probing GUTs at the unification energy scale is impossible by any foreseeable collider experiment. We then need an indirect proof such as baryon number violation, which is predicted by almost every GUT in the form of proton decay, bounded nucleon decay or $n-\bar n$ oscillations on long time-scales. Historically, the main technology used in these searches has been water Cherenkov detectors, with Super-Kamiokande setting all the current experimental limits on the decay lifetimes at the order of $\sim 10 ^{34}$ years. The DUNE far detector and its non-accelerator physics program is a interesting new actor on this stage.  LArTPCs can in fact complement nucleon decay searches in modes where water Cherenkov detectors are less sensitive, especially $p\rightarrow K^+\bar{\nu}$  \cite{Adams:2013qkq}.


Such a diverse physics program speaks to the versatility of the LArTPC technology. LArTPCs provide excellent electron/photon separation \cite{Acciarri:2016sli} lacking in Cherenkov detectors which can be leveraged to abate the photon background from neutral current interactions  in $\nu_e$ searches. LArTPCs also share superb tracking capability with bubble chamber detectors, with several additional benefits. They are electronically read out and self triggered detectors; they provide full 3D-imaging with millimeter resolution, precise calorimetric reconstruction and excellent particle identification. %Plus, female physicists are actually writing code and running experiments, not (only) staring at images \textcolor{red}{-- Too much? --} \cite{2006physics4152G}. 

The amount of information a LArTPC can provide makes these detectors rather complex: a series of dedicated measurements is necessary to obtain meaningful physics results from a LArTPC. The complexity of the LArTPC technology for neutrino detection is due to several reasons. Argon is a fairly heavy element, which means that nuclear effects play an important role in the interaction topology. For example, pions are one of the main products of neutrino interactions; yet,  since data on charged particle interaction in argon is scarce, neutrino event generators have big uncertainties in the re-scattering simulation of pions in argon. %No measurement of hadronic cross sections for pions or kaons is available for argon (yet). 
The amount of details in a LArTPC event can easily be parsed by human eye, but can make automatic event reconstruction rather challenging. Thus, reconstruction algorithms in LArTPC need to be tuned to recognize the different topologies of the neutrino interaction products in argon. This is particularly true for pions, since they are copiously produced of the neutrino interactions: the occurrence of a pion interaction in argon can modify the topology of the neutrino event, causing a misidentification of the neutrino interaction channel.

The LArIAT \cite{Cavanna:2014iqa} experiment is performing precise cross section measurements of charged particles in argon to address this gap of knowledge. 
The LArIAT LArTPC sits on a beam of charged particles at the Fermilab Test Beam Facility which provides charge particles of the type and energy range relevant for neutrino interaction of both SBN and DUNE. The ($\pi^-$-Ar) hadronic cross section is a fundamental input for neutrino detectors in liquid argon, as pion interactions can modify the topology and energy reconstruction of neutrino events in the GeV range, where pion production is abundant. The  ($K^+$-Ar) total hadronic cross section in LArIAT is particularly relevant for a high identification efficiency in the context of proton decay searches in DUNE in the  $p\rightarrow K^+\bar{\nu}$  channel. In fact, the kaon-argon cross section affects the kaon topology by modifying the kaon tracking and energy reconstruction, impacting the basis for kaon identification in a LArTPC.  

The cross section analyses exploit the totality of LArIAT's experimental handles; they rely on beam line detector information as well as both calorimetry and tracking in the TPC. These analyses are LArIAT's first physics results. 
In order to measure total hadronic cross sections on argon, several steps are necessary. The analyses start by identifying a sample of the hadron of interest in the beam line and assessing the beam line contaminations. It proceeds with tracking the hadron candidates in the TPC and measuring their kinetic energy at each point in the tracking: the fine sampling of an hadron in the TPC forms the set of ``incident" hadrons.  Then, the hadronic interaction point is identified and the raw cross section is calculated via the ``thin slice method". Two corrections are then applied to the raw cross section -- a background subtractions and a correction for detector effects -- to obtain the cross section measurement, presented here.\\

This body of work is organized in 8 chapters.
We provide a description of the theoretical framework for the measurements in  Chapter \ref{ch:TheTheory}. Chapter \ref{ch:2} outlines the LArTPC detector technology, while
Chapter \ref{sec:experimentDescription} describes LArIAT experimental setup. We present the event selection for both the pion and kaon analyses, as well as the thin slice method in Chapter \ref{ch:Interactions}.  Chapter \ref{ch:samples}  describes the work done on the data and Monte Carlo samples in preparation of the cross section analyses.
Chapter \ref{ch:PionXS} shows  the results for the ($\pi^-$-Ar) total hadronic cross section measurement. Chapter \ref{ch:KaonXS} shows  the results for the ($K^+$-Ar) total hadronic cross section measurement. We draw the final remarks on this work in Chapter \ref{ch:Conclusions}

A series of additional studies and calibrations were necessary to perform the cross section analyses. Appendix \ref{ch:AppendixB} shows a measurement of the LArIAT LArTPC electric field using cosmic data. Appendix \ref{ch:AppendixTrack} shows an optimization of the tracking algorithms geared towards maximizing the efficiency of finding the hadronic interaction point. Appendix \ref{ch:energyCalibration} shows the calorimetry calibration of the LArIAT LArTPC, which is a pivotal measurement to enable any physics analysis with TPC data.  


