\chapter*{Introduction}\label{ch:-1}

This thesis work concerns the first measurement of the  ($\pi^-$-Ar)  total hadronic cross section  in the 100-1000 MeV  kinetic energy range and the first measurement of the ($K^+$-Ar) total hadronic cross section  in the 100-650 MeV  kinetic energy range. We performed these measurements at the LArIAT experiment,  a small (0.25 ton)  Liquid Argon Time Projection Chamber (LArTPC) on a beam of charged particles at the Fermilab Test Beam Facility.  Albeit particle and nuclear physics have a long history of hadronic cross section measurements, the work outlined in this thesis presents a new methodology -- the ``thin slice method" -- for cross section measurements, possible only thanks to the detection capabilities of the LArTPC technology. The LArTPC technology allows to see unprecedented details of particle interactions in argon given its full 3D-imaging with millimeter resolution and precise calorimetric reconstruction. A renewed interest for precision measurements of hadronic cross sections, particularly in argon, arises from the current  panorama of experimental  particle physics at the intensity frontier.

The discovery of the Higgs boson in 2012 marked the triumph of the Standard Model of Particle Physics; exploring what lays beyond is the real challenge in our field today. 
Since their formulation in 1930, neutrinos have been a source of surprises (and Nobel Prizes) for particle physicists, tiny cracks in our understanding of Nature. In particular, the discovery of neutrino oscillation represents the first evidence of physics Beyond the Standard Model (BSM).  From a theoretical point of view, the field is developing new theories to account for the small but non-zero mass of neutrinos, while trying to remain consistent with the rest of the Standard Model.  From an experimental point of view, we are developing technologies and huge collaborations to probe these theories. Even today, neutrinos might hold the key to the next generation of discoveries in particle physics.

According to the latest Particle Physics Project Prioritization Panel  \cite{P5}, the US particle physics panorama is directing a substantial effort towards the exploration of the intensity frontier. In particular, the near future will see the development of a Short Baseline Neutrino Program (SBN) and long baseline neutrino program with  DUNE as far detector. Both these programs are based on the Liquid Argon Time Projection Chamber (LArTPC) detector technology. The US liquid argon program has the potential to answer many of the fundamental open questions in particle physics today, such as: is there a fourth generation neutrino? is CP violated in the lepton sector? are there any additional symmetries? And, can we find an indication of Grand Unified Theories? 

The SBN program at Fermilab is tasked with conclusively addressing the existence of a fourth neutrino generation in the  $\Delta m^2= \Delta m^2_{14} \sim [0.1 - 10]$ eV$^2$ parameter space. The SBN program entails three surface LArTPCs positioned on the Booster Neutrino Beam at different distances from the neutrino production in oder to fully exploit  the L/E dependence of the oscillation pattern:  SBND (100 m from the decay pipe), MicroBooNE (450 m), and ICARUS (600 m). SBN will also perform an extensive 
program of neutrino cross section measurements, fundamental to abate systematics in the oscillation analyses in both SBN and DUNE.

DUNE has a vast neutrino and non-accelerator physics reach. For what it concerns neutrino physics, oscillation analyses in DUNE have the capability of solving the mass hierarchy and octant problem,  and discovering CP violation in the neutrino sector. Besides its neutrino program, DUNE can open an experimental window on Grand Unified Theories (GUTs). GUTs could potentially answer fundamental questions such as the existence of non-zero neutrino masses and matter-antimatter asymmetry, explaining some ``accidents" in the Standard Models, such as the exact cancellation of the  proton and the electron charge.   Directly probing GUTs at the unification energy scale is impossible by any foreseeable collider experiment. We then need an indirect proof such as baryon number violation, which is predicted by almost every GUT in the form of proton decay, bounded nucleon decay or $n-\bar n$ oscillations on long time-scales. Historically, the dominant technology used in these searches has been water Cherenkov detectors, with Super-Kamiokande setting all the current experimental limits on the decay lifetimes at the order of $\sim 10 ^{34}$ years. The DUNE far detector and its non-accelerator physics program is a interesting new actor on this stage.  LArTPCs can in fact complement nucleon decay searches in modes where water Cherenkov detectors are less sensitive, especially $p\rightarrow K^+\bar{\nu}$  \cite{Adams:2013qkq}.



\textcolor{red}{Brief history of nu detectors}

\begin{itemize}
%\item Challenges Beyond the SM
%\item Many in text generation of neutrino detector
\item Brief history of nu detectors
\item Calibration : LArIAT
\item The measurements
\end{itemize}


LArTPCs provide excellent electron/photon separation \cite{Acciarri:2016sli} lacking in Cherenkov detectors which can be leveraged to abate the photon background from neutral current interactions  in $\nu_e$ searches. LArTPCs share superb tracking capability with bubble chamber detectors, with several additional benefits. They are electronically read out and self triggered detectors; they provide full 3D-imaging with millimeter resolution, precise calorimetric reconstruction and excellent particle identification. Plus, female physicists are actually writing code and running experiments, not (only) staring at images \textcolor{red}{-- Too much? --} \cite{2006physics4152G}. 

The complexity of the LArTPC technology for neutrino detection is due to several reasons. Argon is a fairly heavy element, which means that nuclear effects play an important role in the looks of the interaction topology. Also, data on charged particle interaction in argon is scarce;  no measurement of hadronic cross sections for pions or kaons is available for argon.
Secondly, 

LArIAT \cite{Cavanna:2014iqa}, a small (0.25 ton) LArTPC on a beam of charged particles at the Fermilab Test Beam Facility, is performing precise cross section measurements of charged particles in argon. The LArIAT suite of cross section measurements is a critical set of results which will be employed by both the short- and long- baseline neutrino programs. 
This work is a measurement of the negative pion argon and the positive kaon-argon total hadronic differential cross sections. 

The ($\pi^-$-Ar) hadronic cross section is a fundamental input for neutrino detectors in liquid argon. In fact, reconstructing neutrino interactions at the GeV scale where pion production is abundant. 

The   total hadronic differential cross sections is of particular interest for proton decay searches in DUNE, whose most interesting proton decay channel is $p\rightarrow K^+ \bar \nu$.   


The cross section analyses exploit the totality of LArIAT's experimental handles; they rely on beam line detector information as well as both calorimetry and tracking in the TPC. These analyses are LArIAT's first physics results. 

The  ($K^+$-Ar) total hadronic differential cross section in LArIAT is particularly relevant for a high identification efficiency in the context of proton decay searches in DUNE in the  $p\rightarrow K^+\bar{\nu}$  channel. In fact, the kaon-argon cross section affects the kaon topology by modifying the kaon tracking and energy reconstruction, impacting the basis for kaon identification in a LArTPC.  




A series of additional studies and calibrations were necessary to perform the cross section analyses. Appendix \ref{ch:AppendixB} shows a measurement of the LArIAT LArTPC electric field using cosmic data. Appendix \ref{ch:AppendixB} shows an optimization of the tracking algorithms geared towards maximizing the efficiency of finding the hadronic interaction point. Appendix \ref{ch:AppendixB} shows the calorimetry calibration of the LArIAT LArTPC, which is a pivotal measurement to enable any physics analysis with TPC data.  











In order to measure the kaon-argon differential cross section, several steps were necessary. The analysis starts by identifying a sample of kaons in the beam line and assessing the beam line contaminations. It proceeds with tracking the kaon candidates in the TPC and measuring their calorimetry.  Then,  the hadronic interaction point is identified and distinguished from weak interaction cases. The last step of the analysis, which is the energy unfolding, is now ongoing. 
With  this measurement, I acquired extensive experience in tracking hadrons and measuring their energy inside the TPC. This entailed both an in-depth study of the tracking algorithms performances on data and a dive into the GEANT4 transportation models. Since no previous kaon measurement has been performed on argon, simulation packages such as GEANT4 currently interpolate the kaon transportation on (scarce) data from lighter and heavier nuclei. This is suboptimal for proton decay searches.  In fact, an extremely high signal detection efficiency, measured by testing identification algorithms on simulation, is extremely difficult to obtain if the simulation is not reliable. By adding the experimental data points for kaon-argon cross section to the simulation, my thesis analysis will cover this gap of knowledge.\\


This body of work is divided in 8 chapters.
We provide a description of the theoretical framework for the measurements in  Chapter \ref{ch:TheTheory}. Chapter \ref{ch:2} outlines the LArTPC detector technology, while
Chapter \ref{sec:experimentDescription} describes LArIAT experimental setup. We present the event selection for both the pion and kaon analyses, as well as the ``thin-slice method" in Chapter \ref{ch:Interactions}.  Chapter \ref{ch:samples}  describes the work done on the data and Monte Carlo samples in preparation of the cross section analyses.
Chapter \ref{ch:PionXS} shows  the results for the ($\pi^-$-Ar) total hadronic cross section measurement. Chapter \ref{ch:KaonXS} shows  the results for the ($K^+$-Ar) total hadronic cross section measurement. We draw the final remarks on this work in Chapter \ref{ch:Conclusions}
