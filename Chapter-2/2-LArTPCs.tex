\chapter{Liquid Argon Detectors at the Intensity Frontier}\label{ch:1}


In the next few years, LArTPC experiments -- such as the Short-Baseline Neutrino program (SBN) and DUNE -- will be major players in the intensity frontier field. 


\section{Liquid Argon Time Projection Chambers at the Intensity Frontier}

%Bubble-chamber experiments played a key role in probing the properties of ?-interactions. The Liquid Argon Time Projection Chambers (LArTPC) technology,  first proposed by C.Rubbia in 1977 with ICARUS project [14], is considered the modern evolution of bubble-camber concept, with the additional features of three-dimensional event reconstruction, high-resolution calorimetry, active mass coincident with detector sensitive mass and can intrinsically supply a trigger signal (self-triggering) by means of the scintillation light produced in the liquid noble gas. This technology is ideal to perform $\nu$-studies in a broad energy range, from MeV up to few GeV, with high event reconstruction efficiency, thanks to the capability of particle identifcation and detailed reconstruction of different interaction topologies. In Figure 1.4 is shown a neutrino interaction event, producing a proton, a pion and a muon, as seen in a bubble chamber and in a LArTPC.


\subsection{Time Projection Chamber}
\subsection{Ionization Detectors with Noble Liquids}
\subsection{LArTPC: Principles of Operation}
\subsection{Liquid Argon Ionization Charge Detection}
\subsubsection{Electron Life Time \& purity}
\subsubsection{Space Charge Effect}
\subsubsection{Recombination Effect}
\subsection{Liquid Argon scintillation Light Detection}
\subsubsection{LAr Scintillation Process}
\subsubsection{Wavelength Shifting of LAr Scintillation Light}
\section{The SBN Program: Neutrino Interaction and Detection}
%\subsection{SBN Goals}
%\subsection{Neutrino Interactions and Detection }
\section{DUNE: Rare Decay Searches}
%\subsection{Non-Accelerator Physics Program}
%\subsection{Rare Decay Searches: Experimental Limit}
%\subsection{Nucleon Decay Detection in LAr}
\section{Enabling the next generation of discoveries: LArIAT}
LArIAT, a small Liquid Argon Time Projection Chamber (LArTPC) in a test beam,  is designed to perform an extensive physics campaign centered on charged particle cross section measurements while characterizing the detector performance for future LArTPCs. LArTPC represents one of the most advanced experimental technologies for physics at the Intensity Frontier due to its full 3D-imaging, excellent particle identification and precise calorimetric energy reconstruction. This complex technology however needs a thorough calibration and dedicated measurements of some key quantities to achieve the precision required for the next generation of discoveries at the Intensity Frontier which LArIAT can provide. 

The LArIAT LArTPC is deployed in a dedicated calibration test beamline at Fermilab.
We use the LArIAT beamline to characterize the charge particles before they enter the TPC: the particle type and initial momentum is known from beamline information. The precise calorimetric energy reconstruction of the LArTPC technology enables the measurement of the total differential cross section for  tagged hadrons. 
The Pion-Nucleus and Kaon-Nucleus total hadronic interaction cross section have never been measured before in argon and they are a fundamental step to shed light on light meson interaction in nuclei. Additionally, these measures provides a key input to neutrino physics and proton decay studies in future LArTPC experiments like SBN and DUNE.
\textcolor{red}{add paragraph on all wonderful things lariat can do... some event displays would be nice!}
