\chapter{Liquid Argon Detectors at the Intensity Frontier}\label{ch:2}


In the next few years, LArTPC experiments -- such as the Short-Baseline Neutrino program (SBN) and DUNE -- will be major players in the intensity frontier field. 


\section{Liquid Argon Time Projection Chambers at the Intensity Frontier}


\subsection{Time Projection Chamber}
\subsection{Ionization Detectors with Noble Liquids}
\subsection{LArTPC: Principles of Operation}
\subsection{Liquid Argon Ionization Charge Detection}
\subsubsection{Electron Life Time \& purity}
\subsubsection{Space Charge Effect}
\subsubsection{Recombination Effect}
\subsection{Liquid Argon scintillation Light Detection}
\subsubsection{LAr Scintillation Process}
Liquid argon emits scintillation light at the passage of charge particle. The light yield depends on the argon purity, the electric field, the dE/dx and particle type, averaging at the tens of thousands of photons per MeV. Light emission peaks in the ultraviolet at a 128 nm, shown in comparison to Xenon and Kypton in Figure  [183]. 

The de-excitation of Rydberg dimers in the argon is responsible for the scintillation light.
Rydberg dimers exist in two states:  a singlet and a triplet. The time constant for the singlet radiative decay is 6 ns, resulting in a prompt component for the scintillation light. The decay of the triplet is delayed by intersystem crossing, producing a slow component with a time constant of $\sim$~1500 ns.  ``Self-trapped exciton luminescence" and  ``recombination luminescence" are the two processes responsible for the creation of the Rydberg dimers. In the first process, a charged particle excites an argon atom which becomes self-trapped in the surrounding bulk of argon,  forming a dimer; the dimer is in the singlet state 65\% of the times and in the triplet state 35\% of the times. In case of recombination luminescence, the charged particle transfers enough energy to ionize the argon. The argon form a charged argon dimer state, which quickly recombines with the thermalized free electron cloud. Excimer states are produced in the recombination, roughly half in the singlet and half in the triplet state. The light yield dependency on the electric field, on the dEdx and particle type derives from the role of free charge in the recombination luminescence process. The separation between the argon ions and the free electron cloud depend on the electric field. On one hand, a strong electric field diminishes the recombination probability, leading to a smaller light yield; on the other, it increases the free charge drifting towards the anode plane. Hence, charge and light as a function of the electric field are anti-correlated.


%The recombination rate is proportional to the local densities of both free electrons and
%positive ions, so the square of the local ionization density. Recombination is therefore
%enhanced for highly ionizing particles [102]. This results in a characteristic dE/dx
%dependence in the time profile of scintillation light [106], which can be used for pulseshape-based particle identification [186], [187].

%Although attempts to measure the absorption length of pure argon at 128 nm
%have been hindered by the presence of impurities [188], it is expected to be much
%longer than the length scale of existing or proposed LArTPCs. 

The production mechanism through emission from bound excimer states implies that argon is  transparent  to its own scintillation light: the emitted photons are not energetic enough to re-excite the argon bulk.
%Scattering processes, on the other hand, do occur on length scales comparable to
%the size of existing detectors. In particular, Rayleigh scattering of 128 nm light
%from thermal density fluctuations in the liquid has a predicted scattering length of
%approximately 90 cm [190]. Scattering causes the light propagating between a source
%and optical detector to take a longer path than the direct line of sight, and affects the
%arrival times and directions of detected photons. This is particularly important for ??0
%determination in a LArPC. A direct measurement of the Rayleigh scattering length
%has been made [191] and a value 66 cm was obtained, shorter than the theoretical
%prediction.



\subsubsection{Wavelength Shifting of LAr Scintillation Light}
\subsection{Signal processing}
\section{The SBN Program: Neutrino Interaction and Detection}
%\subsection{SBN Goals}
%\subsection{Neutrino Interactions and Detection }
\section{DUNE: Rare Decay Searches}
The key elements for a rare decay experiment are: massive active volume, long exposure, high identification efficiency and low background. 
%The limit to proton lifetime in case of absence of signal and backgrounds is set by calculating
%$$\tau/B > M\times \epsilon\times T \times 10^{32},$$ 
%where M is the detector mass in kton, $\epsilon$ the signal detection efficiency after cuts to suppress backgrounds (dependent on the considered decay mode), T is the exposure in years, B the assumed branching fraction for the considered mode and  $10^{32}$ is a factor accounting for the number of nucleons in a kton of material \cite{Bueno2007}.
Figure \ref{fig:PDKExperimentalLImit} shows the current best experimental limits on nucleon decay lifetime over branching ratio (dots). Historically, the dominant technology used in these searches has been water Cherenkov detectors: all the best experimental limits on every decay mode are indeed set by Super-Kamiokande \cite{PhysRevD.90.072005,PhysRevLett.115.121803}.  It is particularly important to notice that the kaon energy for the proton decay mode $p \rightarrow K^+ \bar{\nu}$ is under Cherenkov threshold.  Super-Kamiokande set the limit on the lifetime for the $p \rightarrow K^+ \bar{\nu}$ mode by  relying exclusively on photons from nuclear de-excitation. For this reason, an attractive alternative approach to identifying nucleon decay is the use of a Liquid Argon Time Projection Chamber (LArTPC). 

LArTPCs can complement nucleon decay searches in modes where water Cherenkov detectors are less sensitive, especially $p\rightarrow K^+\bar{\nu}$. According to \cite{Acciarri:Dune}, DUNE will have an active volume large enough, have sufficient shielding from the surface, and will run for lengths of time sufficient to compete with Hyper-K, opening up the opportunity for the discovery of nucleon decay. 

\begin{figure}[hbpt]
\centering
\includegraphics[width=6.5in]{Chapter-2/Images/PDKExperimentalLImit.png}
\caption{Proton decay lifetime limits from passed and future experiments.}
\label{fig:PDKExperimentalLImit}
\end{figure}


\begin{figure}[hbpt]
\centering
\includegraphics[width=3.5in]{Chapter-2/Images/pdkGenie.png}
\caption{Momentum of the kaon outgoing a proton decay event as simulated by the Genie 2.8.10 event generator in argon. The red line represent the kaon momentum distribution before undergoing the simulated final state interaction inside the argon nucleus, while the blue line represents the momentum distribution after FSI. }
\label{fig:PDKGENIE}
\end{figure}


%\subsection{Non-Accelerator Physics Program}
%\subsection{Rare Decay Searches: Experimental Limit}
%\subsection{Nucleon Decay Detection in LAr}
\section{Enabling the next generation of discoveries: LArIAT}
LArIAT, a small Liquid Argon Time Projection Chamber (LArTPC) in a test beam,  is designed to perform an extensive physics campaign centered on charged particle cross section measurements while characterizing the detector performance for future LArTPCs. LArTPC represents one of the most advanced experimental technologies for physics at the Intensity Frontier due to its full 3D-imaging, excellent particle identification and precise calorimetric energy reconstruction. This complex technology however needs a thorough calibration and dedicated measurements of some key quantities to achieve the precision required for the next generation of discoveries at the Intensity Frontier which LArIAT can provide. 

The LArIAT LArTPC is deployed in a dedicated calibration test beamline at Fermilab.
We use the LArIAT beamline to characterize the charge particles before they enter the TPC: the particle type and initial momentum is known from beamline information. The precise calorimetric energy reconstruction of the LArTPC technology enables the measurement of the total differential cross section for  tagged hadrons. 
The Pion-Nucleus and Kaon-Nucleus total hadronic interaction cross section have never been measured before in argon and they are a fundamental step to shed light on light meson interaction in nuclei. Additionally, these measures provides a key input to neutrino physics and proton decay studies in future LArTPC experiments like SBN and DUNE.
\textcolor{red}{add paragraph on all wonderful things lariat can do... some event displays would be nice!}



\textcolor{red}{ADD genie proton decay kaon distribution and lariat beamline overlaied}
The signature of a proton decay event in the ``LAr golden mode" is the presence of a single kaon of about 400 MeV in the detector. 
