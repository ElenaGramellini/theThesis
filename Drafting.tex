

%%%%%%%%%%%%%%%%%%%%%%%%%%%%%%%%%%%%%%
% yale_thesis.tex
% Elena Gramellini
% Started: 2017/12/05
%
% A bare, sample template for a Yale PhD thesis using yalephd.cls
%%%%%%%%%%%%%%%%%%%%%%%%%%%%%%%%%%%%%%

\documentclass[letterpaper,12pt]{yalephd}
% remove draft option for final printing.
% font size must be between 10pt-12pt.
\usepackage{caption,setspace}
\usepackage{amsmath} 
\usepackage{rotating}
%\usepackage{newtxtext,newtxmath,amsmath}
\usepackage{ textcomp }


\usepackage{geometry} % you need this for yalephd.cls to work.
\usepackage{graphicx}
\usepackage[graphicx]{realboxes}
 % you probably want the rest of these.
\usepackage{dcolumn}
\usepackage{bm}
\usepackage{hyperref}
\usepackage[export]{adjustbox}

\usepackage{mathtools}
\DeclarePairedDelimiter\bra{\langle}{\rvert}
\DeclarePairedDelimiter\ket{\lvert}{\rangle}
\DeclarePairedDelimiterX\braket[2]{\langle}{\rangle}{#1 \delimsize\vert #2}
\usepackage{textgreek}

\usepackage{multirow}
\usepackage{amsmath}
\usepackage{amsfonts}
\usepackage{amssymb}
\usepackage{appendix}
\usepackage{comment}
\usepackage{cite}
\usepackage{color}
\bibliographystyle{plain}
\usepackage{slashed}  

%\setcounter{chapter}{-1}

\newenvironment{dedication}
  {%\clearpage           % we want a new page          %% I commented this
   \thispagestyle{empty}% no header and footer
   \vspace*{\stretch{1}}% some space at the top
   \itshape             % the text is in italics
   \raggedleft          % flush to the right margin
  }
  {\par % end the paragraph
   \vspace{\stretch{3}} % space at bottom is three times that at the top
   \clearpage           % finish off the page
  }

\usepackage{lineno,xcolor}


%\linenumbers
\begin{document}
\begin{equation}
E^{\text{kin}}_{\text{Front Face}} = \sqrt{p_{Beam}^2 + m_{Beam}^2} - m_{Beam}^2 - E_{Loss} 
\end{equation}



\begin{document}

\begin{equation}
E^\text{kin}_j = E^{\text{kin}}_{\text{Front Face}} - \sum_{j<i} E_{\text{dep i}} 
\end{equation}


\begin{equation}
E_{i} = \sqrt{p_{Beam}^2 + m_{Beam}^2} - m_{Beam}^2 - E_{Loss} - E_{\text{dep FF-i}}
\end{equation}

\begin{equation}
\delta E_{i} = \sqrt{\delta p_{Beam}^2 + \delta E_{Loss}^2 +  \delta  E_{\text{dep FF-i}}^2}
\end{equation}

\begin{equation}
E_{\text{dep FF-i}} = \sum_{j<i} E_{\text{dep i}}  \Rightarrow \delta E_{\text{dep FF-i}} = (i -1) \delta E_{\text{dep i}}
\end{equation}


\begin{equation}
      \sigma^{\pi^-}_{TOT}(E_{i})  = \frac{1}{n\text{ } \delta X}\frac{ \epsilon^{\text{Inc}}(E_i)  \hspace{0.2cm} C^{\pi MC}_{\text{Int}} (E_{i}) \hspace{0.2cm} N^{\text{TOT}}_{\text{Int}} (E_{i}) }{   \epsilon^{\text{Int}}(E_i) \hspace{0.2cm} C^{\pi MC}_{\text{Inc}} (E_{i}) \hspace{0.2cm}  N^{\text{TOT}}_{\text{Inc}} (E_{i})}.
\label{eq:C}
\end{equation}

\begin{equation}
      \sigma^{K^+}_{TOT}(E_{i})  = \frac{1}{n\text{ } \delta X}\frac{ \epsilon^{\text{Inc}}(E_i)  \hspace{0.2cm} C^{K MC}_{\text{Int}} (E_{i}) \hspace{0.2cm} N^{\text{TOT}}_{\text{Int}} (E_{i}) }{   \epsilon^{\text{Int}}(E_i) \hspace{0.2cm} C^{K MC}_{\text{Inc}} (E_{i}) \hspace{0.2cm}  N^{\text{TOT}}_{\text{Inc}} (E_{i})}.
\label{eq:C}
\end{equation}


\begin{equation}
\mathcal{L}(\mu_0; \sigma^2_0; \Delta\theta_0, \Delta\theta_1) = \prod^1_{i=0} f_X(\Delta\theta_i, \mu_0, \sigma^2_0 ) \Rightarrow 
\end{equation}

\begin{equation}\log{\mathcal{L}} = -\frac{1}{2}\log(2\pi) - \log{\sigma_0} - \frac{1}{2}\frac{(\Delta\theta_0 - \mu_0)^2}{\sigma^2_0} + \text{same for $\Delta\theta_1$}
\end{equation}


\begin{table}[]
\centering
\begin{tabular}{|l|c|c|}
\hline
                     & I = -60 A           & I = -100 A \\ \hline
G4Pions       &   70.0 $\pm$ 3.0 \%           &       81.9 $\pm$ 2.1 \%        \\ \hline
G4Muons     &   15.1 $\pm$ 1.0 \%           &      13.4 $\pm$ 0.6 \%         \\ \hline
G4Electrons &   14.9 $\pm$ 1.0 \%           &       4.7 $\pm$ 0.3 \%        \\ \hline
\end{tabular}
\caption{Simulated beamline composition per magnet settings}
\label{tab:beamline}
\end{table}



\begin{table}[b]
\centering
\begin{tabular}{|l|c|}
\hline
                                                        & Run-II Neg Pol     \\ \hline
1. Events Reconstructed in Beamline        &  158396    \\ \hline
2. Events with Plausible Trajectory            &   147468   \\ \hline
3. Beamline $\pi^-/\mu^-/e^-$  Candidate  &   138481    \\ \hline
4. Events Surviving Pile Up Filter              &   108929        \\ \hline
5. Events with WC2TPC Match                 &    41757   \\ \hline
6. Events Surviving Shower Filter             &    40841       \\ \hline
7. Available Events For Cross Section      &   40841       \\ \hline
\end{tabular}
\caption{Number of data events for Run-II Negative and Positive polarity }
\label{tab:beamlineDataSelection}
\end{table}


\end{document}
