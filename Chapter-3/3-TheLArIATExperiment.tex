% This chapter to do:
% Re-write the red parts
% write DAQ part
% write collimators description
% put references
% re-read

\chapter{LArIAT: Liquid Argon In A Testbeam}\label{sec:experimentDescription}
In this chapter, we describe the LArIAT experimental setup. We start by illustrating the journey of the charge particles in the Fermilab accelerator complex, from the gaseous thermal hydrogen at the Fermilab ion source to the delivery of the LArIAT tertiary beam at MC7. We  then describe the LArIAT beamline detectors, the LArTPC, the DAQ and the monitoring system.

%\section{LArIAT \& the Intensity Frontier}
\section{The Particles Path to LArIAT}

LArIAT's particles history begins in the Fermilab accelerator complex with a beam of protons. The process of protons acceleration develops in gradual stages (see picture \ref{fig:Accelerator}): gaseous hydrogen is ionized in order to form H$^{-}$ ions; these ions are boosted to 750 keV by a Cockroft-Walton accelerator and injected to the Linac linear accelerator that increases their energy up to 400 MeV; then, H$^{-}$ ions pass through a carbon foil and lose the two electrons; the resulting protons are then injected into a rapid cycling synchrotron, called Booster; at this stage, protons reach 8 GeV of energy and are compacted into bunches; the next stage of acceleration is the Main Injector, a synchrotron which accelerates the bunches up to 120 GeV; in the Main Injector, several bunches are merged into one and used for the injection in the last stage.


The Fermilab accelerator complex works in supercycles of roughly 60 seconds in duration. The beam is split by electrostatic septa and delivered at different experimental halls all over the lab. A 120~GeV$/c$ primary proton beam with variable intensity is extracted in four-second ``spills" and sent to the Meson Center beam line. 

LArIAT's home at Fermilab is the Fermilab Test Beam Facility (FTBF), where the experiment characterizes a beam of charge particles downstream from the Meson Center beam line. 
Here, the primary beam is focused onto a tungsten target to create LArIAT's secondary beam. The composition of the secondary particle beam is mainly positive pions. The momentum peak of the secondary beam was fixed at 64~GeV/c for the LArIAT data considered in this work, although the beam is tunable in momentum between 8-80\,GeV/c; this configuration of the secondary beamline assured a stable beam delivery at the LArIAT experimental hall.
 
The secondary beam impinges then on a copper target within a steel collimator inside the LArIAT experimental hall (MC7) to create the LArIAT tertiary beam, (shown in  Fig.~\ref{fig:tert-layout}).   The steel collimator selects particles produced with a $13^\circ$ production angle at the target down the beamline.  The particles are then bent by  $~10^\circ$  through a pair of dipole magnets.  By configuring the field intensity of the magnets we allow the particles of LArIAT's tertiary beam to span a momentum range from 0.2 to 1.4~GeV/c. The polarity of the magnet is also configurable and determines the sign of the beamline particles which are focused on the LArTPC. If the magnets polarity is positive the tertiary beam composition counts mostly pions and protons with a small fraction of electrons, muons, and kaons. It is the job of the LArIAT beamline detectors to select the particles polarity,  to perform particle identification (beamPID) and to measure the momentum of the tertiary beam particles before they get to the LArTPC. The LArIAT detectors are described in the following paragraphs.  



%\begin{comment}     
\begin{figure}
  \centering  	
\includegraphics[width=\textwidth,height=\textheight,keepaspectratio]{Chapter-3/Images/AcceleratorFNAL.png}
\caption{Layout of Fermilab Acellerator complex.}
\label{fig:Accelerator}
\end{figure}

%\begin{comment}     
\begin{figure}
  \centering  	
\includegraphics[width=\textwidth,height=\textheight,keepaspectratio]{Chapter-3/Images/Tertiary.png}
\caption{Bird's eye view of the LArIAT tertiary beamline. In grey: upstream and downstream collimators; in yellow: bending magnets; in red: wire chambers; in blue: time of flight; in green: liquid argon TPC volume; in maroon: muon range statck.}
\label{fig:tert-layout}
\end{figure}


%%%%%%%%%%%%%%%%%%%%%%%%%%%%%%%%%%%%%%%%%%%%%%%%%%%%%%%%%%%%
\section{LArIAT Tertiary Beam Instrumentation}\label{sec:Instrumentation}

%%%%%%%%%%%%%%%%%%%%%%%%%%%%%%%%%%%%%%%%%%%%%%%%%%%%%%%%%%%%
The instrumentation of  LArIAT tertiary beam and the TPC components have changed several times during the three years of LArIAT data taking. The following paragraphs describe the components operational during ``Run II", the data taking period relevant to the hadron cross section measurements.

The key components of the tertiary beamline instrumentation for the hadron cross section analyses are the two bending magnets, a set of four wire chambers (WCs) and two time-of-flight scintillating paddles (TOF) and, of course, the LArTPC.  The magnets determine the polarity of the particles in the tertiary beam; the combination of magnets and wire chambers determines the particles' momentum, which is used to determine the particle species in conjunction with the TOF.
A muon range stack downstream from the TPC and two sets of cosmic paddles configured as a telescope surrounding the TPC are also used for calibration purposes.


\subsection{Bending Magnets}\label{sec:Magnets}
%%%%%%%%%%%%%%%%%%%%%%%%%%%%%%%%%%%%%%%%%%%%%%%%%%%%%%%%%%%

LArIAT uses a pair of identical Fermilab type ``NDB" electromagnets, recycled from the Tevatron's anti-proton ring, in a similar configuration used for the  MINERvA T-977 test beam calibration~\cite{MinervaTestbeam}). 
The magnets are a fundamental piece of the LArIAT beamline equipment, as they are used for both particle identification and momentum measurement before the LArTPC. The sign of the current in the magnets allows us to select either positively or negatively charged particles; the value of the magnetic field is used in the momentum determination and in the subsequent particle identification. 

We describe here the characteristics and response of one magnet, as the second one has a similar response, given its identical shape and history. Each magnet is a box with a rectangular aperture gap in the center to allow for the particle passage.  The magnet aperture measures 14.224~cm in height, 31.75~cm in width, and  46.67~cm in length.  Since the wire chambers aperture ($\sim$12.8~cm$^2$) is smaller than the magnet aperture, only the central part of the magnet gap is utilized. The field is extremely uniform over this limited aperture and was measured with two hall probes, both calibrated with nuclear magnetic resonance probes. The probes measured the excitation curve shown in Figure~\ref{fig:magnet_excitation}. 

\begin{figure}[!h]
\begin{centering}
\vspace{-0.3cm}
\includegraphics[height=3.0in]{Chapter-3/Images/ExcitationCurves.png}
\caption{
{ Magnetic field over current as a function of the current, for one NDB magnet (excitation curve). The data was collected using two Hall probes (blue and green). We fit the readings with a cubic function (black) to average of measurements (red) given in the legend.}
}
\label{fig:magnet_excitation}
\end{centering}
\end{figure}

The current through the magnets at a given time is identical in both magnets. For the Run II data taking period, the current settings explored were 60A (B $\sim$0.21 T) and 100A (B $\sim$0.35 T) in both polarities. 
Albeit advantageous to enrich the tertiary beam composition with high mass particles such as kaons, we never pushed the magnets current over 100 A, not to incur in overheating.  During operation, we operated a air and water cooling system on the magnets and we remotely monitored the magnets temperature.
 
\subsection{Multi-Wire Proportional Chambers}\label{sec:MWPC}
%%%%%%%%%%%%%%%%%%%%%%%%%%%%%%%%%%%%%%%%%%%%%%%%%%%%%%%%%%%%
\begin{figure}[!h]
\begin{centering}
\vspace{-0.3cm}
\includegraphics[height=2.3in]{Chapter-3/Images/WireChamber.png}
\caption{
{One of the four Multi Wire Proportional Chambers (WC) used in the LArIAT tertiary beamline.}
}
\label{fig:wirechamber}
\end{centering}
\end{figure}

LArIAT uses four multi-wire proportional chambers, or wire chambers (WC) for short, two upstream and two downstream from the bending magnets. The geometry of one chamber is shown in Figure~\ref{fig:wirechamber}: the WC effective aperture is a square of  12.8~cm perpendicular to the beam direction.  Inside the chamber, the 128 horizontal and 128 vertical wires hang at a distance of 1~mm from each other in a mixture of 85\% Argon and 15\% isobutane gas.  The WC operating voltage is between 2400~V and 2500~V. The LArIAT wire chambers are an upgraded version of the Fenker Chambers~\cite{Fenker}, where an extra grounding improves the signal to noise ratio of the electronic readout.  

Two ASDQ chips~\cite{ASDQchip} mounted on a mother board plugged into the chamber serve as front end amplifier/discriminator. The chips are connected to a multi-hit TDC~\cite{Sten} which provides a fast OR output used as first level trigger. The TDC time resolution is 1.18~ns/bin and can accept 2 edges per 9~ns.  
The maximum event rate acceptable by the chamber system is of 1 MHz: this rate is not a limiting factor considering that \textcolor{red}{the rate of the tertiary particle beam at the first wire chamber is estimated to be less than 15 kHz}. A full spill of data occurring once per supercycle is stored on the TDC board memory at once and read out by a specially designed controller.  We use LVDS cables to carry both power and data between the controller and the TDCs and from the controller to the rest of the DAQ.  
%It is possible to program the time window for acceptance for hits, time offsets, front end threshold, and pulse shaping parameters through the controller via a USB from a PC or through an Ethernet connection.

\subsubsection{Multi-Wire Proportional Chambers functionality}\label{sec:MWPCfunc}
We use the wire chamber system together with the bending magnets to measure the particle's momentum.

In the simplest scenario, only one hit on each and every of the four wire chambers is recorded during a single readout of the detector systems.  Thus, we use the hit positions in the two wire chambers upstream of the magnets to form a trajectory before the bend, and the hit positions in the two wire chambers downstream of the magnets to form a trajectory after the bend. We use the angles in the XZ plane between the upstream and downstream trajectories  to calculate the $Z$ component of the momentum as follows:

\begin{equation}
P_z=\frac{B_{eff}L_{eff}}{3.3(sin(\theta_{DS})-sin(\theta_{US}))},
\label{eq:momformula}
\end{equation}

where $B_{eff}$ is the effective maximum field in a square field approximation,  $L_{eff}$ is the effective length of both magnets (twice the effective length of one magnet), $\theta_{US}$ is the angle off the $z$ axis of the upstream trajectory, $\theta_{DS}$ is the angle off the $z$ axis of the downstream trajectory  and  3.3~$c^{-1}$ is the conversion factor from [T$\cdot$m] to [MeV/c]. By using the hit positions on the third and fourth wire chamber, we estimate the azimutal and polar angles of the particle trajectory, and we are able to calculate the other components of the momentum. 

The presence of multiple hits in a single wire chamber or the absence of hits in one (or more) wire chambers can complicate this simple scenario. The first complication is due to beam pile up, while the latter is due to wire chamber inefficiency. In the case of multiple hits on a single WC, at most one wire chamber track is reconstructed per event. Since the magnets bend particles only in the X direction, we assume the particle trajectory to be roughly constant in the YZ plane, thus we keep the combination of hits which fit best with a straight line. 
It is still possible to reconstruct the particle's momentum  even if the information is missing in either of the two middle wire chambers (WC2 or WC3), by constraining the particle trajectory to cross the plane in between the magnets. 
%Under the assumption of identical magnets, we define a plane centered in the middle of the two magnets (called ``midplane") that the physical particles need to cross. We project the completed half of the wire chamber track to the midplane, assuming no bending, to find the point of intersection. We use this point to complete the other half of the wire chamber track and to calculate the reconstructed momentum  with these four points. To account for the lack of bending in our reconstructed wire chamber track, we apply to the calculated momentum  a correction obtained with a sample of 4-point, single hit tracks.

Events satisfying the simplest scenario of one single hit in each of the four wire chambers form the ``Picky Track" sample.  We construct another, higher statistics sample, where we loosen the requirements on single hit and wire chamber efficiency: the ``High Yield" sample. For LArIAT Run II, the High Yield sample is about three times the Picky Tracks statistics.  For the first measurements of the LArIAT hadronic cross section, we use the Picky Tracks sample because the uncertainty on the momentum is smaller and the comparison with the beamline MC results is straightforward compared with the High Yield sample;  a possible future update and cross check of these analysis would be the use of the High Yield sample. 

%We use Picky Tracks to calibrate the momentum measurement for the High Yield sample, in particular to obtain a momentum correction for  tracks missing information from the central WCs;  this correction adds an uncertainty of approximately 2\% to the momentum calculation of the High Yield sample.

\textcolor{red}{Four point track momentum uncertainty}

\subsection{Time-of-Flight System}\label{sec:TOF}
%%%%%%%%%%%%%%%%%%%%%%%%%%%%%%%%%%%%%%%%%%%%%%%%%%%%%%%%%%%%
Two scintillator paddles, one upstream to the first set of WCs and one downstream to the second set of WCs  form LArIAT  time-of-flight (TOF) detector system. 

The upstream paddle is made of a 10 x 6 x 1~cm scintillator piece, read out by two PMTs mounted on the beam left side which collect the light from light guides mounted on all four edges of the scintillator. The downstream paddle is a   14 x 14 x 1~cm scintillator piece read out by two PMTs on the opposite ends of the scintillator.
The relatively thin width on the beamline direction minimizes energy loss of the particles coming from the target in the scintillator material.

%\begin{figure}[!h]
%\begin{centering}
%\vspace{-0.3cm}
%\includegraphics[height=2.3in]{Chapter-3/Images/tofdelay.png}
%\caption{
%{\scriptsize \sf Pictures of the TOF system as was deployed during Run-I and Run-II data taking. The left image is of the upstream TOF paddle and the right image is of the downstream TOF paddle }
%}
%\label{fig:TOFSystemRunIandII}
%\end{centering}
%\end{figure}



The CAEN 1751 digitizer is used to digitize the TOF PMTs signals at a sampling rate of 1 GHz. The 12 bit samples are stored in a circular memory buffer. At trigger time, data from the TOF PMTs are recorded to output in a 28.7 \textmu s windows starting  approximately 8.4 \textmu s before the trigger time. 



\subsubsection{TOF functionality}\label{sec:MWPCfunc}


The TOF signals rise time (10-90\%) is 4 ns and a full width, half-maximum of 9 ns consistent in time. The signal amplitudes from the upstream TOF and  downstream TOF are slightly different:  200 mV for the upstream PMTs but only 50 mV for downstream PMTs. The time of the pulses was calculated utilizing an oversampled template derived from the data itself. We take the pulse pedestal from samples far from the pulse and subtract it to the pulse amplitude. We then stretch vertically a template to match the pedestal-subtracted pulse amplitude and we move it horizontally to find the time. With this technique, we find a pulse time-pickoff resolution better than 100 ps.  The pulse pile up is not a significant problem given the TOF timing resolution and the rate of the particle beam.  Leveraging on the pulses width uniformity of any given PMT (sigma of 400 ps),  we flag events where two pulses overlap as closely in time as 4 ns with an 90\% efficiency according to simulation. 


We combine the pulses from the two PMTs on each paddle to determine the particles' arrival time by averaging the time measured from the single PMT, so to minimize errors due to optical path differences in the scintillator.  However, a time spread of approximately 300~ps is present in both the upstream and downstream detectors, likely due to transit time jitter in the PMTs themselves.  There is no evidence of systematic timing drift over long data-taking periods such as 3-4 months: the maximum variation of the average time differences between pairs of PMTs reading out the same scintillator is of the order of 150~ps.

\textcolor{red}{calculated TOF with error}


%%%%%%%%%%%%%%%%%%%%%%%%%%%%%%%%%%%%%%%%%%%%%%%%%%%%%%%%%%%%
\subsection{Punch-Through and Muon Range Stack Instruments}\label{sec:MuRS}
%%%%%%%%%%%%%%%%%%%%%%%%%%%%%%%%%%%%%%%%%%%%%%%%%%%%%%%%%%%%

The punch-thorough and the muon range stack (MuRS) detectors are located downstream of the TPC. These detectors provide a sample of  TPC crossing tracks without relying on TPC information and can be used to improve particle ID for  muons and pions with momentum higher than  450 MeV/c.

The punch-thorough is simply a \textcolor{red}{? x ? x ?}~cm scintillator piece, read out by \textcolor{red}{ two? Hamamatsu? 2? inch} PMTs. 
The MuRS is a segmented block of steel with four slots instrumented with scintillation bars. The four steel layers in front of each instrumented slot are 2 cm, 2 cm, 14 cm and 16 cm wide in the beam direction. Each instrumented slot is equipped with four scintillation bars each, positioned vertically in the direction orthogonal to the beam. Each scintillator bar measures  \textcolor{red}{? x ? x 2}~cm and it is read out by  \textcolor{red}{ two? Hamamatsu? 2? inch} PMTs.  

The signals from both the punch-thorough and the MuRS PMTs are digitized in the CAEN V1740, same as the TPC; the details of this discriminator are laid out in~\ref{sec:TPCCharge}. It is worth noticing that the sampling time of the CAEN V1740 is slow (of the order of 128 ns), so pulse shape information from the PMT is lost.
Punch-thorough and MuRS hits are formed utilizing the OR between the PMTs digital discriminator signals under threshold at a given time, where we obtain the threshold for each PMT directly on data distributions.



%%%%%%%%%%%%%%%%%%%%%%%%%%%%%%%%%%%%%%%%%%%%%%%%%%%%%%%%%%%%
\subsection{LArIAT Cosmic Ray Paddle Detectors}\label{sec:CosmicRayPaddle}
%%%%%%%%%%%%%%%%%%%%%%%%%%%%%%%%%%%%%%%%%%%%%%%%%%%%%%%%%%%%
Besides on beam data, LArIAT also triggers on cosmic rays events by using two sets of cosmic ray paddle detectors (a.k.a. ``cosmic towers".) The cosmic towers frame the LArIAT cryostat, as one sits in the downstream left corner and the other sits in the upstream right corner of the cryostat. Two paddle sets of four scintillators pieces each, an upper and a lower set, make up each cosmic tower. 
Of the four paddles, a couple of two matched paddles stands upright while the a second matched pair lies across the top of the assembly in the top sets (or across the bottom of the assembly in the bottom sets). The horizontal couple is used as a veto for particles traveling from the TPC out.  The four signals  from the vertical paddles along one of the body diagonals of the TPC are combined in a logical ``AND''. This allows to select cosmic muons crossing the TPC along one of its diagonals.  Cosmic ray tracks crossing both anode and cathode populate the events triggered this way. This particularly useful sample of tracks (which we can safely assume to be associated with ~5 GeV muons MIPs) can be used for many tasks; for example, we use anode-cathode piercing tracks to cross check the TPC electric field on data (see \ref{ch:AppendixC}), to calibrate the charge response of the TPC wires for the full TPC volume and to measure the electron lifetime in the chamber \textcolor{red}{ ADD reference to different chapters}.

%%%%%%%%%%%%%%%%%%%%%%%%%%%%%%%%%%%%
All the paddles are $3.02~cm$ thick and are  trapezoidal in shape. The paddles come in two sizes: the smaller version has bases $32.2~cm$ and $26.7~cm$, and $61.0~cm$ height, while the bigger version has bases $33.2~cm$ and $27.0~cm$, and $70.8~cm$ height.  A Zener-diode Hamamatsu H5783 PMT collects the light from a wavelength-shifting optical fiber which runs along one of the long sides of each paddle.
A custom-made PMT Amplifier and Discrimination (PAD) circuit mounted at one end of the paddle collects signals from the PMTs and sends them to the Control and Concentrator Unit (CCU). We use the same connection to  power the PMT, control voltage and threshold, and output the PMT signal as logic ECL pulse.
We retrieved the scintillation paddles from the decommissioning of the CDF detector at Fermilab and we used only the paddles with a counting efficiency greater than 95\% and low noise at working voltage. The measured trigger rate of the whole system is $0.032Hz$, corresponding to $\sim 2$ muons per minute.


\begin{figure}[h!]
\centering
 \includegraphics[angle=90,width=0.7\textwidth]{Chapter-3/Images/Cosmic_Paddle.jpg}
\caption{Photograph of one of the scintillation counters used in the cosmic towers. } 
\label{pic:cosmicpaddle}
\end{figure}





\section{In the Cryostat}
\subsection{Cryogenics and Argon Purity}\label{ch:Cryo}
LArIAT repurposed the ArgoNeuT cryostat \cite{argoneut} in order to use it in a beam of charge particles. We also added a new process piping and a new liquid argon filtration system in FTBF.
Inside the LArIAT experimental hall, the cryostat sits on the beam of charge particles with its horizontal main axis oriented parallel to the beam.

Two volumes make up LArIAT cryostat, shown in Figure \ref{fig:LArIATCryoStat}: purified liquid argon fills the inner vessel, while the outer volume insulates it with a vacuum jacket equipped with layers of aluminized mylar superinsulation. The inner vessel is a cylinder of 130~cm length and 6.2~cm diameter, containing about 550~L of LAr, corresponding to a mass of 0.76 ton. We run the signal cables for the LArTPC and the high voltage feedthrough through a ``chimney'' at the top and mid-length of the cryostat.


\begin{figure}[htb]
\centering
\includegraphics[scale=0.18]{Chapter-3/Images/Cryostat1.jpg}
\includegraphics[scale=0.07]{Chapter-3/Images/Cryostat2.jpg}
\caption{\emph{(left)} The LArIAT cryostat open with the TPC placed in the inner volume. \emph{(right)} The LArIAT cryostat fully sealed during initial commissioning prior to installation at Fermilab Testbeam Facility.Access to the internal volume is possible by opening the upstream end caps of the inner and outer vessels. }
\label{fig:LArIATCryoStat}
\end{figure}

Given the different scopes of the ArgoNeuT and LArIAT detectors, we made several modification to the ArgoNeuT cryostat in order to use it in LArIAT. In particular, the modification  shown in Figure \ref{fig:LArIATCryoMods} were necessary to account for the beam of charged particles entering the TPC and to employ the new FTBT liquid argon purification system. 
We added a ``beam window'' on the front outer end cap and an ``excluder'' on the inner endcap, with the scope of minimizing the amount of dead material upstream of the TPC's active volume.  Doing so, we reduced the amount of uninstrumented material before the TPC from $\sim$ 1.6 radiation lengths ($X_{0}$) (ArgoNeuT) to less than 0.3 $X_{0}$ (LArIAT). To allow studies of the scintillation light, we added a side port feedthrough which enables the mounting of the light collection system, as well as the connections for the corresponding signal and high-voltage cables (see Section \ref{sec:TPCLight}).  We modified the bottom of the cryostat adding Conflat and ISO flange sealing to connect the liquid argon transfer line to the new argon cooling and purification system.


\begin{figure}[htb]
\centering
\includegraphics[scale=0.35]{Chapter-3/Images/CryoMods.png}
\caption{Pictures of the modified components of the cryostat. $1)$ The addition of an outlet to the bottom of the cryostat to allow connections to the purification system; $2)$ The ``beam-window'' on the outer endcap and the concave inner surface of the inner endcap (referred to as the excluder) to reduce the amount of material through which beam particles must travel before entering the TPC; $3)$ The modified side port for the LArIAT light collection system.}
\label{fig:LArIATCryoMods}
\end{figure}

%%%%%%%%%%%%%%%%%%%%%
As in any other LArTPC, argon purity is a crucial parameter for LArIAT. Indeed, the presence of contaminants effects both the basic working principles of a LArTPC: electronegative contaminants such as oxygen and water decrease the number of ionization electrons collected on the wires after drifting through the volume, while contaminants such as Nitrogen decrease the light yield from scintillation light, especially in its slow component.
In LArIAT, contaminations should not exceed the level of 100 parts per trillion (ppt). We achieve this level of purity in several stages. The specifics required for the commercial argon bought for LArIAT are 2 parts per million (ppm) oxygen, 3.5~ppm water, and 10~ppm nitrogen. This argon is monitored with the use of commercial gas analyzer.
Argon is stored in a dewar external to LArIAT hall and filtered before filling the TPC. %The argon is delivered from the commercial dewar to the cryostat through 2.54~cm diameter schedule 10 stainless steel piping.  The piping was insulated with 20.32~cm of polyurethane foam by the manufacturer.  The piping was cleaned to remove oil and grease before being welded into the system. 
LArIAT uses a filtration system designed for the Liquid Argon Purity Demonstrator (LAPD)~\cite{LAPD}: half of a 77~liter filter contains a 4A molecular sieve (Sigma-Aldrich~\cite{sigma-aldrich}) apt to remove mainly water, while the other half contains BASF~CU-0226~S, a highly dispersed copper oxide impregnated on a high surface area alumina, apt to remove mainly oxygen~\cite{basf}. A single pass of argon in the filter is sufficient to achieve the necessary purity, unless the filter is saturated. In case the filter saturates, the media needs to be regenerated by using heated gas; this happened twice during the Run II period\footnote{We deemed the filter regeneration necessary every time the electron lifetime dropped under 100 \textmu s}.
The filtered argon reaches the inner vessel via a liquid feedthrough on the top of the cryostat. Argon is not recirculated in the system, but it rather boils off and vent to the atmosphere. During data taking, we replenish the argon in the cryostat several times per day to keep the TPC high voltage feedthrough and cold electronics always submerged. In fact, we need to monitor the level, temperatures, and pressures  of the argon both in the commercial dewar and inside the cryostat. 
\subsection{LArTPC: Charge Collection}\label{sec:TPCCharge}
The LArIAT Liquid Argon Time Projection Chamber is a rectangular box of dimensions 47 cm (width) x 40 cm (height) x 90 cm (length), containing 170 liters of Liquid Argon.
The LArTPC three major subcomponents are 
\begin{itemize} 
\item[1)] the cathode and field cage,
\item [2)] the wire planes, 
\item [3)] the read-out electronics. %
\end{itemize}



\subsubsection{Cathode and field cage}
A G10 plain sheet with copper metallization on one of the 40x90~cm inner surfaces forms the cathode. 
A high-voltage feedthrough on the top of the LArIAT cryostat delivers the high voltage to the cathode; scope of the high voltage system (Figure~\ref{fig:HVScheme}) is to drift ionization electrons from the interaction of charged particles in the liquid argon to the wire planes.  The power supply used in this system is a Glassman LX125N16 ~\cite{GlassmanPS} capable of generating up to -125~kV and 16~mA of current, but operated at -23.5kV during LArIAT Run-II. The power supply is connected via high voltage cables to a series of filter pots before finally reaching the cathode. 

\begin{figure}[htb]
\centering
\includegraphics[scale=0.35]{Chapter-3/Images//HVSchematic.png}\\
\caption{Schematic of the LArIAT high voltage system.}
\label{fig:HVScheme}
\end{figure}%See DocDB 1472



The field cage is made of \textcolor{red}{13 parallel copper rings} framing the inner walls of the G10 TPC structure. A network of voltage-dividing resistors connected to the field cage rings steps down the high voltage from the cathode to form a uniform electric field. The electric field over the entire TPC drift volume is  486 V/cm (see \ref{ch:AppendixA}). The  maximum drift length, i.e. the distance between cathode and anode planes, is 47 cm.

\subsubsection{Wire planes}
The wire planes measure the charge deposited in the TPC active volume. The drifting charge induces a current on the wire of the inner planes and it is collected on the collection plane wires.
LArIAT counts three wire planes separated by 4 mm spaces: in order of distance from the cathode, they are the shield, the induction and the collection plane. The distance between two consecutive wires in each given plane (aka wire pitch) is 4 mm.  The shield plane counts 225 parallel wires of equal length oriented vertically. This plane is not connected with the read-out electronics; rather it shields the outer planes from extremely long induction signals due to the ionization chamber in the whole drift volume. As the shield plane acts almost like a Faraday cage, the shape of signals in the first instrumented plane (induction)  results easier to reconstruct.  Both the induction and collection planes count 240 parallel wires of different length oriented at 60$^\circ$ from the vertical with opposite signs.
Moving electrons moving past the induction plane will induce a bipolar pulse on its wires and will form a unipolar pulse when collected on the last plane wires. 

The three wire planes and the cathode form three drift volumes, as shown in Figure \ref{fig:driftregions}. 
The main drift volume is defined as the region between the cathode plane and the shield plane (C-S). The other two drift regions are those between the shield plane and the induction plane (S-I), and between the induction plane and the collection plane (I-C). The electric field in these regions is chosen to satisfy the charge transparency condition to allow for 100$\%$ transmission of the drifting electrons through the shield and then the induction planes. 

\begin{figure}[htb]
\centering
\includegraphics[scale=0.35]{Chapter-3/Images/DriftRegions.png}\\
\caption{Schematic of the three drift regions inside the LArIAT TPC: the main drift volume between the cathode and the shield plane (C-S) in green, the region between the shield plane and the induction plane (S-I) in purple, and the region between the induction plane and the collection plane (I-C) in pink.}
\label{fig:driftregions}
\end{figure}

Table \ref{tab:voltages} provides the default voltages applied to the cathode and the shield, induction, and collection plane.  

\begin{table}[htpb]
\centering
\caption{Cathode and anode planes default voltages}
\label{tab:voltages}
\begin{tabular}{llll}
\hline
\multicolumn{1}{|l|}{ Cathode} & 
\multicolumn{1}{|l|}{ Shield} & \multicolumn{1}{l|}{ Induction} & \multicolumn{1}{l|}{ Collection}  \\ \hline
\multicolumn{1}{|l|}{-23.17 kV} &
\multicolumn{1}{|l|}{-298.8 V} & \multicolumn{1}{l|}{-18.5 V}      & \multicolumn{1}{l|}{338.5 V} \\ \hline
\end{tabular}
\end{table}


\subsubsection{Electronics}

Dedicated electronics read the induction and collection plane wires, for a total of  480-channel analog signal path from the TPC wires to the signal digitizers. A digital control system for the TPC-mounted electronics, a power supply, and a distribution system complete the front-end system. Figure \ref{pic:FEelectronics} shows a block diagram of the overall system. The direct readout of the ionization electrons in liquid argon forms typically small signals on the wires, which needs to be amplified in oder to be processed. LArIAT  performs the amplification stage directly in cold with  amplifiers developed by Brookhaven National Lab (BNL) and mounted on the TPC frame inside the liquid argon, achieving a remarkable Signal-to-Noise ratio. %The BNL ASICs adopted in LAriAT are designated as LArASIC, version 4-star.%The ASIC signals for each wire are then driven out of the vessel  to DAQ boards that act as waveform recorders.
The signal from the ASICs are driven to the other end of the readout chain, to the CAEN V1740 digitizers. The CAEN V1740 has a 12 bit resolution and a maximum input range of 2~VDC, resulting in about 180 ADC count for a crossing MIP.   

\begin{figure}[htbp]
 \centering
 \includegraphics[width=1.0\textwidth]{Chapter-3/Images/LArIAT_FE_Electronics.png}
\caption{Overview of LArIAT Front End electronics. } 
\label{pic:FEelectronics}
\end{figure}




\subsection{LArTPC: Light Collection System}\label{sec:TPCLight}
The mechanism of particle detection in argon other than drift electrons is the collection of scintillation photons.  Over the course of LArIAT three years of data taking, the light collection system changed several times. We describe here the light collection system for Run II. Two PMTs, a 3-inch diameter Hamamatsu R-11065 and 2-inch diameter ETL D757KFL~\cite{lightsys-pmttests}, as well as three SiPMs arrays (two Hamamatsu S11828-3344M 4x4 arrays and one single-channel SensL MicroFB-60035 ) are mounted on the PEEK support structure. PEEK screws into an access flange as shown in Figure~\ref{lightsys_pmts}, on the anode side, leaving  approximately 5~cm of clearance from the collection plane.  

%------------------------------------------
\begin{figure}
\centering
\includegraphics[height=2.2in]{Chapter-3/Images/lightsys_pmts.png}
\hspace{1cm}
\includegraphics[height=2.2in]{Chapter-3/Images/lightsys_wls.png}
\caption{LArIAT's photodetector system for observing LAr scintillation light inside the TPC (left), and a simplified schematic of VUV light being wavelength-shifting along the TPB-coated reflecting foils (right).}
\label{lightsys_pmts}
\end{figure}
\begin{figure}
\centering
\includegraphics[height=0.25\textheight]{Chapter-3/Images/lightsys_etlbase.jpeg}
\hspace{0.5cm}
\includegraphics[height=0.25\textheight]{Chapter-3/Images/lightsys_hmmbase.jpg}
\caption{\label{voltagedividers}Photos of the voltage divider bases for the ETL PMT (left) and the Hamamatsu PMT (right) used in Run-II.  The cable connections to the bases seen here were used for powering and testing prior to installation.  The yellow through-hole signal coupling capacitors seen on both bases are 18~nF (X7R) and are rated to 2~kV.}
\end{figure}
%------------------------------------------
Liquid argon scintillates in vacuum-ultraviolet (VUV) range at 128 nm; since cryogenic PMTs are not sensitive to VUV wavelengths, we need to shift the light in a region visible to the PMTs. In LArIAT, the wavelength shifting is achieved by installing on the four walls of the TPC highly-reflective VIKUITY dielectric substrate foils coated with a thin layer of tetraphenyl-butadiene (TPB) (see Figure ~\ref{lightsys_foils}). One or more visible photons  are emitted and reflected into the chamber during the interaction of a VUV photon interacts with the TPB. Thus, the light yield is increased and made more uniform across the TPC active volume, allowing the possibility of light-based calorimetry (under study).

%------------------------------------------
%\begin{figure}
%\centering
%\includegraphics[scale=1.3]{Chapter-3/Images/lightsys_foils.png}
%\caption{\label{lightsys_foils}The TPB-coated reflector foils mounted to the TPC field cage walls as viewed through the front cryostat opening. {\textcolor{red}{Better pictures needed.}}}
%\end{figure}
%------------------------------------------

For Run II, we coated both  the windows of the ETL PMT and SensL SiPM  with a thin layer of TPB. In doing so, some of the VUV scintillation light converts into visible right at the sensor faces, keeping information on the direction of the light source. Information about the light directionality is lost for light reflected on foils, as the reflection is uniform in angle. For Run-II, the voltage dividers for the PMTs were configured for positive bias with a DC-coupled anode (AC-coupled anode with grounded photocathode) to minimize induced noise on the TPC wires and modify the PMT bases accordingly.  



\section{Trigger and DAQ}
The LArIAT DAQ and trigger system governs the read out of all the many subsystems forming LArIAT. 
The CAEN V1495 module and its user-programmable FPGA  are the core of this system.  Every 10~ns, this module checks for matches between sixteen logical inputs and user-defined patterns in the trigger menu; if it finds a match for two consecutive clock ticks, that trigger fires.

The beam instruments,  the cosmic ray taggers, and the light collection system provide NIM-standard logic pulse inputs to the trigger decision. We automatically log the trigger inputs configuration with the rest of the DAQ configuration at the beginning of each run.

Fundamental inputs to the trigger card come from the TOF (see Sec.~\ref{sec:TOF}) and the wire chambers (see Sec.~\ref{sec:MWPC}), as activity in these systems points to the presence of a charged particle in tertiary beam line.
In particular, the discriminated pulses from the TOF PMTs form a NIM logic pulse for the trigger logic. We ask for a coincidence within a 20~ns window for all the pulses from the PMTs looking at the same scintillator block and use the coincidence int the upstream and downstream paddle to inform the trigger decision. In order to form a coincidence between the upstream and downstream paddles, we delay the upstream paddle coincidence by 20~ns and widen it by 100~ns. The delay and widening are necessary to account for both  lightspeed particles and slower particles (high-mass) to travel the 6.5~m between the upstream and the downstream paddles. 
Four multi-hit TDCs read out each wire chamber: two TDC per plane (horizontal and vertical), sixty-four wires per TDC. In each TDC, we keep the logical ``OR" for any signal over threshold from the sixty-four wires. We then require a coincidence between the ``OR" for the horizontal TDCs and the ``OR" for the vertical TDCs: with this logic we make sure that at least one horizontal wire and one vertical wire saw significant signal in one wire chamber.  The single logical pulse from each of the four wire chambers feeds into the first four inputs to the V1495 trigger card. We require a coincidence within 20~ns of at least three logical inputs to form a trigger.


%Another primary input to the trigger card is from the cosmic towers (see Section~\ref{sec:CosmicRayPaddle}). To capture cosmic ray events in which a minimally ionizing cosmic ray muon crossed the TPC along the body diagonal, NIM modules form the logical coincidences from the two cosmic towers, one upper and one lower paddle assembly, in each combination.  The OR of these is provided as an input to the V1495. 

%Three important logic pulses are derived from the timing of the beam.  These include a pulse in a brief window before the beam, a pulse indicating that the beam is on, and a pulse which defines the beam-free period which may be used for collecting cosmic-ray events.  An adjustable pulser is a fourth trigger input which does not depend on any particular activity in the experiment hall,  useful for collecting background events with zero bias. 

%%%%%%%%%%%%%%%%%%%%%%

%The PMTs observing liquid argon scintillation light (see Section \ref{sec:PhotonSystem}) produce pulses which form the foundation of several interesting trigger inputs.  Thresholding a copy of each PMT pulse (after amplification), and requiring a coincidence of pulses within $\sim$20~ns, creates simple trigger inputs indicating ionizing radiation was produced in the TPC.  This scintillation logic pulse is used to initiate a gate which spans the length of the TPC drift time, creating a logic signal which is remains ``on'' while significant drift charge may still be present in the TPC.  In addition, requiring a delayed coincidence of two subsequent scintillation logic pulses, separated by a variable length of time ranging from 300~ns to 7~$\mu$s, is used to create a trigger input to select events where a cosmic muon stops and decays to a Michel electron in the TPC.  A few different versions of this light-based trigger were implemented throughout the course of LArIAT's run time to allow reconstruction and calorimetric studies of Michel electrons. Figure~\ref{michel_logic} shows a schematic diagram of the logic comprising the Michel electron trigger. 

%\begin{figure}
%\includegraphics[width=\textwidth]{figures/trigger_michellogic.png}
%\caption{\label{michel_logic}A schematic diagram of the trigger logic used to select Michel electron events during the cosmic readout window of the LArIAT supercycle.  The two PMT signals refer to the Hamamatsu (``HMM'') and ETL PMTs described in Section~\ref{sec:PhotonSystem}.  For some data-taking periods in Run-II, un-amplified pulses were discriminated at 180 mV to act as a veto on events that may saturate the dynamic range of the V1751 digitizer.  The discriminator thresholds used on the amplified (x10) PMT signal copies (\emph{ThA}, \emph{ThB}) as well as the Gate Delay period, were adjusted between run periods while experimenting with different configurations.}
%\end{figure}

%Further trigger inputs come from the beam line instrumentation behind the LArTPC cryostat, the PMTs of the Punch-Through scintillator paddles and those of the scintillator paddles instrumenting the Muon Range Stack.  The PMT pulses of all four of the broad-faced Punch Through paddles are discriminated to form logic pulses.  A single logic pulse is formed from these, indicating activity in at least two overlapping paddles at the rear of cryostat, before the steel block of the range stack.  PMT pulses from the Muon Range Stack are amplified and threshold discriminated.  These MuRS paddle pulses are then combined as in the Punch Through, creating single-bit indicators for each of the four instrumented layers that at least one pair of overlapping scintillator paddles sent signals within a 20~ns coincidence window.

%\subsubsection{Trigger Decision and Issuance}

%The V1495 may be configured to have up to sixteen trigger patterns and sixteen veto patterns, based on the trigger input signals.  A trigger pattern is defined as the AND of one or more defined inputs, and may include the NOT of the AND of further inputs.  Veto patterns are independently defined in the same way, but they have a very different effect.  When any of the trigger patterns fire, a ``fast trigger'' signal is issued and an adjustable countdown is initiated.  If the countdown completes without a veto pattern firing, the ``slow trigger'' signal is also issued and on a distinct hardware channel. Otherwise, if a veto pattern fires during the countdown, the slow trigger signal is vetoed.  

%The fast trigger signal prompts readout of all the `short' data buffers, which include the V1751 modules, the V1495 itself, and the MWPC controller.  The V1751 buffers typically contain digitized PMT signals from the time of flight and cryogenic light collection detectors. Readout of the TPC wire signals, which are much longer and more numerous, is only prompted at the issuance of the slow trigger.



\section{Control Systems}
LArIAT is a complex ensemble of systems which needed to be monitored at once during data taking.  We performed the monitoring of the systems operations with a slow control system, a DAQ monitoring system and a low level data quality monitoring described in the following sections.

\subsubsection{Slow Control}
We used the Synoptic Java Web Start framework as a real-time display of subsystem conditions. Its simple 
Graphical User Interface allowed us to change the operating parameters and to graph the trends of several variables of interest for all the tertiary beam detectors.  Among the most important quantities monitored by Synoptic there are the level of argon in both the inner vessel and the external dewar, the operating voltages of cathode and wire planes, of the PMTs and SiPMs, and of the four wire chambers, as well as the magnets temperature. Figure \ref{fig:synoptics} shows an example of the monitoring system.
LArIAT uses the Accelerator Control NETwork system (ACNET) to monitor the beam conditions of the MCenter beamline. For example, the X and Y position of the beam at the first two wire chambers (WC1 and WC2) are shown in \ref{fig:ACNET}. 

\begin{figure}[htb]
\centering
\includegraphics[width=\textwidth,height=\textheight,keepaspectratio]{Chapter-3/Images/BeamOverview.png}
\caption{Interface of the Synoptic slow control system}
\label{fig:synoptics}
\end{figure}

\begin{figure}[htb]
\centering
\includegraphics[scale=0.5]{Chapter-3/Images/BeamPosition.png}
\caption{Beam position at the upstream wire chambers monitored with ACNET.}
\label{fig:ACNET}
\end{figure}

\subsubsection{DAQ Monitoring}

We monitor the data taking and the run time evolution with the Run Status Webpage (\href{http://lariat-wbm.fnal.gov/lariat/run.html}{http://lariat-wbm.fnal.gov/lariat/run.html}), a  webpage updated in real-time.  The page displays, among other information, the total number of triggers in the event per CAEN digitizer board, the total number of detectors triggered during a beam spill,  the trigger patterns (the number of times a particular trigger pattern was satisfied  during a beam spill) and current time relative to the Fermilab accelerator complex supercycle. A screen shot of the page is show in figure \ref{fig:runcond}.

\begin{figure}[htb]
\centering
\includegraphics[scale=0.6]{Chapter-3/Images/RunConditions.png}
\caption{Run Status page at LArIAT downtime. At the top the yellow bar displays the current position in the Fermilab supercycle. Interesting information to be monitored by the shifter were the run number and number of spills, time elapsed from data taking (here in red), the energy of the secondary beam and the trigger paths.}
\label{fig:runcond}
\end{figure}



\subsubsection{Data Quality Monitoring}
We employ two systems to ensure the quality of out data during data taking: the Near-real-time Data Quality Monitoring and the Event Viewer.

\href{http://lariat-daq01.fnal.gov:5000/}{The Near-real-time Data Quality Monitoring} (DQM). This webpage receives updates from all the VME boards in the trigger system and displays the results of a quick analysis of the DAQ stream of raw data on a spill-by-spill basis. The DQM allows the shifter to monitor almost in real time (typically with a 2-minute delay)  a series of low level-quantities and compare them to past collections of beam spills. Some of the variables monitored in the DQM are  the pedestal mean and RMS on CAEN digitizer boards
of the TPC wires and PMTs of the beamline detectors, the hit occupancy and timing plots on the multi-wire chambers, and number of data fragments recorded that are used to build a TPC event. Abnormal values for  low-level quantity in the data  activate a series of alarms in the DQM; this quick feedback on the DAQ and beam conditions is fundamental to assure a fast debugging of the detector and a very efficient data taking during beam uptime.

The online Event Viewer displays a two dimensional representation of LArIAT TPC events on both the Induction and the Collection planes in near real time. The raw pulses collected by the DAQ on each wire are plotted as a function of drift time, resulting in an image of the TPC event easily readable by the shifter. This tool guarantees a particularly good  check of the TPC operation which activate an immediate feedback for troubleshooting a number of issues. For example,  it easy for the shifter to spot high occupancy events and request a reduction of the primary beam intensity, or to spot a decrease of the argon purity which requires the regeneration of filters, or to catch the presence of electronic noise and reboot the ASICs. An example of high occupancy event is shown in \ref{fig:highOcc}.

\begin{figure}[htb]
\centering
%\includegraphics[scale=0.6]{Chapter-3/Images/RunConditions.png}
\caption{High occupancy event display.}
\label{fig:highOcc}
\end{figure}


