\chapter{Total Hadronic Cross Section Measurement Methodology}\label{ch:Interactions}
{\raggedleft ``\emph{Like a lemon to the lime and the bubble to the bee}" \par}
{\raggedleft -- Eazy-E,   1993 -- \par}%Gimmie that *,
\vspace{0.5cm}

This chapter describes the general procedure employed to measure  total hadronic interaction cross sections on Ar in LArIAT.
Albeit with small differences, both the  ($\pi^{-}$,Ar) and (K$^{+}$,Ar) total hadronic cross section measurements rely on the same procedure. We start by selecting the particle of interest using a combination of beamline detectors and TPC information (Section \ref{ch:ParticleSelectionMethod}). We then perform a handshake between the beamline information and the TPC tracking to assure the selection of the correct TPC track (Section \ref{ch:WC2TPCMatchMethod}) associated to the corresponding beam particle. We then apply the ``thin slice" method to measure the ``raw" hadronic cross section (Section \ref{ch:ThinSliceMethod}). A series of corrections are then evaluated and applied to obtain the final cross section (Section \ref{ch:MCCorrections}). 

At the end of this chapter, we show a sanity check of the methodology by applying the thin slice method employing only MC truth information and retrieving the expected MC cross section for pions and kaons (Section \ref{ch:procedureTesting}).



\section{Event Selection}\label{ch:ParticleSelectionMethod}
The measurement of the ($\pi^{-}$,Ar) and (K$^{+}$,Ar) total hadronic cross section in LArIAT starts by selecting the pool of pion or kaon candidates and measuring their momentum before they enter the LAr volume.  This is done through the series of selections on  beamline and TPC information described in the next sections. The summary of the event selection in data is reported in Table \ref{tab:beamlineDataSelection}.


\begin{table}[b]
\centering
\begin{tabular}{|l|c|c|}
\hline
                                                        & Run-II Neg Pol   &  Run-II Pos Pol  \\ \hline
1. Events Reconstructed in Beamline        &  158396  & 260810  \\ \hline
2. Events with Plausible Trajectory            &   147468 & 240954  \\ \hline
3. Beamline $\pi^-/\mu^-/e^-$  Candidate  &   138481 &     N.A.   \\ \hline
4. Beamline $K^+$   Candidate                 &    N.A       & 2837     \\ \hline
5. Events Surviving Pile Up Filter              &   108929  & 2389       \\ \hline
6. Events with WC2TPC Match                 &    41757   & 1081 \\ \hline
7. Events Surviving Shower Filter             &    40841    &  N.A.     \\ \hline
8. Available Events For Cross Section      &   40841    &   1081    \\ \hline
\end{tabular}
\caption{Number of data events for Run-II Negative and Positive polarity }
\label{tab:beamlineDataSelection}
\end{table}


\subsection{Selection of Beamline Events}\label{ch:beamlineDetectorsData}
We leverage the beamline particle identification and momentum measurement before entering the TPC as an input to evaluate the kinetic energy for the hadrons used in the  cross sections measurements. To this end, we select the LArIAT data to keep only events whose wire chamber and time of flight information is registered (line 1 in in Table \ref{tab:beamlineDataSelection}). Additionally, we perform a check of the plausibility of the trajectory inside the beamline detectors: given the position of the hits in the four wire chambers, we make sure the particle's trajectory does not cross any impenetrable material such as the collimator and the magnets steel (line 2 in in Table \ref{tab:beamlineDataSelection}).


\subsection{Particle Identification in the Beamline}
In data, the main tool to establish the identity of the hadron of interest is the LArIAT tertiary beamline, in its function of mass spectrometer. We combine the measurement of the time of flight, $TOF$, and the beamline momentum, $p_{Beam}$, to reconstruct the invariant mass of the particles in the beamline, $m_{Beam}$, as follows
\begin{equation}
m_{Beam} = \frac{p_{Beam}}{c}\sqrt{\biggl(\frac{TOF*c}{l}\biggr)^2 -1},
\label{eq:mass}
\end{equation}
 where $c$ is the speed of light and $l$ is the length of the particle's trajectory between the time of flight paddels. 

Figure \ref{fig:mass} shows the mass distribution for the Run II negative polarity runs on the left and positive polarity runs on the right. We perform the classification of events into the different samples as follows:

\begin{itemize}
\item \underline{$\pi/\mu/e$:}  mass $<$ 350~MeV/c$^2$

\item \underline{kaon:} 350~MeV $<$ mass $<$ 650~MeV/c$^2$

\item \underline{proton:} 650~MeV $<$ mass $<$ 3000~MeV/c$^2$.

\end{itemize}

Lines 3 and 4 in in Table \ref{tab:beamlineDataSelection} show the number of negative $\pi/\mu/e$ and positive $K$ candidates which pass the mass selection for LArIAT Run-II data.

%\begin{comment}     
\begin{figure}
  \centering  
\includegraphics[width=\textwidth]{Chapter-4/Images/massRunII.png}
\caption{Distribution of the beamline mass as calculated according to equation \ref{eq:mass} for the Run-II events reconstructed in the beamline, negative polarity runs on the left and positive polarity runs on the right. The classification of the events into $\pi^\pm/ \mu^\pm/e^\pm$, K$^\pm$, or (anti)proton is based on these distributions, whose selection cut are represented by the vertical colored lines.}
\label{fig:mass}
\end{figure}

\subsection{TPC Selection: Halo Mitigation }\label{ch:pileUp}
The secondary beam impinging on LArIAT secondary target produces a plethora of particles which propagates downstream. The presence of upstream and downstream collimators greatly abates the number of particles traveling down the LArIAT tertiary beamline. However, it is possible that more than one particle sneaks into the LArTPC during its readout time: the TPC readout is triggered by the particle  firing the series of beamline detectors along our tertiary beamline, but particles from the beam halo might also be present in the TPC at the same time. We call ``pile up" the additional traces in the TPC. We adjusted the primary beam intensity between LArIAT Run I and Run II to reduce the presence of events with high pile up particles in the data sample. For the cross section analyses, we remove events with more than 4 tracks in the first 14 cm upstream portion of the TPC from the sample (line 5 in in Table \ref{tab:beamlineDataSelection}).


\subsection{TPC Selection: Shower Removal}\label{ch:electrons}
In the case of the ($\pi^-$,Ar) cross section, the resolution of  beamline mass spectrometer is not sufficient to select a beam of pure pions. In fact,  muons which are close in mass to the pions and relativistic electrons survive the selection on the beamline mass.  It is important to notice that the composition of the negative polarity beam is mostly pions, as will be discussed in section \ref{ch:beamlineComposition}.
Still, we devise a selection on the TPC information to mitigate the presence of electrons in the sample used for the pion cross section. The selection relies on the different topologies of a pion and an electron event when propagating in liquid argon: while the former will trace a track inside the TPC active volume, the latter will tend to ``shower", i.e. interact with the medium, producing bremsstrahlung photons which pair convert into several short tracks. In order to remove the shower topology, we create a region of interest (ROI) around the TPC track corresponding to the beamline particle. We look for short tracks contained in the ROI, as depicted in Figure \ref{fig:showerFilt}:  if more then 5 tracks shorter than 10 cm are in the ROI, we reject the event. Line 7 in  Table \ref{tab:beamlineDataSelection} shows the number of events surviving this selection; that table also shows that this selection is applied after the beamline event is matched to TPC particle (discussed in the next section). This match already lowers the presence of electrons in the sample, which is further reduced by the shower filter.

\begin{figure}
  \centering  
\includegraphics[width=\textwidth]{Chapter-4/Images/Shower.png}
\caption{Visual rendering of the shower filter. The ROI is a cut cone, with a small radius of 4 cm, a big radius of 10 cm and an height of 42 cm (corresponding to 3 radiation lengths for electrons in Argon).}
\label{fig:showerFilt}
\end{figure}



\section{Beamline and TPC Handshake: the Wire Chamber to TPC Match}\label{ch:WC2TPCMatchMethod}
For each event passing the selection on its beamline information, we need to identify the track inside the TPC corresponding to the particle which triggered the beamline detectors, a procedure we refer to as ``WC to TPC match" (WC2TPC for short). In general, the TPC tracking algorithm can reconstruct more than one track in the event, partially due to the fact that hadrons interact in the chamber and partially because of pile up particles during the triggered TPC readout time, as shown in Figure~\ref{fig:kaonInteraction}. 


\begin{figure}
  \centering  
\includegraphics[width=\textwidth]{Chapter-4/Images/KaonExample.png}
\caption{Kaon candidate event: on the right, event display showing raw quantities; on the left, event display showing reconstructed tracks. In the reconstructed event display, different colors represent different track objects. A kink is visible in the kaon ionization, signature of a hadronic interaction: the tracking correctly stops at the kink position and two tracks are formed. An additional pile-up track is so present in the event (top track in red).}
\label{fig:kaonInteraction}
\end{figure}



We attempt to uniquely match one wire chamber track (see Section \ref{sec:MWPCfunc}) to one and only one reconstructed TPC track. 
In order to determine if a match is present, we apply a geometrical selection on the relative position of the wire chamber and TPC tracks. 
We start by considering only TPC tracks whose first point is in the first 2 cm upstream portion of the TPC for the match.  We project the wire chamber track to the TPC front face where we define the coordinates of the projected point as  $x_{FF}$ and $y_{FF}$.  For each considered TPC track, we define $\Delta$X as the difference between the $x$ position of the most upstream point of the TPC track and $x_{FF}$.  $\Delta$Y is defined analogously. We define the radius difference, $\Delta$R, as $ \Delta \text{R} =  \sqrt{ \Delta \text{X}^2 +  \Delta \text{Y}^2}  $. We define  as $\alpha$ the angle between the incident WC track and the TPC track in the plane that contains them.  If  $\Delta \text{R} < 4 $~cm, $\alpha < 8^\circ $,  a match between WC-track and TPC track is found. We describe  how we determine the value for the radius and angular selection in Appendix \ref{ch:AppendixTrack}.
We discard events with multiple WC2TPC matches. We use only those TPC tracks that are matched to WC tracks in the cross section calculation. Line 6 in Table \ref{tab:beamlineDataSelection} shows the number of events where a unique WC2TPC match was found.

In MC, we mimic the matching between the WC and the TPC track by constructing an artificial WC track using truth information at wire chamber four. We then apply the same WC to TPC matching algorithm as in data. 


\begin{figure}
  \centering  
\includegraphics[width=\textwidth]{Chapter-4/Images/WC2TPCMatchTracks.png}
\caption{Visual rendering of the wire chamber to TPC match.}
\label{fig:showerFilt}
\end{figure}

\section{The Thin Slice Method}\label{ch:ThinSliceMethod}



Once we have selected the 40841 beamline pion candidates  and the 1081 beamline kaon candidates, and we have identified the TPC corresponding track, we apply the thin slice method to measure the cross section, as the following sections describe. 
\subsection{Cross Sections on Thin Target}
Cross section measurements on a thin target have been the bread and butter of nuclear and particle experimentalists since the Geiger-Marsden experiments \cite{Geiger1909}. At their core, this type of experiments consists in shooting a beam of particles with a known flux on a thin slab of material and recording the outgoing flux. 


In general even in the case of thin target, the target is not a single particle, but rather a slab of material containing many diffusion centers. The so-called  ``thin target" approximation assumes that the target centers are uniformly distributed in the material and that the target is thin compared to the projectile interaction length, so that no center of interaction sits in front of another. In this approximation, the ratio between the number of particles interacting in the target $N_{\text{Int}}$ and the number of incident particles $N_{\text{Inc}}$ on the target estimates the interaction probability $P_{Interacting}$, which is the complementary to one of the survival probability $P_{Survival}$. 
Equation \ref{eq:thinTargetXS} 
\begin{equation}
P_{Survival} = 1- P_{Interacting} = 1 - \frac{N_{\text{Int}}}{N_{\text{Inc}}} = e^{-\sigma_{TOT}\text{ } n \text{ }\delta X}
\label{eq:thinTargetXS}
\end{equation}
describes the probability for a particle to survive the thin target. This formula relates  the interaction probability to the total hadronic cross section ($\sigma_{TOT}$), the density of the target centers ($n$)\footnote{The scattering center density in the target, {\emph{n}},  relates to the argon density $\rho$, the Avogadro number  $ N_{A} $ and the argon molar mass $m_A$ as $n=\frac{\rho N_{A} }{m_A}$.}    and  the thickness of the target  along the incident hadron direction ($\delta X$). If the target is thin compared to the interaction length of the process considered, we can Taylor expand the exponential function in equation \ref{eq:thinTargetXS} and find a simple proportionality relationship between the cross section and the number of incident and interacting particles, as shown in equation \ref{eq:thinTargetXSTaylor}:
\begin{equation}
1 - \frac{N_{\text{Int}}}{N_{\text{Inc}}} =  1 -\sigma_{TOT} \text{  }n \text{  }\delta X + O(\delta X^2).
\label{eq:thinTargetXSTaylor}
\end{equation}

Solving for the cross section, we find:
\begin{equation}
 \sigma_{TOT}  = \frac{1}{n \text{ }\delta X}\frac{N_{\text{Int}}}{N_{\text{Inc}}}.
\label{eq:thinTargetXSSolved}
\end{equation}

\subsection{Not-so-Thin Target: Slicing the Liquid Argon Volume}\label{ch:XSRaw}
The interaction length of pions and kaons in liquid argon is expected to be of the order of 50 cm for pions and 100 cm for kaons. Thus, the LArIAT TPC, with its 90 cm of length, is not a thin target. However, the granularity of the LArIAT LArTPC detector allows us to treat the argon volume as a sequence of many adjacent thin targets. 

As described in Chapter \ref{sec:experimentDescription}, LArIAT induction and collection planes consist of 240 wires each at 4 mm spacing. The wires are oriented at +/- $60^{\circ}$ from the vertical direction, while the beam direction is oriented 3 degrees off the $z$ axis in the $XZ$ plane.  The collection wires collect signals proportional to the energy deposited by the hadron along its path in a  $\delta${\emph{X}} = 4 mm/(sin($60^{\circ}$)cos($3^{\circ}$)) $\approx$ 4.7~mm slab of liquid argon. Thus, one can think to slice the TPC into many thin targets of $\delta${\emph{X}} = 4.7~mm thickness along the direction of the incident particle, making a measurement at each wire along the path, as sketched in Figure \ref{fig:TPCGran}.

\begin{figure}
  \centering  
\includegraphics[width=0.8\textwidth]{Chapter-4/Images/LArSlice.png}
\caption{Representation of sliced LAr Volume.}
\label{fig:TPCGran}
\end{figure}


Considering each slice {\emph{j}}  a ``thin target",  we can apply the cross section calculation from Equation~\ref{eq:thinTargetXSSolved} iteratively, evaluating the kinetic energy of the hadron as it enters each slice, $E_{j}^{kin}$.  For each WC2TPC matched particle, the energy of the hadron entering the TPC is known thanks to the momentum and mass determination by the tertiary beamline, 

\begin{equation}
 E^{kin}_{Front Face}  = \sqrt{p^2_{Beam} - m^2_{Beam}} - m_{Beam} - E_{loss},
\label{eq:enFF}
\end{equation}
where $E_{loss}$ is a correction for the kinetic energy loss in the uninstrumented material between the beamline and the TPC front face. While propagating through the target,  the kinetic energy of the hadron at each slab is determined by subtracting the energy deposited by the particle in the previous slabs. For example, at the $j^{th}$ slab of a track, the kinetic energy will be

\begin{equation}
 E_{j}^{kin} =  E^{kin}_{Front Face} - \sum_{i < j} E_{\text{Dep},i},
\label{eq:KEj}
\end{equation}
where $E_{\text{Dep},i}$ is the energy deposited at each argon slice before the $j^{th}$ point as measured by the calorimetry associated with the tracking.


If the particle enters a slice, it contributes to the $N_{\text{Inc}}( E^{kin})$ distribution in the energy bin corresponding to its kinetic energy in that slice. While into the slice, a hadron may or may not interact. If it interacts in the slice, it  contributes also to the $N_{\text{Int}}(E^{kin})$ distribution in the appropriate energy bin; this occurrence corresponds to the end of the hadron tracking. If the hadron does not interact, it will enter the next slice and the interaction evaluation starts again.
The process is applied to all the hadrons in the sample; the cross section as a function of kinetic energy, $\sigma_{TOT}( E^{kin})$ is then evaluated to be proportional to the ratio $\frac{N_{\text{Int}}( E^{kin})}{N_{\text{Inc}}( E^{kin})}$ -- bin by bin ratio. 


Our goal is to measure the total interaction cross section, independently  from the topology of the interaction. Thus, we determine that a hadron interacted simply by requiring that the last point of the WC2TPC matched track lies in a slice within the fiducial volume, whose boundaries are defined in Table \ref{tab:FidVol}. If the TPC track ends within the fiducial volume, its last point will be the interaction point; if the track crosses the boundaries of the fiducial volume, the track will be considered ``through going" and no interaction point will be found. The only points of the hadronic candidate track considered to fill the  $N_{\text{Int}}$ and  $N_{\text{Inc}}$ distributions are the ones contained in the fiducial volume. 
 
 A notable background pertinent only to the $N_{\text{Int}}$  distribution are cases in which the hadrons decays inside the TPC. In those cases in fact, the tracking ends inside the TPC but the interaction is not hadronic. The handling of decay background is treated in a slightly different way for the pion and kaon section, details can be found in sections \ref{ch:PionXSBkgSub} and \ref{ch:KaonXSRaw} respectively.



\begin{table}[t]
\centering
\begin{tabular}{|l|r|r|}
\hline
& min   &  max  \\ \hline
$X$ & 1 cm   & 46 cm  \\ \hline
$Y$ & -15 cm   & 15  cm  \\ \hline
$Z$ & 0 cm   & 86 cm  \\ \hline
\end{tabular}
\caption{Fiducial volume boundaries used to determine cross section interaction point. }
\label{tab:FidVol}
\end{table}



\subsection{Corrections to the Raw Cross Section}\label{ch:MCCorrections}
%%%%%%%%%%%%%%%%%%%%%%%%%%
Equation \ref{eq:thinTargetXSSolved}  is a prescription for measuring the cross section in case of a pure beam of the hadron of interest and 100\% efficiency in the determination of the interaction point.  For example, if LArIAT had a beam of pure pions and were 100\% efficient in determining the interaction point within the TPC, the pion cross section as a function of  kinetic energy (estimated at the central value of the energy bin $E_i$) would be given by

\begin{equation}
 \sigma^{\pi^-}_{TOT}(E_{i})  = \frac{1}{n\text{ } \delta X}\frac{N^{\pi^-}_{ \text{Int}} (E_{i})}{N^{\pi^-}_{ \text{Inc}}(E_{i})}.
\label{eq:thinTargetXSSolved2}
\end{equation}

Unfortunately, this is not the case. In fact, the selection used to isolate pions in the LArIAT beam allows for the presence of some muons and electrons as background, while the kaon selection allows for a small contamination of protons (see Section \ref{ch:beamlineComposition}). Also, the LArTPC tracking algorithm is not 100\% efficient in determining the interaction point. This inefficiency occurs in two fashions: i) the tracking algorithm does not stop at the interaction point and continues adding hits from a particle past it (this happens especially in the case of shallow elastic scattering), ii) the tracking stops prematurely. These two cases have different consequences on the population of the interacting and incident distributions. In the first case, the interacting histogram will be underpopulated and the incident histogram might be overpopulated. In case of premature end of tracking, the interacting histogram will be overpopulated at energies greater than the eventual interaction, while the incident histogram will be underpopulated. Given the importance of tracking for the cross section measurements,  we report an optimization to maximize the identification of the interaction point in Appendix \ref{ch:AppendixTrack}.

Therefore, we apply two corrections evaluated on MC in order to extract the final cross section from LArIAT data: i) a background subtraction and ii) a correction for reconstruction effects. 
Still using the pion case as example, we estimate the pion cross section in each energy bin changing  Equation \ref{eq:thinTargetXSSolved2} into
\begin{equation}
 \sigma^{\pi^-}_{TOT}(E_{i})  =\frac{1}{n\text{ } \delta X}\frac{N^{\pi^-}_{ \text{Int}} (E_{i})}{N^{\pi^-}_{ \text{Inc}}(E_{i})} = \frac{1}{n \text{ }\delta X}\frac{ \epsilon^{\text{Inc}}(E_i) [ N^{ \text{TOT}}_{ \text{Int}} (E_{i}) - B_{ \text{Int}} (E_i)] }{   \epsilon^{\text{Int}}(E_i) [N^{ \text{TOT}}_{ \text{Inc}}(E_{i}) - B_{ \text{Inc}} (E_i)]},
\label{eq:True}
\end{equation}


 
where  $N^{\text{TOT}}_{\text{Int}} (E_{i})$ and $N^{\text{TOT}}_{\text{Incident}} (E_{i})$ is the measured content of the interacting and incident histograms for events that pass the event selection, $B_{\text{Int}} (E_i)$ and $B_{\text{Inc}} (E_i)$ represent the contributions from the background to the interacting and incident histograms respectively, and  $\epsilon^{\text{Int}}(E_i)$ and  $\epsilon^{\text{Inc}}(E_i)$ are the corrections for reconstruction effects.

As we will show in Section \ref{ch:PionXSBkgSub}, the background subtraction for the interacting and incident histograms can be translated into corresponding relative pion content factors $C^{\pi MC}_{\text{Int}} (E_{i})$ and $C^{\pi MC}_{\text{Inc}} (E_{i})$ and the cross section re-written as follows

\begin{equation}
      \sigma^{\pi^-}_{TOT}(E_{i})  = \frac{1}{n\text{ } \delta X}\frac{ \epsilon^{\text{Inc}}(E_i)  \hspace{0.2cm} C^{\pi MC}_{\text{Int}} (E_{i}) \hspace{0.2cm} N^{\text{TOT}}_{\text{Int}} (E_{i}) }{   \epsilon^{\text{Int}}(E_i) \hspace{0.2cm} C^{\pi MC}_{\text{Inc}} (E_{i}) \hspace{0.2cm}  N^{\text{TOT}}_{\text{Inc}} (E_{i})}.
\label{eq:C}
\end{equation}



%The following sections describe the procedures used to evaluate  the background subtraction (section \ref{sec:beamCont}) and the efficiency correction (section \ref{sec:EffCorrection}), as well as  their uncertainties. 
%The reader might be concerned about bin-by-bin migration of events in the interacting and incident plots due to the finite resolution of the energy reconstruction. In section \ref{sec:Energy}, we make an argument to why we expect the smearing matrix to be extremely close to diagonal, such that its calculation and relative corrections are left for an improvement of the analysis.


%%%%%%%%%%%%%%%%%%%%%%%%%%%%%%%%%%%%%%%%%%%%%%%%



\section{Procedure testing with MC truth quantities}\label{ch:procedureTesting}
The ($\pi^{-}$,Ar) and (K$^{+}$,Ar) total hadronic cross section implemented in Geant4 can be used as a tool to validate the measurement methodology.  We describe here a closure test done on Monte Carlo to prove that the methodology of slicing the TPC retrieves the underlying cross section distribution implemented in Geant4 within the MC statistical uncertainty. 

For pions and kaons in the considered energy range, the Geant4 inelastic model adopted is ``BertiniCascade"; the pion elastic cross sections are tabulated from Chips, while the kaon elastic cross sections are tabulated on Gheisha and Chips.

For the validation test, we fire a sample of pions and a sample of kaons inside the LArIAT TPC active volume using the Data Driven Monte Carlo, a procedure described in Section \ref{sec:DDMC}. We apply the thin-sliced method using only true quantities to calculate the hadron kinetic energy at each slab in order to decouple reconstruction effects from possible issues with the methodology.  For each slab of 4.7 mm length along the path of the hadron, we integrate the true energy deposition as given by the Geant4 transport model. Then, we recursively subtracted it from the hadron kinetic energy at the TPC front face to evaluate the kinetic energy at each slab until the true interaction point is reached. Since the MC is a pure beam of the hadron of interest and truth information is used to retrieve the interaction point, no background correction or reconstruction effects correction is applied. Doing so, we obtain the true interacting and incident distributions for the considered hadron, whose ratio leads to  the true MC cross section as a function of the hadron kinetic energy. 

Figure \ref{fig:TrueMCXS2} shows the total hadronic cross section for argon implemented in Geant4 10.03.p1 (solid lines) overlaid with the true MC cross section as obtained with the sliced TPC method (markers) for pions on the left and kaons on the right; the total cross section is shown in green. For completeness, we also report the contributions from  the elastic cross section (in blue) and the inelastic cross section (in red), available at the MC level.  The nice agreement with the Geant4 distribution and the cross section  obtained with the sliced TPC method gives us confidence in the  validity of the methodology. 
        
%\begin{comment}     
\begin{figure}
%\captionsetup{justification=raggedright}  
\begin{minipage}[b]{.53\textwidth}  
  \centering  
\includegraphics[width=3in]{Chapter-4/Images/PionTrueXS.png}
\end{minipage}%  
\begin{minipage}[b]{0.53\textwidth}  
  \centering  
\includegraphics[width=3in]{Chapter-4/Images/KaonTrueXS.png}
\end{minipage}
\caption{Hadronic cross sections for ($\pi^-$,Ar) on the left and (K$^+$,Ar) on the right as implemented in Geant4 10.03.p1 (solid lines) overlaid the true MC cross section as obtained with the sliced TPC method (markers). The total cross section is shown in green,  the elastic cross section in blue and the inelastic cross section in red.}
\label{fig:TrueMCXS2}
\end{figure}
%\end{comment}





