\chapter{Conclusions}\label{ch:Conclusions}
In the era of neutrino precision measurements, of huge liquid argon detectors and of massive amount of information from LArTPCs, a renew interest for an ancient measurement arises: the measurement of hadronic interactions with matter. With this work, we presented the first ever ($\pi^-$-Ar) and ($K^+$-Ar) total hadronic cross section measurements as a function of the hadron kinetic energy. These analyses are the first physics analyses developed by the LArIAT experiment.
Both the analysis follow a similar workflow and  they rely on beam line detector information as well as both calorimetry and tracking in the TPC. 






In order to measure ($\pi^-$-Ar) total hadronic  argon cross sections, we start by selecting pion beamline candidates through a series of selections on the beamline and TPC information apt to maximize the number of pions in the selection over the number of muons and electrons. We use the LArIAT beamline MC to estimate the beam composition of the selected beamline candidare and we propagate the particle species to the LArAIT TPC constructing a properly weighted sample with the DDMC.


 The analyses start by identifying a sample of the hadron of interest in the beam line and assessing the beam line contaminations. It proceeds with tracking the hadron candidates in the TPC and measuring their kinetic energy at each point in the tracking: the fine sampling of an hadron in the TPC forms the set of ``incident" hadrons.  Then, the hadronic interaction point is identified and the raw cross section is calculated. Two corrections are then applied to the raw cross section -- a background subtractions and a correction for detector effects -- to obtain the true cross section measurement.\\



These analyses' work flow will serve as a basis for the future cross section measurements of pions and kaons in the exclusive channels.
