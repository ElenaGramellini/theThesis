\chapter{Preparatory Work}\label{ch:samples}
%{\raggedleft ``\emph{Il dolce non lo mangi mai, ma qualche volta ti rifai.} \par}
%{\raggedleft \emph{Abbracciami}"\par}
%{\raggedleft -- Pietro Ciampi, L'amore e' tutto qui, 1971 -- \par}
%\vspace{0.5cm}

%{\raggedleft ``\emph{You never eat dessert, but, sometimes, you make up for it.} \par}
%{\raggedleft \emph{Hug me}"\par}
%{\raggedleft -- Pietro Ciampi, L'amore e' tutto qui, 1971 -- \par}
%\vspace{0.5cm}


This chapter describes the preparatory work done on the the data and Monte Carlo samples used for the cross section analyses. 
This entails the choice of the data set and the production of the information needed to construct the Monte Carlo Simulation~(section \ref{sec:dataSet}),  the construction and use of said Monte Carlo simulation~(section \ref{sec:MCSet}), the study and optimization of the tracking in the TPC for the cross section analyses~(section \ref{sec:TrackingStudies}),  the calibration of the calorimetry response and related energy studies~(section \ref{ch:energyCal}). 



\section{Cross Section Analyses Data Set}\label{sec:dataSet}
We choose LArIAT Run-II as the data period for the  ($\pi^{-}$,Ar) and (K$^{+}$,Ar) total hadronic cross section analyses. 
Data taking for the this period started on 03/15/2016  and ended on 07/31/2016. 
Since we are interested in beamline and TPC information, we ask basic requirements on the operational status of the time of fight, wire chambers and TPC to form the good run list for this period, which we informally call ``lovely runs".

The subset of lovely runs  chosen for the  ($\pi^{-}$,Ar) total hadronic cross section analysis includes only the -60A and -100A magnet configurations in negative polarity, even if LArIAT explored several other beamline configurations during Run-II. The -60A and -100A combined data set accounts for approximately 90\% of the total Run-II negative polarity runs.   Since the production of beamline Monte Carlo depends on the wanted beamline configuration, the choice of only two beamline settings limits the need for beamline MC production. 

Similarly, the subset of lovely runs chosen  for the (K$^{+}$,Ar)  total hadronic cross section analysis includes only the +60A and +100A magnet configurations in positive polarity. It should be noted that kaons are extremely rare in the +60A sample, thus the data sample for the (K$^{+}$,Ar) cross section after the mass selection is about 90\% +100A runs, as shown in Table \ref{tab:databreakdown}.

For the first measurements in LArIAT that uses both beamline and TPC information, we choose strict requirements on the reconstruction of the WC tracks, the so-called ``Picky Track" sample (see \ref{sec:MWPCfunc}). This choice presents two advantages:  the uncertainty on the momentum reconstruction for the ``Picky Tracks" sample is smaller compared to the ``High Yield" sample, and the comparison with the beamline MC results is straightforward. A possible future update and cross check of these analysis would be the use of the High Yield sample, where the statistics is about three times higher. 

The breakdown of beamline events as a function of the magnets settings is shown in Table \ref{tab:databreakdown}. 
The choice of the data sets determines the production of beamline MC and serves as basis for the production of Data Driven MC, as shown in the next sections.

\begin{table}[b]
\centering
\begin{tabular}{|l|c|c|c|}  
\hline
                                                              & I = 60 A          & I = 100 A   & Total     \\ \hline
Data Events after $\pi/\mu/e$ Mass Selection     &     67068          &  71413  & 138481 \\ \hline
Data Events after $K$ Mass Selection                &     274              &   2563   & 2837  \\ \hline
\end{tabular}
\caption{Number of data events which fit the $\pi/\mu/e$ or $K$ mass hypothesis as a function of magnet settings.}
\label{tab:databreakdown}
\end{table}



\section{Construction of a Monte Carlo Simulation for LArIAT}\label{sec:MCSet}
For the simulation of LArIAT events and their particle make up, we use a combination of two MC generators: the G4Beamline Monte Carlo and the Data Driven single particle Monte Carlo (DDMC). We use the G4Beamline MC to simulate the particle transportation in the beamline and calculate the particle composition of the beam just after the fourth Wire Chamber (WC4). In order to simulate the beamline particles after WC4 and in the TPC, we use the DDMC.

\subsection{G4Beamline}\label{ch:beamlineComposition}
G4Beamline simulates the beam collision with the LArIAT secondary target, the energy deposited by the particles in the LArIAT beamline detectors, and the action of the LArIAT magnets, effectively accounting for particle transportation through the beam line from the LArIAT target until ``Big Disk", a fictional, void detector located just before the LArIAT cryostat. 
 At the moment of this writing, G4Beamline does not simulated the responses of the beam line detectors. It is possible to interrogate the truth level information of the simulated particles in several points of the geometry. In order to ease the handshake between G4Beamline and the DDMC, we ask for the beam composition just after WC4.
Since LArIAT data are taken under different beam conditions, we need to simulate separately the beam composition according to the magnets' settings and the secondary beam intensity with G4Beamline. For the pion cross section analysis the relevant beam conditions are  secondary beam energy of 64 GeV, negative polarity magnet with current of 100 A and 60 A. For the kaon cross section analysis the relevant beam conditions is a secondary beam energy of 64 GeV, positive polarity magnet with current of 100 A. 

\subsubsection{Beam Composition for Negative Pion Cross Section}
Even if pions are by far the biggest beam component in negative polarity runs, the LArIAT tertiary beam is not a pure pion beam. While useful to discriminate between pions, kaons, and protons, the beamline detectors are not sensitive enough to  discriminate among the lighter particles in the beam: electrons, muons and pions fall under the same mass hypothesis. Thus, we need to assess the contamination from beamline particles other than pions in the event selections used for the pion cross section analysis and correct for its effects. The first step of this process is assessing the percentage of electrons and muons in the $\pi/\mu/e$ beamline candidates via the G4Beamline MC. The full treatment of the beamline contamination in the pion cross section calculation is described in section \ref{ch:PionXSBkgSub}.
Since the beamline composition is a function of the magnet settings, we simulate separately events for magnet current of -60A and -100A. 
Figure \ref{fig:BeamComposition} shows the momentum predictions from G4Beamline overlaid with data for the 60A runs (left) and for the 100A runs (right). The predictions for electrons, muons and pions have been staggered and their sum is area normalized to data. Albeit not perfect, these plots show a reasonable agreement between the momentum shapes in data and MC. We attribute  the difference in shape to the lack of simulation of the WC efficiency in the MC which is momentum dependent and leads to enhance the number events in the center of the momentum distribution.

\begin{figure}
\includegraphics[width=0.5\textwidth,height=\textheight,keepaspectratio]{Chapter-5/Images/Beam60A.png}
\includegraphics[width=0.5\textwidth,height=\textheight,keepaspectratio]{Chapter-5/Images/Beam100A.png}
\caption{Beam composition for the -60A runs (left) and -100A runs (right). The solid blue plot represents the simulated pion content, the yellow plot represents the simulated muon content and the grey plot represents the simulated electron content. The plots are area normalized to the number of data events, shown in red. }
\label{fig:BeamComposition}
\end{figure}

Table \ref{tab:beamline} shows the beam composition per magnet setting after the mass selection according to the G4Beamline simulation.
\begin{table}[]
\centering
\begin{tabular}{|l|c|c|}
\hline
                     & I = -60 A           & I = -100 A \\ \hline
G4Pions       &   68.8 \%           &      87.4 \%        \\ \hline
G4Muons     &     4.6 \%           &        3.7 \%         \\ \hline
G4Electrons &   26.6 \%           &        8.9 \%        \\ \hline
\end{tabular}
\caption{Simulated beamline composition per magnet settings}
\label{tab:beamline}
\end{table}


\subsubsection{Beam Composition for Positive Kaon Cross Section}
In the positive polarity runs, the tertiary beam composition is mainly pions and protons. The left side of Figure \ref{fig:BeamCompositionPos} shows the  predictions for the momentum spectra for the 100A positive runs  according to  G4Beamline (solid colors) overlaid with data (black points). 
Since the LArIAT beamline detectors can discriminate between kaons and other particles, we do not rely on the G4Beamline simulation to estimate the beamline contamination in the pool of kaon candidates (as in the case of the pion cross section), but rather we use a data drive approach. 
The basic idea of this data driven approach is to estimate the bleed over from high and low mass peaks under the kaon peak by fitting the tails of the $\pi/\mu/e$ and proton mass distributions, as shown in Figure \ref{fig:BeamCompositionPos} right side. 
Since the shape of the tails is unknown, the estimate is done multiple times varying the range and shape for reasonable functions. 
For example, to estimate the proton content under the kaon peak, we start by fitting the left tail of the proton mass distribution with a gaussian function between 650 $MeV/c^2$ and 750 $MeV/c^2$.% in a reasonable range and with a reasonable function. 
We extend the fit function under the kaon peak and integrate the between 350-650 $MeV/c^2$. We integrate the mass histogram in the same range and calculate the proton contamination as the ratio between the two integrals. We repeat this procedure for several fit shapes (gaussian, linear and exponential functions are used) and tail ranges. Finally, we calculate the contamination as the weighted average of single estimates, where the weights are calculated to be the $1./\chi^2$ of the tail fits. The procedure is repeated for lighter particles mass peak independently.
With 12 iterations of this method we find a proton contamination of  0.2 $\pm$ 0.5 \%  and a contamination from the lighter particles of 5 $\pm$ 2 \% .



\begin{figure}
\includegraphics[width=0.5\textwidth,height=\textheight,keepaspectratio]{Chapter-5/Images/Beam100Pos.png}
\includegraphics[width=0.5\textwidth,height=\textheight,keepaspectratio]{Chapter-5/Images/MassPos.png}
\caption{$Left.$ Beam composition for the +100A runs after WC4 (no mass selection applied). The solid blue plot represents the simulated pion content, the yellow plot represents the simulated muon content and the grey plot represents the simulated positron content, the red the proton content and the mustard the kaon content. The plots are area normalized to the number of data events, shown in black.$Right.$ Mass distribution for the Run-II positive runs, where the area under the kaon mass peak is highlighted in purple. The area under the extension of a possible fit for the proton tail is highlighted in red. }
\label{fig:BeamCompositionPos}
\end{figure}



\subsection{Data Driven MC}\label{sec:DDMC}
The Data Driven single particle Monte Carlo (DDMC) is a single particle gun which simulates the particle transportation from WC4 into the TPC leveraging on the beamline data information. The DDMC uses the data momentum and position at WC4 to derive the event generation: a general sketch of the DDMC workflow is shown in Figure \ref{fig:DDMCSketch}.

When producing a DDMC sample, beam line data from a particular running period and/or running condition are selected first. For example, data for the negative 60A runs and for the negative 100A runs inform the event generation stage of two different DDMC samples. Figure \ref{fig:DDMCQuantities}  schematically shows the data quantities of interest leveraged from data: the momentum ($P_x, P_y, P_z$) and position ($X, Y$) at WC4. For each data event, we obtain the  particle position ($X, Y$) at WC4 directly from the data measurement; we calculate the components of the momentum using the beamline measurement of the momentum magnitude in conjunction with the hits on WC3 and WC4 to determine the direction of the momentum vector, as described in section \ref{sec:MWPCfunc}. The momentum and position of the selected data form a 5-dimensional tupla, which we sample thousands of times through a 5-dimensional hit-or-miss sampling procedure to generate the MC events. This produces MC $P_x, P_y, P_z, X, Y$ distributions  with the same momentum and position distributions as data, with the additional benefit of accounting for the correlations between the considered variables.  As an example, the results of the DDMC generation compared to data for the kaon +100A sample are shown in figure \ref{DDMCComparison} for the $P_z$, $X$ and $Y$ distributions; as expected, MC and data agree within the statistical uncertainty by construction. A LArSoft simulation module then launches single particle MC from z = -100 cm (the location of the WC4) using the MC generated events. The particles are free to decay and interact in their path from WC4 to the TPC according to the Geant4 simulation.

Using the DDMC technique ensures that the MC and data particles have very similar momentum, position and angular distributions at WC4 and allows us to use the MC sample in several occasions, for example to calibrate the energy loss upstream of the TPC (see Section \ref{ch:eloss}) or to study the tracking and the calorimetric performance (sections \ref{sec:TrackingStudies} and \ref{ch:energyCal}). A small caveat is in order here: the DDMC is a single particle Monte Carlo, which means that the beam pile-up is not simulated. 


Six samples are the basis fo the MC used in the pion cross section measurement: three samples of  $\sim$340000 pions, muons and electrons to simulate the negative 60A runs, and three samples of $\sim$340000 pions, muons and electrons for the negative 100A runs.

The MC used for the kaon cross section analysis is a sample of \textcolor{red}{NUMBERS}  kaons.

\begin{figure}[hpbt]
\centering
\includegraphics[width=\textwidth]{Chapter-5/Images/DDMCScheme.png}
\caption{Workflow for Data Driven single particle Monte Carlo production.}
\label{fig:DDMCSketch}
\end{figure}


\begin{figure}[hpbt]
\centering
\includegraphics[width=\textwidth]{Chapter-5/Images/DDMCQuantities.png}
\caption{Scheme of the quantities of interest for the DDMC event generation: $P_x, P_y, P_z, X, Y$ at WC4.}
\label{fig:DDMCQuantities}
\end{figure}


\begin{figure}[hpbt]
\centering
\includegraphics[width=0.48\textwidth]{Chapter-5/Images/DDMCPz.png}
\includegraphics[width=0.48\textwidth]{Chapter-5/Images/DDMCX.png}
\includegraphics[width=0.48\textwidth]{Chapter-5/Images/DDMCY.png}
\caption{Comparison between generated quantities and data distributions for the 100A kaon sample: Z component of the momentum at WC4 (top left), X position at Wire Chamber 4 (top right), Y position at Wire Chamber 4 (bottom).}
\label{fig:DDMCComparison}
\end{figure}



\subsection{Estimate of Energy Loss before the TPC}\label{ch:eloss}
The beamline particles travel a path from where their  momentum is measured in the beamline until they are tracked again inside the TPC. In the LArIAT geometry, a particle leaving the WC4 will encounter the materials listed in Table \ref{tab:budget} before being registered again. The energy lost by the particle in this non-instrumented material modifies the particle's kinetic energy and directly affects the cross section measurement, as shown in equation \ref{eq:enFF}.

\begin{table}[h!]
\centering
\begin{tabular}{|l|l|l|}
\hline
Material  & density {[}g/cm$^3${]} & width {[}cm{]}    \\ \hline
Fiberglass laminate (G10)      & 1.7                             & 1.28                              \\
Liquid Argon                           & 1.4                             & 3.20                             \\
Stainless Steel                        & 7.7                            & 0.23                             \\
Titanium                                  & 4.5                            & 0.04                             \\ 
Air                                            &  1.2 $\cdot10^{-3}$  & 89.43                              \\
Plastic Scintillator                    & 1.03                          & 1.20 (+ 1.30)                             \\ \hline
\end{tabular}
\caption{LArIAT material budget from WC4 to the TPC Front Face.}
\label{tab:budget}
\end{table}


We derive an estimate of the energy loss between the beamline momentum measurement and the TPC ($E_{loss}$) from the pion DDMC sample, since this quantity is not  measurable directly on data. 
The $E_{loss}$ distribution for the 60A  and 100A pion sample is shown in figure \ref{fig:ELoss60A}, left and right respectively. A clear double peaked structure is visible, which is due to the particles either missing or hitting the HALO paddle: a schematic rendering of this occurrence is  shown in figure \ref{fig:Halo}. The kinematic at WC4 determines the trajectory of a particle and whether or not it will hit the halo paddle. In figure \ref{fig:PxVsXTrue} , we plot the true  horizontal component of the momentum $P_x$ versus the true $X$ position at WC4 for pions missing the halo paddle (left) and for pions hitting the halo paddle (right) for the 60A MC simulation runs -- analogous plots are obtained with the 100A simulation. These distributions can be separated drawing a line in this position-momentum space. 
We use a logistic regression  \cite{agresti2013categorical}  as a classifier to find the best separating line, shown in both plots as the red line. We classify as ``hitting the halo paddle" all pions whose $P_x$ and $X$ are such that $$P_x +0.02* X - 0.4 < 0 $$ and as ``missing the halo  paddle" all pions whose $P_x$ and $X$ are such that $$P_x +0.02*X - 0.4 > 0, $$ where the coefficients of the line are empirically found by the logistic regression estimation. Overall, this simple methode classifies in the right category (hit or miss) about 86\% of the pion events. In MC, we assign  $E_{loss} = 32 \pm 4 $~MeV for pion events classified as ``hitting the halo paddle"; we assign  $E_{loss} = 24 \pm 3 $~MeV for pion events classified as ``missing the halo paddle". We apply the same classifier on data. A survey of the simulated geometry showed an excess of 3 cm of un-instrumented argon compared with the measured geometry used in data. Thus, we account for this difference by assigning in data $E_{loss} = 24 \pm 6 $~MeV for pion events classified as ``hitting the halo paddle" and  $E_{loss} = 17 \pm 6 $~MeV for pion events classified as ``missing the halo paddle", where the uncertainty is derived as the standard deviation of the double peaked distribution.

The analogous study on the kaon MC sample resulted in an energy loss of $E_{loss} = 37 \pm 5 $~MeV for kaon events classified as ``hitting the halo paddle" and $E_{loss} = 31 \pm 4 $~MeV for kaon events classified as ``missing the halo paddle".



s\begin{table}[b]
\centering
\begin{tabular}{|l|c|c|}  
\hline
                          & Hitting Halo          & Missing Halo     \\ \hline
Pion  MC           &  $32 \pm 4 $         &    $24 \pm 3$     \\ \hline
Pion Data          &  $25 \pm 6$          &    $17 \pm 6 $    \\ \hline
Kaon  MC          &  $37 \pm 5 $        &     $31 \pm 4 $    \\ \hline
Kaon Data         &  $26 \pm 6 $        &     $22 \pm 6 $    \\ \hline
\end{tabular}
\caption{Energy loss for pions and kaons.}
\label{tab:Eloss}
\end{table}



%We use the separation of these two distributions to decide what the energy loss for each event on data. 
%Thus,  we assign the value for energy loss is used in the data.


\begin{figure}[hbpt]
\centering
\includegraphics[width=0.45\textwidth]{Chapter-5/Images/E_loss60A.png}
\includegraphics[width=0.45\textwidth]{Chapter-5/Images/E_loss100A.png}
\caption{True energy loss between WC4 and the TPC front face according to the MC simulation of negative pions of the 60A runs (left) and of the 100A runs (right). The distribution for the whole data sample is shown in blue, the distribution for the pions missing the halo is shown in red, and the distribution for the pions hitting the halo is shown in green.  }
\label{fig:ELoss60A}
\end{figure}

\begin{figure}[hbpt]
\centering
\includegraphics[scale=0.5]{Chapter-5/Images/Halo.png}
\caption{Schematic rendering of the particle path between WC4 and the TPC front face. The paddle with the hollow central circle represents the Halo paddle. We illustrate two possible trajectories: in black, a trajectory that miss the paddle and goes through the hole in the Halo, in blue a trajectory that hits the Halo paddle and goes through the scintillation material.}
\label{fig:Halo}
\end{figure}



\begin{figure}[hbpt]
\centering
\includegraphics[width=\textwidth]{Chapter-5/Images/PXVsX60A.png}
\caption{Horizontal component of the true momentum vs the horizontal position at WC4 for MC simulated pions of the 60A runs. The plot on the left shows the distribution for pion that miss the halo paddle and the plot on the right shows the distributions for pions that hit the halo. The form of the classifier is overlaid to both plots (red line).}
\label{fig:PxVsXTrue}
\end{figure}



\section{Tracking Studies}\label{sec:TrackingStudies}
\subsection{Study of WC to TPC Match}\label{ch:WC2TPCMatchOptimization}

\section{Energy Calibration and Studies}\label{ch:energyCal} 



\begin{comment}


%%%%%%%%%%%%%%%%%%%%%%%%%%%%%%%%%%%%%%%%%%%
%%%%%%%%%%%%%%         Energy Calibration         %%%%%%%%%%%%%%         
%%%%%%%%%%%%%%%%%%%%%%%%%%%%%%%%%%%%%%%%%%%
\section{Energy Calibration}\label{ch:energyCalibration}
Scope of the energy calibration is to identify the factors which convert the charge collected (dQ) to energy deposited in the chamber(dE). As described in section \ref{sec:SignalProc}, this is a multi-step procedure. In LArIAT, we first correct the raw charge by the electronic noise on the considered wire \cite{technote}, then by the electron lifetime \cite{LArIATLifeTime},  and then by the recombination using the ArgoNeut recombination values. Lastly, we apply overall calibration of the energy, i.e. we determine the ``calorimetry constants" using the procedure described in this section.


We independently determine  the calorimetry constants for Data and Monte Carlo in the LArIAT Run-II Data samples using  a parametrization of the energy deposited per unit length (dE/dX) as a function of momentum. This is done by comparing the stopping power measured on reconstructed quantities against the Bethe-Bloch theoretical prediction for various particle species (see equation \ref{eq:BB}).  We obtain the theoretical expectation for the dE/dX most probable value of pions ($\pi$), muons ($\mu$), kaons ($K$), and protons ($p$) in the momentum range most relevant for LArIAT (Figure \ref{fig:PDGEnergyLossArgon}) using the tables provided by the Particle Data Group \cite{Patrignani:2016xqp} for liquid argon \cite{PDG-Argon}.

The basic idea of this calibration technique is to utilize the most upstream portion of a TPC track which has a well known momentum and particle species to measure its $dE/dX$. Once a sample of particles dE/dX has been measured at various momenta, we then tune to calorimetry constants within the reconstruction software to align these measured values to match the theoretical ones found in Figure \ref{fig:PDGEnergyLossArgon}. 

In data, we start by selecting a sample of beamline positive pion candidates without any restriction on their measured momentum.
We then apply the WC2TPC match and subtract the energy loss upstream to the TPC front face, determining the momentum at the TPC front face. For each surviving pion candidate,  we measure the dE/dx at each of the first 12 spacepoints associated the 3D reconstructed track, corresponding to a $\sim$ 5 cm portion. These dE/dX measurements are then put into a histogram that corresponds to measured momentum of the track. The dE/dX histograms are sampled every 50 MeV in momentum (e.g. 150~MeV/c $< P <$ 200~MeV/c, 200~MeV/c $< P <$ 250/c~MeV, etc...).   This process of selecting, sampling, and recording the dE/dX for various momentum bins is repeated over the entire sample of events, allowing us to collect sufficient statistic in most of the momentum bins between 150~MeV/c and 1100~MeV/c.
Each 50 MeV/c momentum binned dE/dX histogram is now fit with a simple Landau function. The most probable value (MPV) and the associated error on the MPV from the fit are extracted and plotted on Figure \ref{fig:PDGEnergyLossArgon}. Depending on the outcome of the fit, we modify the calorimetry constants and we repeat the procedure until a qualitative agreement is achieved.  We perform this  tuning for the collection and induction plane separately. 
As a cross check to the determined calorimetry constants using the positive pions, we plot the dE/dx vs momentum distribution all the other particle species identifiable in the beamline data ($\pi$, K , p, in both polarities) against the corresponding Beth-Bloch prediction in a similar fashion. On average, pions and muons only lose $\sim$10 MeV in this 5~cm section of the track and protons lose $\sim$20 MeV. Thus choosing 50 MeV size bins for our histograms covers the energy spread within those bins due to energy loss from ionization.  The results of the tuning and cross check for Run-II data on the collection plane is shown in Figure \ref{fig:BBandData} positive polarity data on the left, negative polarity data on the right.

In MC, we simulate the corresponding positive pion sample with the DDMC (see section \ref{sec:DDMC}) and follow the same steps as in data. The calorimetry tuning is explained in all his gory details in \cite{technote}.


\begin{figure}[htb]
\centering
%\includegraphics[width=0.50\textwidth]{images/PDGdEdX.png}
\caption{Mean energy loss in various materials over a range of particle momenta as produced in Reference \cite{PDG}.}
\label{fig:PDGEnergyLoss}
\end{figure}


\begin{figure}[htb]
\centering
%\includegraphics[width=0.50\textwidth]{images/dEdXvsMomentumTemplate}
\caption{Mean energy loss for pions, muons, and protons in liquid argon over the momentum range most relvant for LArIAT.}
\label{fig:PDGEnergyLossArgon}
\end{figure}




\begin{figure}[htb]
\centering
%\includegraphics[width=0.50\textwidth]{images/CalibrationExample.png}
\caption{Illustration of the calibration technique. Here we depict a 325 MeV wire chamber track (shown in green) which enters the TPC (taking into account the energy loss from the upstream material) and we sample the first 12 spacepoints (shown in teal) to extract the dE/dX distribution which is fit with a Landau.}
\label{fig:CalibrationExample}
\end{figure}





%%%%%%%%%%%%%%%%%%%%%%%%%%%%%%%%%%%%%%%%%%%
%%%%%%%%%%%%%%         Energy Calibration         %%%%%%%%%%%%%%         
%%%%%%%%%%%%%%%%%%%%%%%%%%%%%%%%%%%%%%%%%%%




\section{Tracking Studies}
In this section, we describe three studies. The first is a justification of the selection criteria for the beamline handshake with the TPC information.  We perform this study to boost  the correct identification of the particles in the TPC associated with the beamline information, while maintaining sufficient statistics for the cross section measurement. 
The second study is an optimization of the tracking algorithm, with the scope of maximizing the identification of the hadronic interaction point inside the TPC. These two studies are related, since the optimization of the tracking is performed on TPC tracks which have been matched to the wire chamber track; in turn, the tracking algorithm for TPC tracks determine the number of reconstructed tracks in each event used to try the matching with the wire chamber track. Starting with a sensible tracking reconstruction, we perform the WC2TPC matching optimization first, then the tracking optimization. The WC2TPC match purity and efficiency  are then calculated again with the optimized tracking.


%\section{MC sample and WC2TPC match}
We perform the following studies on a MC sample of 191000 kaons and 359000 pions produced with the DDMC technique. DDMC particles are shot from the WC4 location into the TPC following the beam profile.
We mimic the matching between the WC and the TPC track on Monte Carlo by constructing a fake WC track using truth information at wire chamber four. We then apply the same WC to TPC matching algorithm as in data described in \ref{ch:WC2TPCMatchMethod}. 



%In data, we attempt to uniquely match one WC-Track to one and only one reconstructed TPC track. This match is done by using in the $X$ and $Y$ coordinate of the extrapolated WC-Track to the upstream most point of the reconstructed TPC Track and by using the angle between the incoming track angle and the reconstructed TPC. We define $\Delta$X as the difference between the $x$ position of the most upstream point of the TPC track and the $x$ position of the WC track as projected to the TPC front face. $\Delta$Y is defined analogously. We define  $\Delta$R as $ \Delta \text{R} =  \sqrt{ \Delta \text{X}^2 +  \Delta \text{Y}^2}  $. The angle between the incident WC Track and the TPC track in the plane that contains them defines $\alpha$.  

%We define a match between WC-track and TPC reconstructed track if  $\Delta \text{R} < r_{T}$, $\alpha < \alpha_{T}$ and the Z position of the first reconstructed point of the TPC track is within 2 cm from the TPC front face. The determination of the best $r_{T}$ and $\alpha_{T}$ is the scope of the following section.

%In MC, we mimic the matching between the WC and the TPC track on Monte Carlo by constructing a fake WC track using truth information at wire chamber four. We then apply the same WC to TPC matching algorithm as in data. 

\subsection{Selection Study for the Wire Chamber to TPC Match}\label{ch:WC2TPCMatchOptimization}
Plots I want in this section:
\begin{enumerate}
\item WC2TPC MC DeltaX, DeltaY and $\alpha$
\end{enumerate}


Scope of this study is assessing the goodness of the wire chamber to TPC match on Monte Carlo and decide the selection values we will use on data. A word of caution is necessary here. With this study, we want to minimize pathologies associated with the presence of the primary hadron itself, e.g. the incorrect association between the beamline hadron and its decay products inside the TPC.  Assessing the contamination from pile-up\footnote{We remind the reader that the DDMC is a single particle Monte Carlo, where the beam pile up is not simulated.}, albeit related, is beyond the scope of this study.

In MC, we are able to define a correct WC2TPC match using the Geant4 truth information. We are thus able to count how many times the WC tracks is associated with the wrong TPC reconstructed track. 

We define a correct match if the all following conditions are met:
\begin{itemize}
\item[-] the length of the true primary Geant4 track in the TPC is greater than 2 cm,  
\item[-] the length of the reconstructed track length is greater than 2 cm,
\item[-] the Z position of the first reconstructed point is within 2 cm from the TPC front face
\item[-] the distance between the reconstructed track and the true entering point is the minimum compared with all the other reconstructed tracks.
\end{itemize}

In order to count the wrong matches, we consider all the reconstructed tracks whose Z position of the first reconstructed point lies within 2 cm from the TPC front face. Events with true length in TPC $<$ 2 cm are included. 
Since hadrons are shot 100 cm upstream from the TPC front face, the following two scenarios are possible from a truth standpoint: 
\begin{itemize}
\item[[$Ta$]] the primary hadron decays or interact strongly before getting to the TPC,
\item[[$Tb$]] the primary hadron enters the TPC.
\end{itemize}

Once we choose the selection cuts to determine a reconstructed wire chamber-to-TPC match $r_{T}$ and $\alpha_{T}$, the following five scenarios are possible in the truth to reconstruction interplay : 
\begin{itemize}
\item[1)] only the correct track is matched
\item[2)] only one wrong track is matched 
\item[3)] the correct track and one (or more) wrong tracks are matched
\item[4)] multiple wrong tracks  matched.
\item[5)] no reconstructed tracks are matched
\end{itemize}

Since we keep only events with one and only one match, we discard cases 3), 4) and 5) from the events used in the cross section measurement. For each set of $r_{T}$ and $\alpha_{T}$ selection value, we define purity and efficiency of the selection as follows:
\begin{equation}
\text{Efficiency} = \frac{\text{Number of events correctly matched}}{\text{ Number of events with primary in TPC}}
\end{equation}

\begin{equation}
\text{Purity} = \frac{\text{Number of events correctly matched}}{\text{Total number of matched events}}.
\end{equation}

Figure \ref{fig:EffPurityK} shows the efficiency (left) and purity (right) for wire chamber-to-TPC match as a function of the radius, $r_{T}$, and angle, $\alpha_{T}$, selection value. It is apparent how both efficiency and purity are fairly flat as a function of the radius selection value at a given angle. This is not surprising. Since we are studying a single particle gun Monte Carlo sample, the wrong matches can occur only for mis-tracking of the primary or for association with decay products;  decay products will tend to be produced at large angles compared to the primary, but could be fairly close to the in $x$ and $y$ projection of the primary. The radius cut would play a key role in removing pile up events. 

For LArIAT cross section measurements, we generally prefer purity over efficiency, since a sample of particles of a pure species will lead to a better measurement. Obviously, purity should be balanced with a sensible efficiency to avoid rejecting the whole sample. 

We choose $(\alpha_{T}$, $r_{T}) = (8 \text{ deg}, 4 \text{ cm} )$ and get a MC 85\% efficiency and 98\% purity for the kaon sample and a MC \textcolor{red}{BOH}\% efficiency and 98\% purity for the \textcolor{red}{BOH} sample.


\begin{figure}[hpbt]
\centering
\includegraphics[width=15cm]{Chapter-5/Images/KEffPurity.png}
\caption{Efficiency (left) and purity (right) for wire chamber-to-TPC match as a function of the radius and angle selections.}
\label{fig:EffPurityK}
\end{figure}

\subsection{Interaction Point Optimization}\label{ch:TrackingOptimization}
Scheme of this subsection
\subsubsection{Brief Explanation of the reconstruction chain}
\subsubsection{Explanation of clustering parameters}
\subsubsection{Figure of merit and  spanning of cluster}
\subsubsection{Important numbers out of this optimization}


Plots I want in this section:
\begin{enumerate}
\item Delta L, reco - true
\item Delta L, reco - true Elastic, Delta L, reco - true Inelastic, other
\item Length Quality cut
\item Efficiency as a function of true KE and Angle
\end{enumerate}


\subsection{Tracking spatial and angular resolution}
Scope of this study is understanding and comparing the tracking spatial and angular resolution on data and MC.
We start by selecting all the WC2TPC matched tracks. 
We fit a line on all the space points of the track and calculate the $\chi^2$. The $\chi^2$ distribution for data and MC is shown in Figure \ref{fig:Chi2AllPts}.

For the spatial and angular resolution study, we reject tracks with less than 14 space points. For each track, we order the space points according to their Z position and we split them in two sets: the first set counts all the points belonging to the first half of the track and the second set counts all the points belonging to the second half of the track. We remove the last 5 points in the first set and the first 5 points in the second set, so to have a gap in the middle of the original track. We fit the first and the second set of points with a line separately. We reject the event entirely if the  $\chi^2$ for the fit of either of the halves is greater than four.  We define a track middle plane as the plane perpendicular to the original track fit, positioned in the middle of its length. We project the tracks on the middle plane and calculate the impact parameter, $d$, i.e. the distance between the projected points. We also calculate the angle between the original track direction and the fit of the first and second half, called $\alpha_1$ and $\alpha_2$ respectively. The spatial resolution of the track will be $\sigma_S = \frac{d}{\sqrt 2}$ while the angular resolution of the tracks will be  $\sigma_\alpha = \alpha_1 - \alpha_2$. The distributions for data and MC for $\sigma_\alpha$ and $\sigma_S$ are given in \ref{fig:trackingResolution}.




\end{comment}




