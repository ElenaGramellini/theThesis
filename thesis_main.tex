%%%%%%%%%%%%%%%%%%%%%%%%%%%%%%%%%%%%%%
% yale_thesis.tex
% Elena Gramellini
% Started: 2017/12/05
%
% A bare, sample template for a Yale PhD thesis using yalephd.cls
%%%%%%%%%%%%%%%%%%%%%%%%%%%%%%%%%%%%%%

\documentclass[letterpaper,12pt]{yalephd}
% remove draft option for final printing.
% font size must be between 10pt-12pt.
\usepackage{caption,setspace}
\usepackage{amsmath} 
\usepackage{rotating}

\usepackage{geometry} % you need this for yalephd.cls to work.
\usepackage{graphicx}
\usepackage[graphicx]{realboxes}
 % you probably want the rest of these.
\usepackage{dcolumn}
\usepackage{bm}
\usepackage{hyperref}
\usepackage[export]{adjustbox}

\usepackage{mathtools}
\DeclarePairedDelimiter\bra{\langle}{\rvert}
\DeclarePairedDelimiter\ket{\lvert}{\rangle}
\DeclarePairedDelimiterX\braket[2]{\langle}{\rangle}{#1 \delimsize\vert #2}
\usepackage{textgreek}

\usepackage{multirow}
\usepackage{amsmath}
\usepackage{amsfonts}
\usepackage{amssymb}
\usepackage{appendix}
\usepackage{comment}
\usepackage{cite}
\usepackage{color}
\bibliographystyle{plain}
\usepackage{slashed}  

\setcounter{chapter}{-1}

\newenvironment{dedication}
  {%\clearpage           % we want a new page          %% I commented this
   \thispagestyle{empty}% no header and footer
   \vspace*{\stretch{1}}% some space at the top
   \itshape             % the text is in italics
   \raggedleft          % flush to the right margin
  }
  {\par % end the paragraph
   \vspace{\stretch{3}} % space at bottom is three times that at the top
   \clearpage           % finish off the page
  }

\usepackage{lineno,xcolor}


\linenumbers

\begin{document}

% Need to define title before the abstract.
\title{Measurement of ($\pi^-$-Ar) and ($K^+$-Ar) total hadronic cross sections in the LArIAT experiment}
\author{Elena Gramellini}
\advisor{Bonnie T. Fleming}
\date{Date you'll receive your degree} % usually not \today.

% All the stuff at the front of your thesis.
\frontmatter

\begin{abstract}
Abstract goes here. Limit 750 words.
\end{abstract}


\maketitle
\makecopyright{2018}

 \begin{dedication}
A mia mamma e mio babbo,\\
grazie per le radici e grazie per le ali.
    \par   %% or a blank line
    \vspace{2\baselineskip}
To my mom and dad,\\
thank you for the roots and thank you for the wings.
    \vspace{\baselineskip}
  \end{dedication}

\tableofcontents
%\listoffigures % remove this if you have no figures.
%\listoftables % remove this if you have no tables.

\chapter{Acknowledgements} % this needs to be before \mainmatter.

{\raggedleft ``\emph{Dunque io ringrazio tutti quanti.} \par}
{\raggedleft \emph{Specie la mia mamma che mi ha fatto cos\'i funky.}"\par}
{\raggedleft -- Articolo 31, Tanqi Funky, 1996 -- \par}
\vspace{0.5cm}

{\raggedleft ``\emph{At last, I thank everyone.} \par}
{\raggedleft \emph{Especially my mom who made me so funky.}"\par}
{\raggedleft -- Articolo 31, Tanqi Funky, 1996 -- \par}
\vspace{0.5cm}

A lot of people are awesome, especially you, since you probably agreed to read this when it was a draft. 


%I jumped the pond a little less than 5 years ago because I had a taste of what being a physicists in the US might mean and wanted more. 
%I came to this vast (and at times inhospitable) land thinking I would learn a lot of physics (which I did) and trying to keep doing research for as long as I could (which I'm fortunate enough to say I'll still do for a while). Little did I know that getting my PhD would be first and foremost a journey to self discovery, the beginning of a path to asymptotically tend to the person and the scientist I want to be. A journey impossible without the help of many people, which I will now proceed to thank, hoping that the acknowledgements won't be longer than the thesis itself.

%Mom, you are and will always be the hero of my story; I know this has been challenging for you, thanks for being there for me at every hour of the day and the night, for being always honest with me and for somehow handling this whole US-Italy thing gracefully. I can't believe you still haven't met Bonnie. Ah, and don't worry, I'll always translate things for you. 

%Bonnie Fleming, my advisor, I don't even know where to start in thanking you. Thanks for listening to me before I proved I was worth be listened to; thanks for always believing I could make it, even at the bottom of my doubts; thanks for the laughters, thanks for leading by example, thanks for  letting me part of the most wonderful research group ; 

%Flavio
%

%A big thank you to my academic sisters, from the youngest to the oldest (again academically speaking): Supraja Balasubramanian, you worked the magic of enriching my past, thank you from the bottom of my heart;  Brooke Morrison, the depth of your knowledge, your drive and love for our field has been and will always be a constant inspiration;  Ariana  Hackenburg, thanks for keeping me sane for the last four years, if you run for president -- as you should -- remember to delete our Skype chats, love and hugs to you dudette; and last but not least Xiao Luo, Fermilab has given me lots of extraordinary gifts, but no one  bigger than your friendship. 

%Jen Raaf
%Jonathan Asaadi
%Andrzej M. Szelc
%Roberto Acciarri
%LArIAT & MicroBooNE
%To all the members of the NPG, thanks for all the laughters and all the wise insights. You're the smartest bunch of idiots I've ever met, I love y'all.
%Shany %Louise

%To the extraordinary people who stayed close, even with an ocean in between: la Tasso, Marci, la Bea, la Kittirn, Alvin, la Viola e Brogna, Sighi and all the people from the ``La Chiappia Nuova" group, thanks for making me feel like I never left.
%Finally, Anto, cause people might think I'm nice and easy, but you know I'm not. And you're still here.


% Starts proper arabic numbering of pages and chapters.
\mainmatter
\chapter{Introduction}

\section{The Standard Model}
The Standard Model of particle physics the most accurate theoretical description of the subatomic world.  The Standard Model describes the strong, electromagnetic and weak interactions among  elementary particles in the framework of quantum field theory. The weak and the electromagnetic interactions are unified into the electroweak interaction. The Standard Model does not describe gravity or general relativity.
The Standard Model is a gauge theory based on the local group of symmetry
\begin{equation}
G_{SM} = SU(3)_C  \otimes SU(2)_T \otimes U(1)_Y
\end{equation}

where the subscripts indicate the conserved charges: the strong charge, or color C, the weak isospin T (or rather its third component T3) and the hypercharge Y. These quantities can be related to the electric charge Q through the Gell-Mann-Nishijima relation:
\begin{equation}
Q = \frac{Y}{2} + T_3
\end{equation}

In the quantum field framework, the elementary particles correspond to the irreducible representations of the G$_{SM}$ symmetry group. In particular, the particles are divided in two categories, fermions and bosons, according to their spin-statistics. Described by the Fermi-Dirac statistics, Fermions have half-integer spin and are sometimes called ``matter-particles". Bosons or ``force carriers" have integer spin, follow the Bose-Einstein statistics and mediate the interaction between fermions. The fundamental fermions and their quantum numbers are listed in Tab \ref{tab:SMParticles}\textcolor{red}{INSERT TABLE}
\begin{table}[!htbp]

\begin{tabular}{|l ccc|c|c|c|}
\hline
Generation: & I & II & III & T$_3$ & Y & Q \\
\hline
Leptonic doubles: & e$_L$ & e$_L$ & e$_L$ & 0 & -2 & -2 \\
Leptonic singlets: & e$_R$ & e$_R$ & e$_R$ & 0 & -2 & -2 \\
\hline
Quark doubles: & I & II & III & T$_3$ & Y & Q \\
Quark singlets: & I & II & III & T$_3$ & Y & Q \\
\hline
\end{tabular}
\caption{SM elementary fermions. The subscripts L and R indicate respectively the negative helicity (left-handed) and the positive helicity (right- handed).}
\label{tab:SMParticles}
\end{table}




Quarks can interact via all three the fundamental forces; they are triplets of SU(3)$_C$, that is they can exist in three different colors: C = R, G, B. If one chooses a base where u, c and t quarks are simultaneously eigenstates of both the strong and the weak interactions, the remaining eigenstates are usually written as d, s and b for the strong interaction and d', s' and b' for the weak interaction, because the latter ones are the result of a Cabibbo rotation on the first ones.
Charged leptons interact via the weak and the electromagnetic forces, while neutrinos only interact via the weak force. 
The gauge group univocally determines the number of gauge bosons that carry the interaction; the gauge bosons correspond to the generators of the group: eight gluons (g) for the strong interaction, one photon ($\gamma$) and three bosons (W$^\pm$, Z$^0$) for the electroweak interaction.
A gauge theory by itself can not provide a description of massive particles, but it is experimentally well know that most of the elementary particles have non-zero masses. The introduction of massive fields in the Standard Model lagrangian would make the theory non-renormalizable, and - so far - mathematically impossible to handle. This problem is solved in the Standard Model by the introduction of a scalar iso-doublet $\phi$(x), the Higgs field, see Tab \textcolor{red}{INSERT TABLE}, which gives masse to W$^\pm$ and Z$^0$ gauge bosons and to the fermions [1, 2].


\section{Neutrinos in the Standard Model}
\section{Beyond the Standard Model}
The discovery of neutrino oscillation marks  the beginning of a new, exciting era in neutrino physics: the era of physics Beyond the Standard Model (BSM) in the neutrino sector.
We are currently searching for new, deeper theories that can accommodate neutrinos with non-zero mass, while remaining consistent with the rest of the Standard Model. More excitingly, we need to probe these theories experimentally. 
\subsection{Neutrino Oscillations}
\subsection{Proton Decay}

% Have someone read this chapter
% Add citations
  % Wu experiment (parity violation)
  %Super Kamiokande, MINOS
  %No$\nu$a, MACRO 
  %SNO,Gallex,
  %SAGE,  KamLAND 
  % DAYA  Bay,   RENO
% Red stuff!

\chapter{The theoretical framework}

\section{The Standard Model}
The Standard Model (SM) of particle physics is the most accurate theoretical description of the subatomic world and, in general, one of the most precisely tested theories in the history of physics.  The SM describes the strong, electromagnetic and weak interactions among  elementary particles in the framework of quantum field theory, accounting for the unification of electromagnetic and weak interactions for energies above the  vacuum expectation value of the Higgs field. The SM does not describe gravity or general relativity.

The Standard Model is a gauge theory based on the local group of symmetry
\begin{equation}
G_{SM} = SU(3)_C  \otimes SU(2)_T \otimes U(1)_Y
\label{eq:SMGroup}
\end{equation}

where the subscripts indicate the conserved charges: the strong charge, or color C, the weak isospin T (or rather its third component T3) and the hypercharge Y. These quantities can be related to the electric charge Q through the Gell-Mann-Nishijima relation:
\begin{equation}
Q = \frac{Y}{2} + T_3.
\end{equation}

In the quantum field framework, the elementary particles correspond to the irreducible representations of the G$_{SM}$ symmetry group. In particular, the particles are divided in two categories, fermions and bosons, according to their spin-statistics. Described by the Fermi-Dirac statistics, fermions have half-integer spin and are sometimes called ``matter-particles". Bosons or ``force carriers" have integer spin, follow the Bose-Einstein statistics and mediate the interaction between fermions. The fundamental fermions and their quantum numbers are listed in Tab \ref{tab:SMParticles}.

\begin{table}[]
\centering
\begin{tabular}{|cccc|c|c|c|}\hline
Generation               & I                   & II                  & III                 & T                       & Y                   & Q                   \\\hline

\multirow{6}{*}{Leptons}                          &                     &                     &                     &                         &                     &                     \\
& $\begin{pmatrix}\ \nu_e\\ e \end{pmatrix}_L$ & $\begin{pmatrix}\ \nu_\mu\\ \mu \end{pmatrix}_L$ & $\begin{pmatrix}\ \nu_\tau\\ \tau \end{pmatrix}_L$ & $\begin{matrix}\ 1/2\\ -1/2 \end{matrix}$ & $\begin{matrix}\ -1\\ -1 \end{matrix}$ & $\begin{matrix}\ 0\\ -1 \end{matrix}$ \\
                         &                     &                     &                     &                         &                     &                     \\
                         & $e_R$         & $\mu_R$      & $\tau_R$                  & 0                      & -2                  & 1    \\
                         &                     &                     &                     &                         &                     &                     \\\hline
\multirow{7}{*}{Quarks} 
                         &                     &                     &                     &                         &                     &                     \\
& $\begin{pmatrix}\ u\\ d' \end{pmatrix}_L$ & $\begin{pmatrix}\ c\\ s' \end{pmatrix}_L$ & $\begin{pmatrix}\ t\\ b' \end{pmatrix}_L$ & $\begin{matrix}\ 1/2\\ -1/2 \end{matrix}$ & $\begin{matrix}\ 1/3\\ 1/3 \end{matrix}$ & $\begin{matrix}\ 2/3\\ -1/3 \end{matrix}$ \\
                         &                     &                     &                     &                         &                     &                     \\
& $\begin{matrix}\ u_R\\ d'_R \end{matrix}$ & $\begin{matrix}\ c_R\\ s'_R \end{matrix}$ & $\begin{matrix}\ t_R\\ b'_R \end{matrix}$ & $\begin{matrix}\ 0\\ 0 \end{matrix}$ & $\begin{matrix}\ 4/3\\ -2/3 \end{matrix}$ & $\begin{matrix}\ 2/3\\ -1/3 \end{matrix}$ \\
                         &                     &                     &                     &                         &                     &                    \\\hline
\end{tabular}
\caption{SM elementary fermions. The subscripts L and R indicate respectively the negative helicity (left-handed) and the positive helicity (right-handed).}
\label{tab:SMParticles}
\end{table}

Quarks can interact via all three the fundamental forces; they are triplets of SU(3)$_C$, that is they can exist in three different colors: C = R, G, B. If one chooses a base where $u$, $c$ and $t$ quarks are simultaneously eigenstates of both the strong and the weak interactions, the remaining eigenstates are usually written as $d$, $s$ and $b$ for the strong interaction and $d'$, $s'$ and $b'$ for the weak interaction, because the latter ones are the result of a Cabibbo rotation on the first ones.
Charged leptons interact via the weak and the electromagnetic forces, while neutrinos only interact via the weak force. 
The gauge group univocally determines the number of gauge bosons that carry the interaction; the gauge bosons correspond to the generators of the group: eight gluons (g) for the strong interaction, one photon ($\gamma$) and three bosons (W$^\pm$, Z$^0$) for the electroweak interaction.
A gauge theory by itself can not provide a description of massive particles, but it is experimentally well know that most of the elementary particles have non-zero masses. The introduction of massive fields in the Standard Model lagrangian would make the theory non-renormalizable, and - so far - mathematically impossible to handle. This problem is solved in the Standard Model by the introduction of a scalar iso-doublet $\Phi(x)$, the Higgs field, which gives mass to W$^\pm$ and Z$^0$ gauge bosons through the electroweak symmetry breaking and to the fermions through Yukawa coupling \cite{Higgs1964,Higgs19642}.  The discovery of the Higgs boson in 2012 by the LHC experiments \cite{201230,Aad2012} marked the ultimate confirmation  of a long history of Standard Model successful predictions.

\section{Neutrinos:  tiny cracks in the Standard Model}
\subsection{Neutrinos in the Standard Model}
Neutrino were introduced in the SM as left-handed massless Weyl spinors.
The Dirac equation of motion
\begin{equation}
(i\gamma^ \mu \partial_\mu - m) \psi = 0
\end{equation}
for a fermionic field 
\begin{equation}
 \psi =  \psi_L +  \psi_R
\end{equation}
is equivalent to the equaitons
\begin{equation}
i\gamma^ \mu \partial_\mu  \psi_L = m \psi_R
\label{eq:15}
\end{equation}
\begin{equation}
i\gamma^ \mu \partial_\mu  \psi_R = m \psi_L
\label{eq:16}
\end{equation}

for the chiral fields $\psi_R$ and $\psi_L$, whose evolution in space and time is coupled through the mass $m$.
If the fermion is massless, the chiral fields decouple and the fermion can be described by a single Weyl spinor with two independent components~\cite{Weyl:10.2307}. Pauli initially rejected the description of a physical particle through a single Weyl spinor because of its implication of parity violation. In fact, since the spatial inversion operator throws $\psi_R \leftrightarrow \psi_L$, parity is conserved only if the both the chiral components exist at the same time.  For the neutrino introduction in the SM, experiments came in help of the theoretical description.  The constraint of parity conservation weakened after Wu's experiment in 1957 \cite{PhysRev.105.1413}. Additionally,  there was no experimental indication for massive neutrinos nor evidence of interaction via the neutrino right-handed component.% neutrinos likely interacted only via the left-handed component. 

The symmetry group $SU(2)_T \otimes U(1)_Y$ is the only group relevant for neutrino interactions. The SM electroweak lagrangian is the most general renormalizable lagrangian invariant under the local symmetry group $SU(2)_T \otimes U(1)_Y$. The lagrangian couples the weak isotopic spin doublets and singlets described in \ref{tab:SMParticles} with the gauge bosons  $A^{\mu}_{a}$ ($a$ $=$ 1,2,3) and $B^{\mu}$, and Higgs doublet $\Phi(x)$:

\begin{eqnarray}
\lefteqn{\mathcal{L} = i\sum_{\alpha=e,\mu,\tau} \bar{L}'_{\alpha L}  \slashed D L'_{\alpha L} + 
 i\sum_{\alpha=1,2,3} \bar{Q}'_{\alpha L}  \slashed D Q'_{\alpha L} {}}
 \nonumber\\
 & & {} + i\sum_{\alpha=e,\mu,\tau} \bar{l}'_{\alpha R}  \slashed D l'_{\alpha R} + i\sum_{\alpha=d,s,b} \bar{q}'^D_{\alpha R}  \slashed D q'^D_{\alpha R} + i\sum_{\alpha=u,c,t} \bar{q}'^U_{\alpha R}  \slashed D q'^U_{\alpha R}
 \nonumber\\
 & & {} -\frac{1}{4}A_{\mu \nu}A^{\mu \nu} - \frac{1}{4}B_{\mu \nu}B^{\mu \nu}
 \nonumber\\
 & & {} +(D_{\rho}\Phi)^\dagger(D^{\rho}\Phi) - \mu^2\Phi^\dagger\Phi - \lambda(\Phi^\dagger\Phi)^2 
 \nonumber\\
 & & {} -\sum_{\alpha,\beta=e,\mu,\tau} \Big(Y'^l_{\alpha\beta}\bar{L}'_{\alpha L}  \Phi l'_{\beta R} + Y'^{l*}_{\alpha\beta}\bar{l}'_{\beta R}  \Phi^\dagger L'_{\alpha L}\Big)
  \nonumber\\
 & & {} -\sum_{\alpha=1,2,3} \sum_{\beta=d,s,b} \Big(Y'^D_{\alpha\beta}\bar{Q}'_{\alpha L}  \Phi q'^D_{\beta R} + Y'^{D*}_{\alpha\beta}\bar{q}'^D_{\beta R}  \Phi^\dagger Q'_{\alpha L}\Big)
  \nonumber\\
 & & {} -\sum_{\alpha=1,2,3} \sum_{\beta=u,c,t} \Big(Y'^U_{\alpha\beta}\bar{Q}'_{\alpha L}   \widetilde{\Phi} q'^U_{\beta R} + Y'^{U*}_{\alpha\beta}\bar{q}'^U_{\beta R} \widetilde{\Phi}^\dagger Q'_{\alpha L}\Big).
\end{eqnarray}

The first two lines of the lagrangian summarize the kinetic terms for the fermionic fields and their coupling to the gauge bosons $A^{\mu\nu}_a$, $B^{\mu\nu}$ \footnote{In gauge theories the ordinary derivative $\partial_\mu$  is substitued with the covariant derivative $D_\mu$. Here $D_\mu = \partial_\mu + igA_\mu \cdot I + ig'B_\mu\frac{Y}{2}$, where I and Y are the SU(2)$_L$ and U(1)$_Y$ generators, respectively.}.
The third line describes the kinetic terms and the self-coupling terms of the gauge bosons. The forth line is the Higgs lagrangian, which results in the spontaneous symmetry breaking. The last three lines describe the Yukawa coupling between fermions and the Higgs field, origin of the fermion's mass.

The coupling between left-handed and right-handed field generates the mass term for fermions. The SM assumes only left-handed components for neutrinos, thus implying zero neutrino mass. Since any linear combination of massless fields results in a massless field, the flavor eigenstates are identical to the mass eigenstates in the SM.

\subsection{Neutrino Oscillations}
The determination of the flavor of a neutrino dynamically arises from the corresponding charged lepton associated in a change current interaction; for example, a $\nu_e$ is a neutrino which produces an $e^-$, a $\bar\nu_\mu$ is a neutrino which produces a $\mu^+$, $etc$. 
The neutrino flavor eigenstates $\ket{\nu_\alpha}$,  with $\alpha = e,\mu,\tau$, are orthogonal to each other and form a base for the the weak interaction matrix.

Overwhelming experimental data show neutrinos change flavor during their propagation~\cite{Patrignani:2016xqp}. This phenomenon, called ``neutrino oscillations",  was predicted first by Bruno Pontecorvo in 1957 ~\cite{Pontecorvo:1967fh}.  Neutrino oscillations are possible only if the neutrino flavor eigenstate are not identical to the mass eigenstates, thus resulting in the first evidence of physics beyond the Standard Model.  A minimal extension of the Standard Model introduces three mass eigenstates, $\ket{\nu_i}$ ($i = 1,2, 3$), whose mass $m_i$ is well defined. 
The unitary Pontecorvo-Maki-Nakagawa-Sakata matrix transforms the spinor wave functions ($\psi$) of each component  between flavor and mass bases as follows

\begin{equation} 
\sum \psi_\alpha \ket{\nu_\alpha} =  \sum \psi_i \ket{\nu_i}, \rightarrow \psi_\alpha =  U_{PMNS} \psi_i, 
\end{equation}

with

\begin{equation}
U_{PMNS}=\\
\left[ 
\begin{array}{ccc}
c_{12} & s_{12} & 0\\
-s_{12} & c_{12} & 0\\
0 & 0 & 1\\
\end{array}
\right]
\left[ 
\begin{array}{ccc}
c_{13} & 0 &s_{13}e^{-i\delta} \\
0 & 1 & 0\\
-s_{13}e^{-i\delta}& 0 & c_{13}\\
\end{array}
\right]
\left[
\begin{array}{ccc}
1 & 0 & 0\\
0 & c_{23} & s_{23} \\
0& -s_{23} & c_{23} \\
\end{array}
\right]
\left[ 
\begin{array}{ccc}
e^{i\alpha_1} & 0 & 0\\
0& e^{i\alpha_2} & 0\\
0 & 0 & 1\\
\end{array}
\right]
\label{eq:PMNS}
\end{equation}





where $c$ e $s$ stand respectively for cosine and sine of the corresponding  mixing angles ($\theta_{12}$, $\theta_{23}$ and $\theta_{13}$), $\delta$ �is the Dirac CP violation phase, $\alpha_1$ and $\alpha_{2}$   is the eventual Majorana CP violation phases.  Experimental results on neutrino oscillations are generally reported in terms of the mixing angles and of the squared mass splitting $\Delta m^2_{ab} = m^2_{a} - m^2_{b}$, where $a$ and $b$ represent the mass eigenstates. A summary of the current status of experimental results, albeit partial, is given in table \ref{tab:nuosc}.


\begin{table}[htpb]
\centering
\caption{Summary of experimental results on neutrino oscillation parameters. \textcolor{red}{ADD CITATIONS}}
\label{tab:nuosc}
\begin{tabular}{|r|c|c|c|}
\hline
        & Value   & Precision & Experiment \\
\hline
$\theta_{23}$            & 45$^\circ$                      & 9.0\%   & Super Kamiokande, MINOS,\\
$\Delta m^2_{23}$    & 2.5 $10^{-3}$ eV$^2$   & 1.8\% &  No$\nu$a, MACRO \\
\hline
$\theta_{12}$            & 34$^\circ$                     & 5.8\% &  SNO,Gallex,\\
$\Delta m^2_{12}$    & 7.4 $10^{-5}$ eV$^2$   & 2.8\% &  SAGE,  KamLAND \\
\hline
$\theta_{13}$            &   9$^\circ$                    & 4.7\%  & DAYA  Bay, \\
$\Delta m^2_{13}$    & 2.5 $10^{-3}$ eV$^2$   & 1.8\% &  RENO\\
\hline
\end{tabular}
\end{table}

\subsection{Make up of Neutrino Interactions}
All neutrino experiments involving the detection of single neutrinos are concerned with neutrino interactions (and neutrino cross sections) on nuclei. 
Given the invisible nature of the neutrino, characterizing the products of its interaction is the only method to a) assess the neutrino presence, b) detect its flavor in case of a charge current interaction and c) eventually reconstruct its energy. 

Historically, neutrino interactions with the nucleus in the GeV region are divided into three categories as a function of increasing neutrino energy: quasi elastic (QE), resonant and deep inelastic scattering (DIS). All current and forthcoming oscillation experiments live in the 0.1-10 GeV transition region, which encompasses the energy where the QE neutrino-nucleus interaction transitions into resonant scattering and the energy  where resonance scattering  transitions in to DIS. 
 Neutrino and antineutrino QE charge current scattering refers to the process $\nu_\mu n \rightarrow \mu^- p$ and $\bar\nu_\mu p \rightarrow \mu^+ n$ where a charged lepton and single nucleon are ejected in the elastic interaction and the target nucleus remains at ground state. %of a neutrino (or antineutrino) with a nucleon in the target material.  
 Resonant scattering refers to an inelastic collision producing a nucleon excited state ($\Delta$,N*); the excited resonance quickly decays, most often to a nucleon and single-pion final state. DIS refers to the head-on collision between the neutrino and a parton inside the nucleon, producing hadronization and subsequent  abundant production of mesons and nucleons.  In addition to such interactions between the neutrino and a single component of the nucleus, neutrinos can also interact with the nucleus as a whole, albeit more rarely, a well documented process called coherent meson production scattering \cite{Vilain1993}; the signature of such process is the production of a distinctly forward-scattered single meson final state, most often a pion.  This simple picture of neutrino interactions works rather well for scattering 
off of light targets, such as the H$_2$ and D$_2$ of bubble chamber experiments \cite{RevModPhys.84.1307}, but the complexity of the nuclear structure for heavier nuclei such as argon complicates this model. 



As we will discuss in Chapter \ref{ch:2}, the properties of argon make it a good candidate for interacting medium in neutrino experiments; in particular the density of its interaction centers  augments the yield of neutrino interactions and allows for relatively compact detectors. Though, the choice of a relatively heavy nuclear target comes at the cost of enhancing  nuclear effects which modify the kinematic and final state of the neutrino interaction products.  

Nuclear effects can potentially affect the neutrino event rates, nucleon emission, neutrino energy reconstruction, and neutrino/antineutrino ratios, carrying deep implications for oscillation experiments. 
Even in the case of ``simple" QE scattering  nuclear effects can impact the size and shape of the cross section as well as the final state kinematics and topology. Intra-nuclear hadron rescattering and correlation effects between the target nucleons can cause the ejection of additional nucleons in the final state. In case of resonant and DIS scattering, the hadronic interactions of meson and nucleons produced in the decay of the resonance or during hadronization  complicate this picture even more.
A large source of uncertainty in modeling  nuclear effects in neutrino interactions come from mesons interactions (and re-interactions) in the nucleus, e.g., pion re-scattering, charge exchange, and absorption.

A renewed interest for neutrino cross section measurements surged in recent year, along with lively discussion on the data reporting; the historical method of reporting the neutrino cross section as a function of the neutrino energy or momentum transferred shakes under the weight of its dependency to the chosen nuclear model.   On one hand, correcting for nuclear effects in neutrino interaction can introduce unwanted sources of uncertainty and model dependence due to the mis-modeling of the meson interactions. On the other, avoiding this correction however makes a comparison between neutrino interactions on different target nuclei extremely difficult.

Data on neutrino scattering off many different nuclei are available for both charged current (CC) and neutral current (NC) channels, as summarized here \cite{RevModPhys.84.1307}. A summary of is reported in Figure \textcolor{red}{PLOT} where the (NUANCE) [46] event generator provides a theoretical comparator. 



\section{Beyond the Standard Model}
The discovery of neutrino oscillation and its implication of non-zero neutrino mass mark  the beginning of a new, exciting era in neutrino physics: the era of physics Beyond the Standard Model (BSM) at the intensity frontier.
We are currently searching for new, deeper theories that can accommodate neutrinos with tiny but non-zero masses, while remaining consistent with the rest of the Standard Model. 

\subsection{Open Questions in Neutrino Physics}
On one hand, the last three decades of experiments in neutrino oscillations brought spectacular advancements in the understanding of the oscillations pattern, measuring the neutrino mixing angles and mass splitting with a precision of less than 10\%.  On the other, it opened the field for a series of questions needing experimental answers. 


%%%%%%%%%%
\textbf{Sterile neutrinos.} Hints to the existence of at least one additional neutrino, in the form of various anomalies, have been puzzling physicists almost from the beginning of neutrino oscillation searches. 
Originally designed to look for evidence of neutrino oscillation, the Liquid Scintillator Neutrino Detector (LSND) \cite{Athanassopoulos1997} provided a first conflicting result with the Standard Model expectation of only three neutrino flavors. A second conflicting result has also been provided by the MiniBooNE experiment \cite{AGUILARAREVALO200928}.
The  LSND and MiniBooNE $\nu_e$ and $\bar\nu_e$ appearance results, known as the ``LSND and MiniBooNE anomalies" \cite{Aguilar:2001ty, Athanassopoulos:1997pv, Aguilar-Arevalo:2013pmq}, may be interpreted under the assumption of a new right-handed neutrino. The additional neutrino needs to be ``sterile", i.e needs not to couple with the electroweak force carriers, in order to meet the constraint imposed by  the measurement of the width of the Z boson~\cite{2006257}.  The new sterile neutrino would mainly be composed of a heavy neutrino $\nu_4$ with mass $m_4$ such that  $m_1, m_2, m_3 \ll m_4$ and  $\Delta m^2= \Delta m^2_{14} \sim [0.1 - 10]$ eV$^2$.
The introduction of sterile neutrinos is an appealing line of thinking, since this renormalizable generalization of the Standard Model has the potential to impact long standing questions in high energy physics and cosmology: light sterile neutrinos are candidates for dark matter particles and there are ideas that the theory could be adjusted to explain the baryon asymmetry of the Universe via leptogenesis \cite{1063-7869-57-5-503}. 

%%%%%%%%%%
\textbf{CP Violation In Lepton Sector.} The measurement of non-zero value for the oscillation parameter $\theta_{13}$ allows the exploration of low-energy CP violation in the lepton sector at neutrino long baseline oscillation experiments, enabling the possibility to measure the Dirac CP-violating phase $\delta$. Exciting theoretical results tie $\delta$ directly to the generation of the baryon asymmetry of the Universe at the Grand Unified Theory scale \textcolor{red}{a couple of cit would be nice}. According to the theoretical model described in \cite{PASCOLI20071}, for example, leptogenesis can be achieved if $|\sin\theta_{13} \sin \delta| > 0.11$, i.e. $\sin \delta > 0.7$.\\
The asymmetry in the oscillation probability of neutrinos and antineutrinos is the observable sensitive to the Dirac CP-violating phase $\delta$ leveraged in neutrino oscillation experiments. Using the parameterization of the PMNS matrix shown in Equation \ref{eq:PMNS},  the difference in the probability of $\nu_e \rightarrow \nu_\mu$ oscillation and the probability of $\bar\nu_e \rightarrow \bar\nu_\mu$ oscillation can be parametrized as follows \cite{Cervera2000},
\begin{equation}
P_{\nu_e\rightarrow\nu_\mu} - P_{\bar\nu_e\rightarrow\bar\nu_\mu} = J \cos\Big(\pm\delta - \frac{\Delta_{31}L}{2}\Big) \sin\Big(\frac{\Delta_{21}L}{2}\Big)\sin\Big(\frac{\Delta_{31}L }{2}\Big)
\end{equation}
where 
\begin{equation}
J = \cos\theta_{13}\sin2\theta_{13}\sin2\theta_{12}\sin2\theta_{23}
\end{equation}
is the Jarlskog invariant \cite{Jarlskog1985}, $L$ the neutrino baseline and $\Delta_{ab}$ a factor proportional to the sign and magnitude of the mass splitting. 
From these equations, it is clear how the relative  large value of $\theta_{13}$ is a happy accident necessary not to completely suppress the sensitivity to CP violation.  The equations also show how the sensitivity to $\delta$ is tied to the measurement of the least precisely measured mixing angle,  $\theta_{23}$ (via the $\sin2\theta_{23}$ term) and to an other unknown quantity, the neutrino ``mass hierarchy" (via the $\Delta_{ab}$ terms). The precise determination of $\theta_{23}$ is often referred as to ``the octant problem". Current experimental results \textcolor{red}{cite NOVA T2K} are consistent with  $\theta_{23}=45^\circ$, which would imply maximal mixing between $\nu_\mu$ - $\nu_\tau$, hinting to an intriguing new symmetry. Therefore, a precise measurement of $\theta_{23}$  is of great interest for theoretical models of quark-lepton universality [59,84,85,86,87,88] , whose  quark and lepton mixing matrices are proportional to the deviation of $\theta_{23}$ from $45^\circ$. 


\textbf{Neutrino mass hierarchy.} The  ``mass hierarchy" problem refers to the unknown ordering of the value of absolute mass of the neutrino mass eigenstates. Current oscillation experiments are sensitive only to the magnitude of the mass splitting, but not to its sign. \textcolor{red}{cite hints} In a framework where the lightest neutrino mass (arbitrarily) corresponds to the first eigenstate $m_1$, it is unknown whether $m_2 - m_1 < m_3 - m_1$ (Normal Hierarchy) or $m_2 - m_1 > m_3 - m_1$ (Inverted Hierarchy). The mass hierarchy affects not only the sensitivity to CP violation searches in long baseline oscillation experiments, but also the sensitivity to determine whether neutrinos are Majorana particles in neutrinoless double beta decay experiments.

\textbf{Majorana or Dirac?}
Evidence of neutrino oscillations demand the introduction in the theory of a mechanism which can give mass to the neutrinos.  This mechanism should possibly also explain why neutrino masses are at least six orders of magnitude lower than the electron mass (the second lightest SM fermion).
In a description of neutrinos as Dirac 4-component spinors, the neutrino field acquires mass  via the Higgs mechanism as any other fermion of the SM. In this case, the neutrino mass is given by 
%\begin{equation} 
$m_a = \frac{y^\nu_a v}{\sqrt2}$, %\end{equation} 
where $v$ is the Higgs VEV and   $y^\nu_a$ is the Yukawa coupling between the Higgs and the neutrino. The smallness of neutrino masses can only be pinned on a tiny Yukawa coupling which is not justified by the theory.\\
In 1937, Majorana demonstrated that the introduction of a two components spinor is sufficient to describe a massive fermion \cite{Majorana1937}. The Dirac equations of motion for the chiral fields (equations \ref{eq:15} and \ref{eq:16}) hold true in the case of two components spinor under the assumption that the chiral components $\psi_R$ and $\psi_L$ are correlated through the charge conjugation matrix $\mathcal{C}$, $\psi_R = \mathcal{C}\bar\psi_L$, thus the theory is applicable only to neutral fermions.  Neutrinos are the only neutral elementary particles in the SM -- the only possible Majorana particle candidate. This theory constructs a  neutrino Majorana mass term $\mathcal{L}_5 $ of the following form in the Higgs unitary gauge
\begin{equation}
\mathcal{L}_5 = \frac{1}{2}\frac{gv^2}{\mathcal{M}}\nu^T_L\mathcal{C^\dagger}\nu_L,
\label{eq:majoMass}
 \end{equation}
 where $g$ is the coupling coefficient, $v$ the Higgs VEV and $\mathcal{M}$ a constant with the dimension of the mass proportional to the scale of new physics. The  $\mathcal{L}_5$ term would introduce a non-renormalizable term in the lagrangian, since it has dimensions of energy to the fifth power.
This is not allowed in the SM term; however, the existence of such terms is plausible if the we consider the SM an effective theory at low energy, manifestation of the symmetry breaking of a Grand Unified Theory (GUT) at higher energy, and not the definitive theory.
The mass term in eq \ref{eq:majoMass} implies the neutrino mass to be  $m = \frac{g v^2}{\mathcal{M}}$. The coupling coefficient can be of the order of any other fermion's coupling coefficient, since the smallness of neutrino masses is achieved by the big value of the new physics mass scale alone. This vanilla formulation is the conceptual basis for many flavors of GUT-based \emph{seesaw mechanism} ~\cite{Yanagida1980}, which we will not discuss here in any detail. However, it is fascinating how the puzzle of the neutrino mass hints to the existence of a deeper and more complete theory.\\
From a kinematic point of view, Dirac and Majorana neutrinos satisfy the same energy-momentum dispersion relationship. Thus, it is impossible to discern the neutrino nature through kinematic effects such as neutrino oscillations. Neutrinoless double beta decay searches are the most promising way to understand the nature of the neutrino and are therefore subject of great theoretical and experimental interest. Observation of the lepton number violating process  $0\nu\beta\beta$  would imply neutrinos have a Majorana component. Depending on the mass hierarchy, the theory also predicts  $0\nu\beta\beta$  exclusion regions and confirmation of the sole Dirac component for neutrinos\textcolor{red}{find CIT}.\\


Critical challenges await  the next decade of experimental neutrino physics. % will entail an even deeper understanding of the neutrino mixing pattern, investigation of the neutrino mass origin, the determination of the number of neutrinos and their nature, and the measurement of CP violation in the lepton sector. 
%%% Plug to LArTPC %%%%%%%%%%%%
Following the recommendation of the latest Particle Physics Project Prioritization Panel  \cite{P5}, the US  is dedicating substantial resources to the development of a short- and long- baseline neutrino program to address many of these fundamental questions.  This program pivots on the Liquid Argon Time Projection Chamber (LArTPC) detector technology which will be described in \ref{ch:1}.  

The main goals of these research programs include:
\begin{itemize}
\item[-] Assessment of the existence of right-handed sterile neutrinos. 
\item[-] Determination of the sign of $\Delta m^2_{13}$ (or $\Delta m^2_{23}$), i.e., the neutrino mass hierarchy.
\item[-] Determination of the octant, i.e.  whether $\theta_{23}$ is maximal.
\item[-] Determination the status of CP symmetry in the lepton sector.
\end{itemize}

\subsection{Towards a more fundamental theory: GUTs}
Despite its highly predictive power, a number of conceptual issues arise in the SM which disfavor it to be a good candidate for a fundamental theory.

The SM rather complex group structure, where a gauge group is formed with the direct product of other three groups as shown in eq. \ref{eq:SMGroup},  is unexplained. Also, the SM fails to include a suitable dark matter candidate and a mechanisms that accounts for the baryon asymmetry of the universe. Within the SM, a total of 25 parameters remain seemingly arbitrary and need to be fitted to data: 3 gauge couplings, 9 charged fermion masses, 3 mixing angles and one CP phase in the CKM matrix, the Higgs mass and quartic coupling, $\theta_{QCD}$, 3 neutrino masses, 3 neutrino mixing angles, 1 Dirac phase and, eventually,  2 Majorana phases.

\subsubsection{Nucleon decay}\label{sec:theoryPDK}
Baryon number is accidentally conserved in the Standard Model. Even though no baryon number violation has been experimentally observed thus far, no underlying symmetry in line with the Noether paradigm \cite{Noether1971} explains its conservation. Almost all Grand Unified Theories predict at some level baryon number violation in the form of nucleon decay on long time-scales.  Given the impossibility to reach grand unification energy scales with collider experiments ($\sqrt{s} > 10^{15}$ GeV),  an indirect proof of GUT is needed. The experimental observation of nucleon decay may be the only viable way to explore these theories and it is therefore a subject of great interest \cite{Adams:LBNE}. %Both experiments and theory indeed suggest the energy scale for convergence of the running coupling constants of the Standard Model to be over $10^{15}$ GeV. This energy scale seems impossible to access by any foreseeable accelerator experiment, leaving baryon number violation  to be the only testable process. 


\section{Motivations for Hadronic Cross Sections in Argon}
%%%%%%%%%%%%%%%%%%%%%%%%%%%%%%%%%%%%%%%%%%%%%%%%%%%%%
%%%%%%%%%%%%%%%%%%%%%%%%%%%%%%%%%%%%%%%%%%%%%%%%%%%%%
%%%%%%%%%%%%%%%%%%%%%%%%%%%%%%%%%%%%%%%%%%%%%%%%%%%%%
%%%%% ---------------------------------------- Hadronic Cross sections ---------------------------------------- %%%%%
%%%%% ----------------------------------------     Pion Cross Section     ---------------------------------------- %%%%%

\subsection{Pion-Argon Total Hadronic Cross Section}
This section outlines the importance of the pion-argon total hadronic cross section. We start by discussing the measurement in the context of neutrino interaction searches and of light mesons interaction with nuclei studies. We then describe the signal signature and historical measurements of pion-nucleus cross section, as well as the implementation of this cross sections in the current version of the simulation package used by LArIAT.
\subsubsection{$\pi^{-}$Ar Cross section in the Context of Neutrino Searches}
\subsubsection{$\pi^{-}$Ar Cross Section in the Context of Light Mesons Interaction with Nuclei}
\subsubsection{Signal Signatures}
\subsubsection{Previous measurements: Lighter and Heavier Nuclei}
\subsubsection{Pion Interaction Cross Section for thin target in Geant4}


%%%%%%%%%%%%%%%%%%%%%%%%%%%%%%%%%%%%%%%%%%%%%%%%%%%%%
%%%%%%%%%%%%%%%%%%%%%%%%%%%%%%%%%%%%%%%%%%%%%%%%%%%%%
%%%%%%%%%%%%%%%%%%%%%%%%%%%%%%%%%%%%%%%%%%%%%%%%%%%%%
%%%%% ---------------------------------------- Hadronic Cross sections ---------------------------------------- %%%%%
%%%%% ----------------------------------------     Kaon Cross Section     ---------------------------------------- %%%%%

\subsection{Kaon-Argon Total Hadronic Cross Section}
This section outlines the importance of the kaon-argon total hadronic cross section. We start by discussing the measurement in the context of nucleon decay searches and of light mesons interaction with nuclei studies. We then describe the signal signature and historical measurements of kaon-nucleus cross section, as well as the implementation of this cross sections in the current version of the simulation package used by LArIAT.

\subsubsection{K$^{+}$Ar Cross section in the Context of Nucleon Decay Searches}
In case of nucleon decay discovery, the dominant decay mode may uncover additional information about the GUT type.  Supersymmetric GUTs \cite{Dimopoulos:1981dw,Bajc20161} prefer the presence of kaons in the products of the decay, e.g. $p\rightarrow K^+\bar{\nu}$  (see fig \ref{fig:MandatoryFeynmannDiagrams}, left).
Gauge mediated GUTs, in which new gauge bosons are introduced that allow for the transformation of quarks into leptons, and vice versa, prefer the mode $p\rightarrow e^+\pi^0$ (see fig \ref{fig:MandatoryFeynmannDiagrams}, right).



\begin{figure}[hbpt]
\centering
\includegraphics[width=6.5in]{Chapter-1/Images/MandatoryFeynmannDiagrams.png}
\caption{Feynman diagrams for proton decay ``golden modes": $p \rightarrow K^+ \bar{\nu}$ for supersymmetric GUTs on the left and  $p \rightarrow e^+ \pi^0$ for gauge-mediated GUTs  on the right.}
\label{fig:MandatoryFeynmannDiagrams}
\end{figure}


LArIAT tiny active volume makes it impossible for the experiment to place competitive limits on nucleon decay searches.  However,  LArIAT provides excellent data to characterize kaons in liquid argon for the ``LAr golden mode", $p \rightarrow K^+ \bar{\nu}$.  The result of these studies will affect future proton decay searches in LArTPCs.  Previous work has been done to assess the potential identification efficiency for different decay modes in a LArTPC \cite{Bueno2007}, but, as the time of this  writing, no study of kaon selection efficiency in LArTPCs has been performed on data. 
The K$^+$-Ar interaction cross section has never been measured before and can affect the possibility of detecting and measuring kaons when produced in a proton decay event. 
Kaon interactions with argon can distort the kaon energy spectrum as well as change the topology of single kaon events. In a LArTPC, non-interacting kaons appear as straight tracks with a high ionization depositions at the end (Bragg peak). The topology of interacting kaons can be quite different. In case of elastic scattering, a distinct kink will be present in the track. In case of inelastic scattering the Bragg peak will not be present and additional tracks will populate the event.
Performing the total hadronic K$^+$-Ar cross section measurement on data serves the double purpose of identifying the rate of ``unusual" topologies (kinks and additional tracks) and of developing tools for kaon tracking in LAr.

\subsubsection{K$^{+}$Ar Cross section in the Context of Light Mesons Interaction with Nuclei}
\label{sec:theoryStrangeMeson}
The intrinsic value of the total hadronic K$^{+}$-Ar cross section measurement is that kaon interactions complement the measurements of $\pi$ interactions as a probe of  hadron interaction inside the nucleus in the strange sector.  
\textcolor{red}{High theoretical interest in probing constituent quark model of nuclear structure with
KAON-NUCLEON INTERACTIONS CHIEDI REFERENZE A FLAVIO}


%Total cross sections for the interaction of mildly relativistic kaons with several nuclei were derived from transmission experiments performed at the alternating-gradient synchrotron in Brookhaven National Laboratory. The high precision of these cross sections ~about 1\% led to analyses of the data in terms of KN nucleus potentials, based on the expectation that the KN nucleus interaction is simply related to the KN interaction. In particular, in this energy range the KN interaction does not vary strongly with energy and together with the relative weakness of the interaction, one expects that optical potentials close to the ??tr?? approximation ~see below will be capable of describing the data. However, all such analyses showed disagreement between calculation and experiment at the level of 5?15\%, which caused speculations about modifications in the nuclear medium of the KN interaction @5?8#. 

\subsubsection{Signal Signatures}\label{sec:KSignalSignature}
%%%%%%%%%%%%%%%%%%%%%%%%%%%%%%%%%%%%%%%%%%%%%%%%%%%%%%%%%%%%%%%%%%%%%%%%%%%%%%%%%
The interaction of a mildly relativistic charged kaon with an argon nucleus is determined largely by the strong force. The total hadronic K$^{+}$-Ar interaction cross section is defined as the one related to the single (hadronic) process driven only by the strong interaction.
In this case, ``total" indicates all strong interactions regardless of the final state. This condition purposefully includes both elastic and inelastic (reaction) channels. Indeed, the total cross section section can be then decomposed into
$$\sigma_{Tot} = \sigma_{Elastic}+ \sigma_{Reaction}.$$


%For this analysis, kaons are selected from the LArIAT beamline in the momentum range between \textcolor{red}{500} MeV/c and \textcolor{red}{1000} MeV/c (see Fig \ref{fig:TOFK}).

%\begin{figure}[hpbt]
%\centering
%\includegraphics[width=5in]{Chapter-1/Images/KaonTOF}
%\caption{Time of flight versus momentum distributions as produced by the LArIAT TOF and Wire Chambers systems. The Kaon population lies between the proton and the muon/pion populations, allowing PID of Kaons in the beam line.  }
%\label{fig:TOFK}
%\end{figure}

For the LArIAT cross section analysis, the kaons considered span a momentum inside the TPC from 800 MeV/c and 100 MeV/c. In this energy range, the relevant K-Nucleon interactions are according to \cite{fesbach1992theoretical}:

\begin{align}
K^{+} + N &\rightarrow K^{+} + N\textit{ (elastic)}\\
K^{+} + n &\rightarrow K^{0} + p\textit{ (elastic)}\\
K^{+} + N &\rightarrow K + N + \pi \textit{ (inelastic)}\\
K^{+} + N &\rightarrow K^{*} + N\textit{ (inelastic)}.
\end{align}

\subsubsection{Previous Measurements: Lighter and Heavier Nuclei}
In general, measurements on kaon cross sections are  extremely scarce. The measurement of the kaon interaction cross section would bring the additional benefit of reducing the uncertainties associated  with hadron interaction models adopted in MC simulations for argon targets, beneficial for both proton decay studies and kaon production from neutrino interaction studies, where the  uncertainties for final state interaction models are big \cite{Drakoulakos:2004gn}. 

Figure \ref{fig:Friedmann} shows a 1997 measurement on several elements as performed by  Friedmann et al.  \cite{Friedman:1997eq}. As a reference, this paper measures a $\sigma_{Tot}$ for Si of  366.5  $\pm$  4.8 mb and a $\sigma_{Tot}$ for Ca of 494.6  $\pm$ 7.7 mb at 488 MeV/c.  The cross section for argon is expected to lie in between these two measurements. 
Additional data on the kaon cross section are provided by Bugg et al. \cite{PhysRev.168.1466}. Bugg performs a measurement of the total 
K$^+$ and K$^-$ cross sections on protons and deuterons over the range of 0.6-2.65 GeV/c, as well as a measurement of the total K$^+$ and K$^-$  cross sections on carbon for a number of momenta; the results of this paper on carbon are reported in Figure \ref{fig:Bugg}.



\begin{figure}
\captionsetup{justification=raggedright}  
	\begin{minipage}[t]{.53\textwidth}  
	  \centering  
	   \includegraphics[width=3in]{Chapter-1/Images/Friedmann.png}
	   	        \caption{Ratios between experimental and calculated cross sections as from \cite{Friedman:1997eq}. Top: Total cross sections. \\Bottom: reaction cross sections.}
        \label{fig:Friedmann}
	\end{minipage}%  
	\begin{minipage}[t]{0.53\textwidth}  
	  \centering  
	\includegraphics[width=3in]{Chapter-1/Images/Bugg.png}
        \caption{Total K$^+$  and K$^-$ cross sections on carbon as from \cite{PhysRev.168.1466}.}
        \label{fig:Bugg}
	\end{minipage}
	\par
\end{figure}



%%%%%%%%%%%%%%%%%%%%%%%%%%%%%%%%%%%%%%%%%%%%%%%%%%%%%%%
%% PRETTY EVENT DISPLAY WITH TEXT, NOT SURE IF USEFUL
%\begin{figure}[h!]
%\centering
%\includegraphics[width=6.5in]{Chapter-1/Images/KLariat.png}
%\caption{LArIAT Data $K^+$ candidate. $K^+$ enters TPC, undergoes a hadronic scatter, and then decays into $\pi^+$ and $\pi^0$. The the $\pi^0$ decays into 2 photons while the $\pi^+$ stops quickly in the TPC. Collection plane view.}
%\label{Fig:KLariat}
%\end{figure}

%Fig \ref{Fig:KLariat} shows a $K^+$ candidate event in the LArIAT TPC. Following the kaon candidate track from left to right, two important elements are visible by eye: a change in the K momentum due to hadronic scatter and a Bragg peak by the end of the track due to an augment of ionization as the kaon slows down in the TPC. The track "kink" is only visible thanks to the millimetric spacial resolution of the TPC, while the Bragg shows the calorimetric power of this technology. The kaon in this event decays hadronically into $\pi^+$ and $\pi^0$. The the $\pi^0$ decays into 2 photons while the $\pi^+$ stops quickly in the TPC. The ability to distinguish the topology of this decay from the most frequent one, i.e. $K^+\rightarrow\mu^+\nu$, remarks the versatility of the LArTPC technology.
%%%%%%%%%%%%%%%%%%%%%%%%%%%%%%%%%%%%%%%%%%%%%%%%%%%%%%%




 \subsubsection{Kaon Interaction Cross Section for thin target in Geant4}
Since the kaon cross section in argon has never been measured before, simulation packages tune kaon transportation in argon by extrapolation from lighter and heavier nuclei. LArIAT uses the Geant4 suite for particle transportation.  Since kaon data on carbon are available, we used it as a metric to evaluate the Geant4 prediction performances.  Figure \ref{fig:TrueCarbon} shows the total hadronic cross section for carbon implemented in Geant4 10.01.p3 overlaid with the Bugg and Friedman data. Unfortunately, the current version of Geant4 does not reproduce the data for carbon closely. On one hand, this evidence makes us even more wary when using the Monte Carlo in simulating the kaon-argon interactions. On the other, it further highlights the importance of the kaon measurement.



\begin{figure}
\captionsetup{justification=raggedright}  
  \centering  
\includegraphics[width=3in]{Chapter-1/Images/CarbonG4.png}
\caption{total hadronic cross section for carbon implemented in Geant4  10.01.p3  with overlaid with the Bugg and Frideman data.}
\label{fig:TrueCarbon}
\end{figure}







\chapter{Liquid Argon Detectors at the Intensity Frontier}\label{ch:2}


In the next few years, LArTPC experiments -- such as the Short-Baseline Neutrino program (SBN) and DUNE -- will be major players in the intensity frontier field. 


\section{Liquid Argon Time Projection Chambers at the Intensity Frontier}

%Bubble-chamber experiments played a key role in probing the properties of ?-interactions. The Liquid Argon Time Projection Chambers (LArTPC) technology,  first proposed by C.Rubbia in 1977 with ICARUS project [14], is considered the modern evolution of bubble-camber concept, with the additional features of three-dimensional event reconstruction, high-resolution calorimetry, active mass coincident with detector sensitive mass and can intrinsically supply a trigger signal (self-triggering) by means of the scintillation light produced in the liquid noble gas. This technology is ideal to perform $\nu$-studies in a broad energy range, from MeV up to few GeV, with high event reconstruction efficiency, thanks to the capability of particle identifcation and detailed reconstruction of different interaction topologies. In Figure 1.4 is shown a neutrino interaction event, producing a proton, a pion and a muon, as seen in a bubble chamber and in a LArTPC.


\subsection{Time Projection Chamber}
\subsection{Ionization Detectors with Noble Liquids}
\subsection{LArTPC: Principles of Operation}
\subsection{Liquid Argon Ionization Charge Detection}
\subsubsection{Electron Life Time \& purity}
\subsubsection{Space Charge Effect}
\subsubsection{Recombination Effect}
\subsection{Liquid Argon scintillation Light Detection}
\subsubsection{LAr Scintillation Process}
\subsubsection{Wavelength Shifting of LAr Scintillation Light}
\section{The SBN Program: Neutrino Interaction and Detection}
%\subsection{SBN Goals}
%\subsection{Neutrino Interactions and Detection }
\section{DUNE: Rare Decay Searches}
The key elements for a rare decay experiment are: massive active volume, long exposure, high identification efficiency and low background. 
%The limit to proton lifetime in case of absence of signal and backgrounds is set by calculating
%$$\tau/B > M\times \epsilon\times T \times 10^{32},$$ 
%where M is the detector mass in kton, $\epsilon$ the signal detection efficiency after cuts to suppress backgrounds (dependent on the considered decay mode), T is the exposure in years, B the assumed branching fraction for the considered mode and  $10^{32}$ is a factor accounting for the number of nucleons in a kton of material \cite{Bueno2007}.
Figure \ref{fig:PDKExperimentalLImit} shows the current best experimental limits on nucleon decay lifetime over branching ratio (dots). Historically, the dominant technology used in these searches has been water Cherenkov detectors: all the best experimental limits on every decay mode are indeed set by Super-Kamiokande \cite{PhysRevD.90.072005,PhysRevLett.115.121803}.  It is particularly important to notice that the kaon energy for the proton decay mode $p \rightarrow K^+ \bar{\nu}$ is under Cherenkov threshold.  Super-Kamiokande set the limit on the lifetime for the $p \rightarrow K^+ \bar{\nu}$ mode by  relying exclusively on photons from nuclear de-excitation. For this reason, an attractive alternative approach to identifying nucleon decay is the use of a Liquid Argon Time Projection Chamber (LArTPC). 

LArTPCs can complement nucleon decay searches in modes where water Cherenkov detectors are less sensitive, especially $p\rightarrow K^+\bar{\nu}$. According to \cite{Acciarri:Dune}, DUNE will have an active volume large enough, have sufficient shielding from the surface, and will run for lengths of time sufficient to compete with Hyper-K, opening up the opportunity for the discovery of nucleon decay. 

\begin{figure}[hbpt]
\centering
\includegraphics[width=6.5in]{Chapter-2/Images/PDKExperimentalLImit.png}
\caption{Proton decay lifetime limits from passed and future experiments.}
\label{fig:PDKExperimentalLImit}
\end{figure}


\begin{figure}[hbpt]
\centering
\includegraphics[width=3.5in]{Chapter-2/Images/pdkGenie.png}
\caption{Momentum of the kaon outgoing a proton decay event as simulated by the Genie 2.8.10 event generator in argon. The red line represent the kaon momentum distribution before undergoing the simulated final state interaction inside the argon nucleus, while the blue line represents the momentum distribution after FSI. }
\label{fig:PDKGENIE}
\end{figure}


%\subsection{Non-Accelerator Physics Program}
%\subsection{Rare Decay Searches: Experimental Limit}
%\subsection{Nucleon Decay Detection in LAr}
\section{Enabling the next generation of discoveries: LArIAT}
LArIAT, a small Liquid Argon Time Projection Chamber (LArTPC) in a test beam,  is designed to perform an extensive physics campaign centered on charged particle cross section measurements while characterizing the detector performance for future LArTPCs. LArTPC represents one of the most advanced experimental technologies for physics at the Intensity Frontier due to its full 3D-imaging, excellent particle identification and precise calorimetric energy reconstruction. This complex technology however needs a thorough calibration and dedicated measurements of some key quantities to achieve the precision required for the next generation of discoveries at the Intensity Frontier which LArIAT can provide. 

The LArIAT LArTPC is deployed in a dedicated calibration test beamline at Fermilab.
We use the LArIAT beamline to characterize the charge particles before they enter the TPC: the particle type and initial momentum is known from beamline information. The precise calorimetric energy reconstruction of the LArTPC technology enables the measurement of the total differential cross section for  tagged hadrons. 
The Pion-Nucleus and Kaon-Nucleus total hadronic interaction cross section have never been measured before in argon and they are a fundamental step to shed light on light meson interaction in nuclei. Additionally, these measures provides a key input to neutrino physics and proton decay studies in future LArTPC experiments like SBN and DUNE.
\textcolor{red}{add paragraph on all wonderful things lariat can do... some event displays would be nice!}



\textcolor{red}{ADD genie proton decay kaon distribution and lariat beamline overlaied}
The signature of a proton decay event in the ``LAr golden mode" is the presence of a single kaon of about 400 MeV in the detector. 

% This chapter to do:
% Re-write the red parts
% write DAQ part
% write collimators description
% put references
% re-read

\chapter{LArIAT: Liquid Argon In A Testbeam}\label{sec:experimentDescription}
In this chapter, we describe the LArIAT experimental setup. We start by illustrating the journey of the charge particles in the Fermilab accelerator complex, from the gaseous thermal hydrogen at the Fermilab ion source to the delivery of the LArIAT tertiary beam at MC7. We  then describe the LArIAT beamline detectors, the LArTPC, the DAQ and the monitoring system.

%\section{LArIAT \& the Intensity Frontier}
\section{The Particles Path to LArIAT}

LArIAT's particles history begins in the Fermilab accelerator complex with a beam of protons. The process of protons acceleration develops in gradual stages (see picture \ref{fig:Accelerator}): gaseous hydrogen is ionized in order to form H$^{-}$ ions; these ions are boosted to 750 keV by a Cockroft-Walton accelerator and injected to the Linac linear accelerator that increases their energy up to 400 MeV; then, H$^{-}$ ions pass through a carbon foil and lose the two electrons; the resulting protons are then injected into a rapid cycling synchrotron, called Booster; at this stage, protons reach 8 GeV of energy and are compacted into bunches; the next stage of acceleration is the Main Injector, a synchrotron which accelerates the bunches up to 120 GeV; in the Main Injector, several bunches are merged into one and used for the injection in the last stage.


The Fermilab accelerator complex works in supercycles of roughly 60 seconds in duration. The beam is split by electrostatic septa and delivered at different experimental halls all over the lab. A 120~GeV$/c$ primary proton beam with variable intensity is extracted in four-second ``spills" and sent to the Meson Center beam line. 

LArIAT's home at Fermilab is the Fermilab Test Beam Facility (FTBF), where the experiment characterizes a beam of charge particles downstream from the Meson Center beam line. 
Here, the primary beam is focused onto a tungsten target to create LArIAT's secondary beam. The composition of the secondary particle beam is mainly positive pions. The momentum peak of the secondary beam was fixed at 64~GeV/c for the LArIAT data considered in this work, although the beam is tunable in momentum between 8-80\,GeV/c; this configuration of the secondary beamline assured a stable beam delivery at the LArIAT experimental hall.
 
The secondary beam impinges then on a copper target within a steel collimator inside the LArIAT experimental hall (MC7) to create the LArIAT tertiary beam, (shown in  Fig.~\ref{fig:tert-layout}).   The steel collimator selects particles produced with a $13^\circ$ production angle at the target down the beamline.  The particles are then bent by  $~10^\circ$  through a pair of dipole magnets.  By configuring the field intensity of the magnets we allow the particles of LArIAT's tertiary beam to span a momentum range from 0.2 to 1.4~GeV/c. The polarity of the magnet is also configurable and determines the sign of the beamline particles which are focused on the LArTPC. If the magnets polarity is positive the tertiary beam composition counts mostly pions and protons with a small fraction of electrons, muons, and kaons. It is the job of the LArIAT beamline detectors to select the particles polarity,  to perform particle identification (beamPID) and to measure the momentum of the tertiary beam particles before they get to the LArTPC. The LArIAT detectors are described in the following paragraphs.  



%\begin{comment}     
\begin{figure}
  \centering  	
\includegraphics[width=\textwidth,height=\textheight,keepaspectratio]{Chapter-3/Images/AcceleratorFNAL.png}
\caption{Layout of Fermilab Acellerator complex.}
\label{fig:Accelerator}
\end{figure}

%\begin{comment}     
\begin{figure}
  \centering  	
\includegraphics[width=\textwidth,height=\textheight,keepaspectratio]{Chapter-3/Images/Tertiary.png}
\caption{Bird's eye view of the LArIAT tertiary beamline. In grey: upstream and downstream collimators; in yellow: bending magnets; in red: wire chambers; in blue: time of flight; in green: liquid argon TPC volume; in maroon: muon range statck.}
\label{fig:tert-layout}
\end{figure}


%%%%%%%%%%%%%%%%%%%%%%%%%%%%%%%%%%%%%%%%%%%%%%%%%%%%%%%%%%%%
\section{LArIAT Tertiary Beam Instrumentation}\label{sec:Instrumentation}

%%%%%%%%%%%%%%%%%%%%%%%%%%%%%%%%%%%%%%%%%%%%%%%%%%%%%%%%%%%%
The instrumentation of  LArIAT tertiary beam and the TPC components have changed several times during the three years of LArIAT data taking. The following paragraphs describe the components operational during ``Run II", the data taking period relevant to the hadron cross section measurements.

The key components of the tertiary beamline instrumentation for the hadron cross section analyses are the two bending magnets, a set of four wire chambers (WCs) and two time-of-flight scintillating paddles (TOF) and, of course, the LArTPC.  The magnets determine the polarity of the particles in the tertiary beam; the combination of magnets and wire chambers determines the particles' momentum, which is used to determine the particle species in conjunction with the TOF.
A muon range stack downstream from the TPC and two sets of cosmic paddles configured as a telescope surrounding the TPC are also used for calibration purposes.


\subsection{Bending Magnets}\label{sec:Magnets}
%%%%%%%%%%%%%%%%%%%%%%%%%%%%%%%%%%%%%%%%%%%%%%%%%%%%%%%%%%%

LArIAT uses a pair of identical Fermilab type ``NDB" electromagnets, recycled from the Tevatron's anti-proton ring, in a similar configuration used for the  MINERvA T-977 test beam calibration~\cite{MinervaTestbeam}). 
The magnets are a fundamental piece of the LArIAT beamline equipment, as they are used for both particle identification and momentum measurement before the LArTPC. The sign of the current in the magnets allows us to select either positively or negatively charged particles; the value of the magnetic field is used in the momentum determination and in the subsequent particle identification. 

We describe here the characteristics and response of one magnet, as the second one has a similar response, given its identical shape and history. Each magnet is a box with a rectangular aperture gap in the center to allow for the particle passage.  The magnet aperture measures 14.224~cm in height, 31.75~cm in width, and  46.67~cm in length.  Since the wire chambers aperture ($\sim$12.8~cm$^2$) is smaller than the magnet aperture, only the central part of the magnet gap is utilized. The field is extremely uniform over this limited aperture and was measured with two hall probes, both calibrated with nuclear magnetic resonance probes. The probes measured the excitation curve shown in Figure~\ref{fig:magnet_excitation}. 

\begin{figure}[!h]
\begin{centering}
\vspace{-0.3cm}
\includegraphics[height=3.0in]{Chapter-3/Images/ExcitationCurves.png}
\caption{
{ Magnetic field over current as a function of the current, for one NDB magnet (excitation curve). The data was collected using two Hall probes (blue and green). We fit the readings with a cubic function (black) to average of measurements (red) given in the legend.}
}
\label{fig:magnet_excitation}
\end{centering}
\end{figure}

The current through the magnets at a given time is identical in both magnets. For the Run II data taking period, the current settings explored were 60A (B $\sim$0.21 T) and 100A (B $\sim$0.35 T) in both polarities. 
Albeit advantageous to enrich the tertiary beam composition with high mass particles such as kaons, we never pushed the magnets current over 100 A, not to incur in overheating.  During operation, we operated a air and water cooling system on the magnets and we remotely monitored the magnets temperature.
 
\subsection{Multi-Wire Proportional Chambers}\label{sec:MWPC}
%%%%%%%%%%%%%%%%%%%%%%%%%%%%%%%%%%%%%%%%%%%%%%%%%%%%%%%%%%%%
\begin{figure}[!h]
\begin{centering}
\vspace{-0.3cm}
\includegraphics[height=2.3in]{Chapter-3/Images/WireChamber.png}
\caption{
{One of the four Multi Wire Proportional Chambers (WC) used in the LArIAT tertiary beamline.}
}
\label{fig:wirechamber}
\end{centering}
\end{figure}

LArIAT uses four multi-wire proportional chambers, or wire chambers (WC) for short, two upstream and two downstream from the bending magnets. The geometry of one chamber is shown in Figure~\ref{fig:wirechamber}: the WC effective aperture is a square of  12.8~cm perpendicular to the beam direction.  Inside the chamber, the 128 horizontal and 128 vertical wires hang at a distance of 1~mm from each other in a mixture of 85\% Argon and 15\% isobutane gas.  The WC operating voltage is between 2400~V and 2500~V. The LArIAT wire chambers are an upgraded version of the Fenker Chambers~\cite{Fenker}, where an extra grounding improves the signal to noise ratio of the electronic readout.  

Two ASDQ chips~\cite{ASDQchip} mounted on a mother board plugged into the chamber serve as front end amplifier/discriminator. The chips are connected to a multi-hit TDC~\cite{Sten} which provides a fast OR output used as first level trigger. The TDC time resolution is 1.18~ns/bin and can accept 2 edges per 9~ns.  
The maximum event rate acceptable by the chamber system is of 1 MHz: this rate is not a limiting factor considering that \textcolor{red}{the rate of the tertiary particle beam at the first wire chamber is estimated to be less than 15 kHz}. A full spill of data occurring once per supercycle is stored on the TDC board memory at once and read out by a specially designed controller.  We use LVDS cables to carry both power and data between the controller and the TDCs and from the controller to the rest of the DAQ.  
%It is possible to program the time window for acceptance for hits, time offsets, front end threshold, and pulse shaping parameters through the controller via a USB from a PC or through an Ethernet connection.

\subsubsection{Multi-Wire Proportional Chambers functionality}\label{sec:MWPCfunc}
We use the wire chamber system together with the bending magnets to measure the particle's momentum.

In the simplest scenario, only one hit on each and every of the four wire chambers is recorded during a single readout of the detector systems.  Thus, we use the hit positions in the two wire chambers upstream of the magnets to form a trajectory before the bend, and the hit positions in the two wire chambers downstream of the magnets to form a trajectory after the bend. We use the angles in the XZ plane between the upstream and downstream trajectories  to calculate the $Z$ component of the momentum as follows:

\begin{equation}
P_z=\frac{B_{eff}L_{eff}}{3.3(sin(\theta_{DS})-sin(\theta_{US}))},
\label{eq:momformula}
\end{equation}

where $B_{eff}$ is the effective maximum field in a square field approximation,  $L_{eff}$ is the effective length of both magnets (twice the effective length of one magnet), $\theta_{US}$ is the angle off the $z$ axis of the upstream trajectory, $\theta_{DS}$ is the angle off the $z$ axis of the downstream trajectory  and  3.3~$c^{-1}$ is the conversion factor from [T$\cdot$m] to [MeV/c]. By using the hit positions on the third and fourth wire chamber, we estimate the azimutal and polar angles of the particle trajectory, and we are able to calculate the other components of the momentum. 

The presence of multiple hits in a single wire chamber or the absence of hits in one (or more) wire chambers can complicate this simple scenario. The first complication is due to beam pile up, while the latter is due to wire chamber inefficiency. In the case of multiple hits on a single WC, at most one wire chamber track is reconstructed per event. Since the magnets bend particles only in the X direction, we assume the particle trajectory to be roughly constant in the YZ plane, thus we keep the combination of hits which fit best with a straight line. 
It is still possible to reconstruct the particle's momentum  even if the information is missing in either of the two middle wire chambers (WC2 or WC3), by constraining the particle trajectory to cross the plane in between the magnets. 
%Under the assumption of identical magnets, we define a plane centered in the middle of the two magnets (called ``midplane") that the physical particles need to cross. We project the completed half of the wire chamber track to the midplane, assuming no bending, to find the point of intersection. We use this point to complete the other half of the wire chamber track and to calculate the reconstructed momentum  with these four points. To account for the lack of bending in our reconstructed wire chamber track, we apply to the calculated momentum  a correction obtained with a sample of 4-point, single hit tracks.

Events satisfying the simplest scenario of one single hit in each of the four wire chambers form the ``Picky Track" sample.  We construct another, higher statistics sample, where we loosen the requirements on single hit and wire chamber efficiency: the ``High Yield" sample. For LArIAT Run II, the High Yield sample is about three times the Picky Tracks statistics.  For the first measurements of the LArIAT hadronic cross section, we use the Picky Tracks sample because the uncertainty on the momentum is smaller and the comparison with the beamline MC results is straightforward compared with the High Yield sample;  a possible future update and cross check of these analysis would be the use of the High Yield sample. 

%We use Picky Tracks to calibrate the momentum measurement for the High Yield sample, in particular to obtain a momentum correction for  tracks missing information from the central WCs;  this correction adds an uncertainty of approximately 2\% to the momentum calculation of the High Yield sample.

\textcolor{red}{Four point track momentum uncertainty}

\subsection{Time-of-Flight System}\label{sec:TOF}
%%%%%%%%%%%%%%%%%%%%%%%%%%%%%%%%%%%%%%%%%%%%%%%%%%%%%%%%%%%%
Two scintillator paddles, one upstream to the first set of WCs and one downstream to the second set of WCs  form LArIAT  time-of-flight (TOF) detector system. 

The upstream paddle is made of a 10 x 6 x 1~cm scintillator piece, read out by two PMTs mounted on the beam left side which collect the light from light guides mounted on all four edges of the scintillator. The downstream paddle is a   14 x 14 x 1~cm scintillator piece read out by two PMTs on the opposite ends of the scintillator.
The relatively thin width on the beamline direction minimizes energy loss of the particles coming from the target in the scintillator material.

%\begin{figure}[!h]
%\begin{centering}
%\vspace{-0.3cm}
%\includegraphics[height=2.3in]{Chapter-3/Images/tofdelay.png}
%\caption{
%{\scriptsize \sf Pictures of the TOF system as was deployed during Run-I and Run-II data taking. The left image is of the upstream TOF paddle and the right image is of the downstream TOF paddle }
%}
%\label{fig:TOFSystemRunIandII}
%\end{centering}
%\end{figure}



The CAEN 1751 digitizer is used to digitize the TOF PMTs signals at a sampling rate of 1 GHz. The 12 bit samples are stored in a circular memory buffer. At trigger time, data from the TOF PMTs are recorded to output in a 28.7 \textmu s windows starting  approximately 8.4 \textmu s before the trigger time. 



\subsubsection{TOF functionality}\label{sec:MWPCfunc}


The TOF signals rise time (10-90\%) is 4 ns and a full width, half-maximum of 9 ns consistent in time. The signal amplitudes from the upstream TOF and  downstream TOF are slightly different:  200 mV for the upstream PMTs but only 50 mV for downstream PMTs. The time of the pulses was calculated utilizing an oversampled template derived from the data itself. We take the pulse pedestal from samples far from the pulse and subtract it to the pulse amplitude. We then stretch vertically a template to match the pedestal-subtracted pulse amplitude and we move it horizontally to find the time. With this technique, we find a pulse time-pickoff resolution better than 100 ps.  The pulse pile up is not a significant problem given the TOF timing resolution and the rate of the particle beam.  Leveraging on the pulses width uniformity of any given PMT (sigma of 400 ps),  we flag events where two pulses overlap as closely in time as 4 ns with an 90\% efficiency according to simulation. 


We combine the pulses from the two PMTs on each paddle to determine the particles' arrival time by averaging the time measured from the single PMT, so to minimize errors due to optical path differences in the scintillator.  However, a time spread of approximately 300~ps is present in both the upstream and downstream detectors, likely due to transit time jitter in the PMTs themselves.  There is no evidence of systematic timing drift over long data-taking periods such as 3-4 months: the maximum variation of the average time differences between pairs of PMTs reading out the same scintillator is of the order of 150~ps.

\textcolor{red}{calculated TOF with error}


%%%%%%%%%%%%%%%%%%%%%%%%%%%%%%%%%%%%%%%%%%%%%%%%%%%%%%%%%%%%
\subsection{Punch-Through and Muon Range Stack Instruments}\label{sec:MuRS}
%%%%%%%%%%%%%%%%%%%%%%%%%%%%%%%%%%%%%%%%%%%%%%%%%%%%%%%%%%%%

The punch-thorough and the muon range stack (MuRS) detectors are located downstream of the TPC. These detectors provide a sample of  TPC crossing tracks without relying on TPC information and can be used to improve particle ID for  muons and pions with momentum higher than  450 MeV/c.

The punch-thorough is simply a \textcolor{red}{? x ? x ?}~cm scintillator piece, read out by \textcolor{red}{ two? Hamamatsu? 2? inch} PMTs. 
The MuRS is a segmented block of steel with four slots instrumented with scintillation bars. The four steel layers in front of each instrumented slot are 2 cm, 2 cm, 14 cm and 16 cm wide in the beam direction. Each instrumented slot is equipped with four scintillation bars each, positioned vertically in the direction orthogonal to the beam. Each scintillator bar measures  \textcolor{red}{? x ? x 2}~cm and it is read out by  \textcolor{red}{ two? Hamamatsu? 2? inch} PMTs.  

The signals from both the punch-thorough and the MuRS PMTs are digitized in the CAEN V1740, same as the TPC; the details of this discriminator are laid out in~\ref{sec:TPCCharge}. It is worth noticing that the sampling time of the CAEN V1740 is slow (of the order of 128 ns), so pulse shape information from the PMT is lost.
Punch-thorough and MuRS hits are formed utilizing the OR between the PMTs digital discriminator signals under threshold at a given time, where we obtain the threshold for each PMT directly on data distributions.



%%%%%%%%%%%%%%%%%%%%%%%%%%%%%%%%%%%%%%%%%%%%%%%%%%%%%%%%%%%%
\subsection{LArIAT Cosmic Ray Paddle Detectors}\label{sec:CosmicRayPaddle}
%%%%%%%%%%%%%%%%%%%%%%%%%%%%%%%%%%%%%%%%%%%%%%%%%%%%%%%%%%%%
Besides on beam data, LArIAT also triggers on cosmic rays events by using two sets of cosmic ray paddle detectors (a.k.a. ``cosmic towers".) The cosmic towers frame the LArIAT cryostat, as one sits in the downstream left corner and the other sits in the upstream right corner of the cryostat. Two paddle sets of four scintillators pieces each, an upper and a lower set, make up each cosmic tower. 
Of the four paddles, a couple of two matched paddles stands upright while the a second matched pair lies across the top of the assembly in the top sets (or across the bottom of the assembly in the bottom sets). The horizontal couple is used as a veto for particles traveling from the TPC out.  The four signals  from the vertical paddles along one of the body diagonals of the TPC are combined in a logical ``AND''. This allows to select cosmic muons crossing the TPC along one of its diagonals.  Cosmic ray tracks crossing both anode and cathode populate the events triggered this way. This particularly useful sample of tracks (which we can safely assume to be associated with ~5 GeV muons MIPs) can be used for many tasks; for example, we use anode-cathode piercing tracks to cross check the TPC electric field on data (see \ref{ch:AppendixC}), to calibrate the charge response of the TPC wires for the full TPC volume and to measure the electron lifetime in the chamber \textcolor{red}{ ADD reference to different chapters}.

%%%%%%%%%%%%%%%%%%%%%%%%%%%%%%%%%%%%
All the paddles are $3.02~cm$ thick and are  trapezoidal in shape. The paddles come in two sizes: the smaller version has bases $32.2~cm$ and $26.7~cm$, and $61.0~cm$ height, while the bigger version has bases $33.2~cm$ and $27.0~cm$, and $70.8~cm$ height.  A Zener-diode Hamamatsu H5783 PMT collects the light from a wavelength-shifting optical fiber which runs along one of the long sides of each paddle.
A custom-made PMT Amplifier and Discrimination (PAD) circuit mounted at one end of the paddle collects signals from the PMTs and sends them to the Control and Concentrator Unit (CCU). We use the same connection to  power the PMT, control voltage and threshold, and output the PMT signal as logic ECL pulse.
We retrieved the scintillation paddles from the decommissioning of the CDF detector at Fermilab and we used only the paddles with a counting efficiency greater than 95\% and low noise at working voltage. The measured trigger rate of the whole system is $0.032Hz$, corresponding to $\sim 2$ muons per minute.


\begin{figure}[h!]
\centering
 \includegraphics[angle=90,width=0.7\textwidth]{Chapter-3/Images/Cosmic_Paddle.jpg}
\caption{Photograph of one of the scintillation counters used in the cosmic towers. } 
\label{pic:cosmicpaddle}
\end{figure}





\section{In the Cryostat}
\subsection{Cryogenics and Argon Purity}\label{ch:Cryo}
LArIAT repurposed the ArgoNeuT cryostat \cite{argoneut} in order to use it in a beam of charge particles. We also added a new process piping and a new liquid argon filtration system in FTBF.
Inside the LArIAT experimental hall, the cryostat sits on the beam of charge particles with its horizontal main axis oriented parallel to the beam.

Two volumes make up LArIAT cryostat, shown in Figure \ref{fig:LArIATCryoStat}: purified liquid argon fills the inner vessel, while the outer volume insulates it with a vacuum jacket equipped with layers of aluminized mylar superinsulation. The inner vessel is a cylinder of 130~cm length and 6.2~cm diameter, containing about 550~L of LAr, corresponding to a mass of 0.76 ton. We run the signal cables for the LArTPC and the high voltage feedthrough through a ``chimney'' at the top and mid-length of the cryostat.


\begin{figure}[htb]
\centering
\includegraphics[scale=0.18]{Chapter-3/Images/Cryostat1.jpg}
\includegraphics[scale=0.07]{Chapter-3/Images/Cryostat2.jpg}
\caption{\emph{(left)} The LArIAT cryostat open with the TPC placed in the inner volume. \emph{(right)} The LArIAT cryostat fully sealed during initial commissioning prior to installation at Fermilab Testbeam Facility.Access to the internal volume is possible by opening the upstream end caps of the inner and outer vessels. }
\label{fig:LArIATCryoStat}
\end{figure}

Given the different scopes of the ArgoNeuT and LArIAT detectors, we made several modification to the ArgoNeuT cryostat in order to use it in LArIAT. In particular, the modification  shown in Figure \ref{fig:LArIATCryoMods} were necessary to account for the beam of charged particles entering the TPC and to employ the new FTBT liquid argon purification system. 
We added a ``beam window'' on the front outer end cap and an ``excluder'' on the inner endcap, with the scope of minimizing the amount of dead material upstream of the TPC's active volume.  Doing so, we reduced the amount of uninstrumented material before the TPC from $\sim$ 1.6 radiation lengths ($X_{0}$) (ArgoNeuT) to less than 0.3 $X_{0}$ (LArIAT). To allow studies of the scintillation light, we added a side port feedthrough which enables the mounting of the light collection system, as well as the connections for the corresponding signal and high-voltage cables (see Section \ref{sec:TPCLight}).  We modified the bottom of the cryostat adding Conflat and ISO flange sealing to connect the liquid argon transfer line to the new argon cooling and purification system.


\begin{figure}[htb]
\centering
\includegraphics[scale=0.35]{Chapter-3/Images/CryoMods.png}
\caption{Pictures of the modified components of the cryostat. $1)$ The addition of an outlet to the bottom of the cryostat to allow connections to the purification system; $2)$ The ``beam-window'' on the outer endcap and the concave inner surface of the inner endcap (referred to as the excluder) to reduce the amount of material through which beam particles must travel before entering the TPC; $3)$ The modified side port for the LArIAT light collection system.}
\label{fig:LArIATCryoMods}
\end{figure}

%%%%%%%%%%%%%%%%%%%%%
As in any other LArTPC, argon purity is a crucial parameter for LArIAT. Indeed, the presence of contaminants effects both the basic working principles of a LArTPC: electronegative contaminants such as oxygen and water decrease the number of ionization electrons collected on the wires after drifting through the volume, while contaminants such as Nitrogen decrease the light yield from scintillation light, especially in its slow component.
In LArIAT, contaminations should not exceed the level of 100 parts per trillion (ppt). We achieve this level of purity in several stages. The specifics required for the commercial argon bought for LArIAT are 2 parts per million (ppm) oxygen, 3.5~ppm water, and 10~ppm nitrogen. This argon is monitored with the use of commercial gas analyzer.
Argon is stored in a dewar external to LArIAT hall and filtered before filling the TPC. %The argon is delivered from the commercial dewar to the cryostat through 2.54~cm diameter schedule 10 stainless steel piping.  The piping was insulated with 20.32~cm of polyurethane foam by the manufacturer.  The piping was cleaned to remove oil and grease before being welded into the system. 
LArIAT uses a filtration system designed for the Liquid Argon Purity Demonstrator (LAPD)~\cite{LAPD}: half of a 77~liter filter contains a 4A molecular sieve (Sigma-Aldrich~\cite{sigma-aldrich}) apt to remove mainly water, while the other half contains BASF~CU-0226~S, a highly dispersed copper oxide impregnated on a high surface area alumina, apt to remove mainly oxygen~\cite{basf}. A single pass of argon in the filter is sufficient to achieve the necessary purity, unless the filter is saturated. In case the filter saturates, the media needs to be regenerated by using heated gas; this happened twice during the Run II period\footnote{We deemed the filter regeneration necessary every time the electron lifetime dropped under 100 \textmu s}.
The filtered argon reaches the inner vessel via a liquid feedthrough on the top of the cryostat. Argon is not recirculated in the system, but it rather boils off and vent to the atmosphere. During data taking, we replenish the argon in the cryostat several times per day to keep the TPC high voltage feedthrough and cold electronics always submerged. In fact, we need to monitor the level, temperatures, and pressures  of the argon both in the commercial dewar and inside the cryostat. 
\subsection{LArTPC: Charge Collection}\label{sec:TPCCharge}
The LArIAT Liquid Argon Time Projection Chamber is a rectangular box of dimensions 47 cm (width) x 40 cm (height) x 90 cm (length), containing 170 liters of Liquid Argon.
The LArTPC three major subcomponents are 
\begin{itemize} 
\item[1)] the cathode and field cage,
\item [2)] the wire planes, 
\item [3)] the read-out electronics. %
\end{itemize}



\subsubsection{Cathode and field cage}
A G10 plain sheet with copper metallization on one of the 40x90~cm inner surfaces forms the cathode. 
A high-voltage feedthrough on the top of the LArIAT cryostat delivers the high voltage to the cathode; scope of the high voltage system (Figure~\ref{fig:HVScheme}) is to drift ionization electrons from the interaction of charged particles in the liquid argon to the wire planes.  The power supply used in this system is a Glassman LX125N16 ~\cite{GlassmanPS} capable of generating up to -125~kV and 16~mA of current, but operated at -23.5kV during LArIAT Run-II. The power supply is connected via high voltage cables to a series of filter pots before finally reaching the cathode. 

\begin{figure}[htb]
\centering
\includegraphics[scale=0.35]{Chapter-3/Images//HVSchematic.png}\\
\caption{Schematic of the LArIAT high voltage system.}
\label{fig:HVScheme}
\end{figure}%See DocDB 1472



The field cage is made of \textcolor{red}{13 parallel copper rings} framing the inner walls of the G10 TPC structure. A network of voltage-dividing resistors connected to the field cage rings steps down the high voltage from the cathode to form a uniform electric field. The electric field over the entire TPC drift volume is  486 V/cm (see \ref{ch:AppendixA}). The  maximum drift length, i.e. the distance between cathode and anode planes, is 47 cm.

\subsubsection{Wire planes}
The wire planes measure the charge deposited in the TPC active volume. The drifting charge induces a current on the wire of the inner planes and it is collected on the collection plane wires.
LArIAT counts three wire planes separated by 4 mm spaces: in order of distance from the cathode, they are the shield, the induction and the collection plane. The distance between two consecutive wires in each given plane (aka wire pitch) is 4 mm.  The shield plane counts 225 parallel wires of equal length oriented vertically. This plane is not connected with the read-out electronics; rather it shields the outer planes from extremely long induction signals due to the ionization chamber in the whole drift volume. As the shield plane acts almost like a Faraday cage, the shape of signals in the first instrumented plane (induction)  results easier to reconstruct.  Both the induction and collection planes count 240 parallel wires of different length oriented at 60$^\circ$ from the vertical with opposite signs.
Moving electrons moving past the induction plane will induce a bipolar pulse on its wires and will form a unipolar pulse when collected on the last plane wires. 

The three wire planes and the cathode form three drift volumes, as shown in Figure \ref{fig:driftregions}. 
The main drift volume is defined as the region between the cathode plane and the shield plane (C-S). The other two drift regions are those between the shield plane and the induction plane (S-I), and between the induction plane and the collection plane (I-C). The electric field in these regions is chosen to satisfy the charge transparency condition to allow for 100$\%$ transmission of the drifting electrons through the shield and then the induction planes. 

\begin{figure}[htb]
\centering
\includegraphics[scale=0.35]{Chapter-3/Images/DriftRegions.png}\\
\caption{Schematic of the three drift regions inside the LArIAT TPC: the main drift volume between the cathode and the shield plane (C-S) in green, the region between the shield plane and the induction plane (S-I) in purple, and the region between the induction plane and the collection plane (I-C) in pink.}
\label{fig:driftregions}
\end{figure}

Table \ref{tab:voltages} provides the default voltages applied to the cathode and the shield, induction, and collection plane.  

\begin{table}[htpb]
\centering
\caption{Cathode and anode planes default voltages}
\label{tab:voltages}
\begin{tabular}{llll}
\hline
\multicolumn{1}{|l|}{ Cathode} & 
\multicolumn{1}{|l|}{ Shield} & \multicolumn{1}{l|}{ Induction} & \multicolumn{1}{l|}{ Collection}  \\ \hline
\multicolumn{1}{|l|}{-23.17 kV} &
\multicolumn{1}{|l|}{-298.8 V} & \multicolumn{1}{l|}{-18.5 V}      & \multicolumn{1}{l|}{338.5 V} \\ \hline
\end{tabular}
\end{table}


\subsubsection{Electronics}

Dedicated electronics read the induction and collection plane wires, for a total of  480-channel analog signal path from the TPC wires to the signal digitizers. A digital control system for the TPC-mounted electronics, a power supply, and a distribution system complete the front-end system. Figure \ref{pic:FEelectronics} shows a block diagram of the overall system. The direct readout of the ionization electrons in liquid argon forms typically small signals on the wires, which needs to be amplified in oder to be processed. LArIAT  performs the amplification stage directly in cold with  amplifiers developed by Brookhaven National Lab (BNL) and mounted on the TPC frame inside the liquid argon, achieving a remarkable Signal-to-Noise ratio. %The BNL ASICs adopted in LAriAT are designated as LArASIC, version 4-star.%The ASIC signals for each wire are then driven out of the vessel  to DAQ boards that act as waveform recorders.
The signal from the ASICs are driven to the other end of the readout chain, to the CAEN V1740 digitizers. The CAEN V1740 has a 12 bit resolution and a maximum input range of 2~VDC, resulting in about 180 ADC count for a crossing MIP.   

\begin{figure}[htbp]
 \centering
 \includegraphics[width=1.0\textwidth]{Chapter-3/Images/LArIAT_FE_Electronics.png}
\caption{Overview of LArIAT Front End electronics. } 
\label{pic:FEelectronics}
\end{figure}




\subsection{LArTPC: Light Collection System}\label{sec:TPCLight}
The mechanism of particle detection in argon other than drift electrons is the collection of scintillation photons.  Over the course of LArIAT three years of data taking, the light collection system changed several times. We describe here the light collection system for Run II. Two PMTs, a 3-inch diameter Hamamatsu R-11065 and 2-inch diameter ETL D757KFL~\cite{lightsys-pmttests}, as well as three SiPMs arrays (two Hamamatsu S11828-3344M 4x4 arrays and one single-channel SensL MicroFB-60035 ) are mounted on the PEEK support structure. PEEK screws into an access flange as shown in Figure~\ref{lightsys_pmts}, on the anode side, leaving  approximately 5~cm of clearance from the collection plane.  

%------------------------------------------
\begin{figure}
\centering
\includegraphics[height=2.2in]{Chapter-3/Images/lightsys_pmts.png}
\hspace{1cm}
\includegraphics[height=2.2in]{Chapter-3/Images/lightsys_wls.png}
\caption{LArIAT's photodetector system for observing LAr scintillation light inside the TPC (left), and a simplified schematic of VUV light being wavelength-shifting along the TPB-coated reflecting foils (right).}
\label{lightsys_pmts}
\end{figure}
\begin{figure}
\centering
\includegraphics[height=0.25\textheight]{Chapter-3/Images/lightsys_etlbase.jpeg}
\hspace{0.5cm}
\includegraphics[height=0.25\textheight]{Chapter-3/Images/lightsys_hmmbase.jpg}
\caption{\label{voltagedividers}Photos of the voltage divider bases for the ETL PMT (left) and the Hamamatsu PMT (right) used in Run-II.  The cable connections to the bases seen here were used for powering and testing prior to installation.  The yellow through-hole signal coupling capacitors seen on both bases are 18~nF (X7R) and are rated to 2~kV.}
\end{figure}
%------------------------------------------
Liquid argon scintillates in vacuum-ultraviolet (VUV) range at 128 nm; since cryogenic PMTs are not sensitive to VUV wavelengths, we need to shift the light in a region visible to the PMTs. In LArIAT, the wavelength shifting is achieved by installing on the four walls of the TPC highly-reflective VIKUITY dielectric substrate foils coated with a thin layer of tetraphenyl-butadiene (TPB) (see Figure ~\ref{lightsys_foils}). One or more visible photons  are emitted and reflected into the chamber during the interaction of a VUV photon interacts with the TPB. Thus, the light yield is increased and made more uniform across the TPC active volume, allowing the possibility of light-based calorimetry (under study).

%------------------------------------------
%\begin{figure}
%\centering
%\includegraphics[scale=1.3]{Chapter-3/Images/lightsys_foils.png}
%\caption{\label{lightsys_foils}The TPB-coated reflector foils mounted to the TPC field cage walls as viewed through the front cryostat opening. {\textcolor{red}{Better pictures needed.}}}
%\end{figure}
%------------------------------------------

For Run II, we coated both  the windows of the ETL PMT and SensL SiPM  with a thin layer of TPB. In doing so, some of the VUV scintillation light converts into visible right at the sensor faces, keeping information on the direction of the light source. Information about the light directionality is lost for light reflected on foils, as the reflection is uniform in angle. For Run-II, the voltage dividers for the PMTs were configured for positive bias with a DC-coupled anode (AC-coupled anode with grounded photocathode) to minimize induced noise on the TPC wires and modify the PMT bases accordingly.  



\section{Trigger and DAQ}
The LArIAT DAQ and trigger system governs the read out of all the many subsystems forming LArIAT. 
The CAEN V1495 module and its user-programmable FPGA  are the core of this system.  Every 10~ns, this module checks for matches between sixteen logical inputs and user-defined patterns in the trigger menu; if it finds a match for two consecutive clock ticks, that trigger fires.

The beam instruments,  the cosmic ray taggers, and the light collection system provide NIM-standard logic pulse inputs to the trigger decision. We automatically log the trigger inputs configuration with the rest of the DAQ configuration at the beginning of each run.

Fundamental inputs to the trigger card come from the TOF (see Sec.~\ref{sec:TOF}) and the wire chambers (see Sec.~\ref{sec:MWPC}), as activity in these systems points to the presence of a charged particle in tertiary beam line.
In particular, the discriminated pulses from the TOF PMTs form a NIM logic pulse for the trigger logic. We ask for a coincidence within a 20~ns window for all the pulses from the PMTs looking at the same scintillator block and use the coincidence int the upstream and downstream paddle to inform the trigger decision. In order to form a coincidence between the upstream and downstream paddles, we delay the upstream paddle coincidence by 20~ns and widen it by 100~ns. The delay and widening are necessary to account for both  lightspeed particles and slower particles (high-mass) to travel the 6.5~m between the upstream and the downstream paddles. 
Four multi-hit TDCs read out each wire chamber: two TDC per plane (horizontal and vertical), sixty-four wires per TDC. In each TDC, we keep the logical ``OR" for any signal over threshold from the sixty-four wires. We then require a coincidence between the ``OR" for the horizontal TDCs and the ``OR" for the vertical TDCs: with this logic we make sure that at least one horizontal wire and one vertical wire saw significant signal in one wire chamber.  The single logical pulse from each of the four wire chambers feeds into the first four inputs to the V1495 trigger card. We require a coincidence within 20~ns of at least three logical inputs to form a trigger.


%Another primary input to the trigger card is from the cosmic towers (see Section~\ref{sec:CosmicRayPaddle}). To capture cosmic ray events in which a minimally ionizing cosmic ray muon crossed the TPC along the body diagonal, NIM modules form the logical coincidences from the two cosmic towers, one upper and one lower paddle assembly, in each combination.  The OR of these is provided as an input to the V1495. 

%Three important logic pulses are derived from the timing of the beam.  These include a pulse in a brief window before the beam, a pulse indicating that the beam is on, and a pulse which defines the beam-free period which may be used for collecting cosmic-ray events.  An adjustable pulser is a fourth trigger input which does not depend on any particular activity in the experiment hall,  useful for collecting background events with zero bias. 

%%%%%%%%%%%%%%%%%%%%%%

%The PMTs observing liquid argon scintillation light (see Section \ref{sec:PhotonSystem}) produce pulses which form the foundation of several interesting trigger inputs.  Thresholding a copy of each PMT pulse (after amplification), and requiring a coincidence of pulses within $\sim$20~ns, creates simple trigger inputs indicating ionizing radiation was produced in the TPC.  This scintillation logic pulse is used to initiate a gate which spans the length of the TPC drift time, creating a logic signal which is remains ``on'' while significant drift charge may still be present in the TPC.  In addition, requiring a delayed coincidence of two subsequent scintillation logic pulses, separated by a variable length of time ranging from 300~ns to 7~$\mu$s, is used to create a trigger input to select events where a cosmic muon stops and decays to a Michel electron in the TPC.  A few different versions of this light-based trigger were implemented throughout the course of LArIAT's run time to allow reconstruction and calorimetric studies of Michel electrons. Figure~\ref{michel_logic} shows a schematic diagram of the logic comprising the Michel electron trigger. 

%\begin{figure}
%\includegraphics[width=\textwidth]{figures/trigger_michellogic.png}
%\caption{\label{michel_logic}A schematic diagram of the trigger logic used to select Michel electron events during the cosmic readout window of the LArIAT supercycle.  The two PMT signals refer to the Hamamatsu (``HMM'') and ETL PMTs described in Section~\ref{sec:PhotonSystem}.  For some data-taking periods in Run-II, un-amplified pulses were discriminated at 180 mV to act as a veto on events that may saturate the dynamic range of the V1751 digitizer.  The discriminator thresholds used on the amplified (x10) PMT signal copies (\emph{ThA}, \emph{ThB}) as well as the Gate Delay period, were adjusted between run periods while experimenting with different configurations.}
%\end{figure}

%Further trigger inputs come from the beam line instrumentation behind the LArTPC cryostat, the PMTs of the Punch-Through scintillator paddles and those of the scintillator paddles instrumenting the Muon Range Stack.  The PMT pulses of all four of the broad-faced Punch Through paddles are discriminated to form logic pulses.  A single logic pulse is formed from these, indicating activity in at least two overlapping paddles at the rear of cryostat, before the steel block of the range stack.  PMT pulses from the Muon Range Stack are amplified and threshold discriminated.  These MuRS paddle pulses are then combined as in the Punch Through, creating single-bit indicators for each of the four instrumented layers that at least one pair of overlapping scintillator paddles sent signals within a 20~ns coincidence window.

%\subsubsection{Trigger Decision and Issuance}

%The V1495 may be configured to have up to sixteen trigger patterns and sixteen veto patterns, based on the trigger input signals.  A trigger pattern is defined as the AND of one or more defined inputs, and may include the NOT of the AND of further inputs.  Veto patterns are independently defined in the same way, but they have a very different effect.  When any of the trigger patterns fire, a ``fast trigger'' signal is issued and an adjustable countdown is initiated.  If the countdown completes without a veto pattern firing, the ``slow trigger'' signal is also issued and on a distinct hardware channel. Otherwise, if a veto pattern fires during the countdown, the slow trigger signal is vetoed.  

%The fast trigger signal prompts readout of all the `short' data buffers, which include the V1751 modules, the V1495 itself, and the MWPC controller.  The V1751 buffers typically contain digitized PMT signals from the time of flight and cryogenic light collection detectors. Readout of the TPC wire signals, which are much longer and more numerous, is only prompted at the issuance of the slow trigger.



\section{Control Systems}
LArIAT is a complex ensemble of systems which needed to be monitored at once during data taking.  We performed the monitoring of the systems operations with a slow control system, a DAQ monitoring system and a low level data quality monitoring described in the following sections.

\subsubsection{Slow Control}
We used the Synoptic Java Web Start framework as a real-time display of subsystem conditions. Its simple 
Graphical User Interface allowed us to change the operating parameters and to graph the trends of several variables of interest for all the tertiary beam detectors.  Among the most important quantities monitored by Synoptic there are the level of argon in both the inner vessel and the external dewar, the operating voltages of cathode and wire planes, of the PMTs and SiPMs, and of the four wire chambers, as well as the magnets temperature. Figure \ref{fig:synoptics} shows an example of the monitoring system.
LArIAT uses the Accelerator Control NETwork system (ACNET) to monitor the beam conditions of the MCenter beamline. For example, the X and Y position of the beam at the first two wire chambers (WC1 and WC2) are shown in \ref{fig:ACNET}. 

\begin{figure}[htb]
\centering
\includegraphics[width=\textwidth,height=\textheight,keepaspectratio]{Chapter-3/Images/BeamOverview.png}
\caption{Interface of the Synoptic slow control system}
\label{fig:synoptics}
\end{figure}

\begin{figure}[htb]
\centering
\includegraphics[scale=0.5]{Chapter-3/Images/BeamPosition.png}
\caption{Beam position at the upstream wire chambers monitored with ACNET.}
\label{fig:ACNET}
\end{figure}

\subsubsection{DAQ Monitoring}

We monitor the data taking and the run time evolution with the Run Status Webpage (\href{http://lariat-wbm.fnal.gov/lariat/run.html}{http://lariat-wbm.fnal.gov/lariat/run.html}), a  webpage updated in real-time.  The page displays, among other information, the total number of triggers in the event per CAEN digitizer board, the total number of detectors triggered during a beam spill,  the trigger patterns (the number of times a particular trigger pattern was satisfied  during a beam spill) and current time relative to the Fermilab accelerator complex supercycle. A screen shot of the page is show in figure \ref{fig:runcond}.

\begin{figure}[htb]
\centering
\includegraphics[scale=0.6]{Chapter-3/Images/RunConditions.png}
\caption{Run Status page at LArIAT downtime. At the top the yellow bar displays the current position in the Fermilab supercycle. Interesting information to be monitored by the shifter were the run number and number of spills, time elapsed from data taking (here in red), the energy of the secondary beam and the trigger paths.}
\label{fig:runcond}
\end{figure}



\subsubsection{Data Quality Monitoring}
We employ two systems to ensure the quality of out data during data taking: the Near-real-time Data Quality Monitoring and the Event Viewer.

\href{http://lariat-daq01.fnal.gov:5000/}{The Near-real-time Data Quality Monitoring} (DQM). This webpage receives updates from all the VME boards in the trigger system and displays the results of a quick analysis of the DAQ stream of raw data on a spill-by-spill basis. The DQM allows the shifter to monitor almost in real time (typically with a 2-minute delay)  a series of low level-quantities and compare them to past collections of beam spills. Some of the variables monitored in the DQM are  the pedestal mean and RMS on CAEN digitizer boards
of the TPC wires and PMTs of the beamline detectors, the hit occupancy and timing plots on the multi-wire chambers, and number of data fragments recorded that are used to build a TPC event. Abnormal values for  low-level quantity in the data  activate a series of alarms in the DQM; this quick feedback on the DAQ and beam conditions is fundamental to assure a fast debugging of the detector and a very efficient data taking during beam uptime.

The online Event Viewer displays a two dimensional representation of LArIAT TPC events on both the Induction and the Collection planes in near real time. The raw pulses collected by the DAQ on each wire are plotted as a function of drift time, resulting in an image of the TPC event easily readable by the shifter. This tool guarantees a particularly good  check of the TPC operation which activate an immediate feedback for troubleshooting a number of issues. For example,  it easy for the shifter to spot high occupancy events and request a reduction of the primary beam intensity, or to spot a decrease of the argon purity which requires the regeneration of filters, or to catch the presence of electronic noise and reboot the ASICs. An example of high occupancy event is shown in \ref{fig:highOcc}.

\begin{figure}[htb]
\centering
%\includegraphics[scale=0.6]{Chapter-3/Images/RunConditions.png}
\caption{High occupancy event display.}
\label{fig:highOcc}
\end{figure}



\chapter{Total Hadronic Cross Section Measurement Methodology}\label{ch:Interactions}
{\raggedleft ``\emph{Like a lemon to the lime and the bubble to the bee}" \par}
{\raggedleft -- Eazy-E,   1993 -- \par}%Gimmie that *,
\vspace{0.5cm}

This chapter describes the general procedure employed to measure  total hadronic interaction cross sections on Ar in LArIAT.
Albeit with small differences, both the  ($\pi^{-}$,Ar) and (K$^{+}$,Ar) total hadronic cross section measurements rely on the same procedure. We start by selecting the particle of interest using a combination of beamline detectors and TPC information (Section \ref{ch:ParticleSelectionMethod}). We then perform a handshake between the beamline information and the TPC tracking to assure the selection of the correct TPC track (Section \ref{ch:WC2TPCMatchMethod}) associated to the corresponding beam particle. We then apply the ``thin slice" method to measure the ``raw" hadronic cross section (Section \ref{ch:ThinSliceMethod}). A series of corrections are then evaluated and applied to obtain the final cross section (Section \ref{ch:MCCorrections}). 

At the end of this chapter, we show a sanity check of the methodology by applying the thin slice method employing only MC truth information and retrieving the expected MC cross section for pions and kaons (Section \ref{ch:procedureTesting}).



\section{Event Selection}\label{ch:ParticleSelectionMethod}
The measurement of the ($\pi^{-}$,Ar) and (K$^{+}$,Ar) total hadronic cross section in LArIAT starts by selecting the pool of pion or kaon candidates and measuring their momentum before they enter the LAr volume.  This is done through the series of selections on  beamline and TPC information described in the next sections. The summary of the event selection in data is reported in Table \ref{tab:beamlineDataSelection}.


\begin{table}[b]
\centering
\begin{tabular}{|l|c|c|}
\hline
                                                        & Run-II Neg Pol   &  Run-II Pos Pol  \\ \hline
1. Events Reconstructed in Beamline        &  158396  & 260810  \\ \hline
2. Events with Plausible Trajectory            &   147468 & 240954  \\ \hline
3. Beamline $\pi^-/\mu^-/e^-$  Candidate  &   138481 &     N.A.   \\ \hline
4. Beamline $K^+$   Candidate                 &    N.A       & 2837     \\ \hline
5. Events Surviving Pile Up Filter              &   108929  & 2389       \\ \hline
6. Events with WC2TPC Match                 &    41757   & 1081 \\ \hline
7. Events Surviving Shower Filter             &    40841    &  N.A.     \\ \hline
8. Available Events For Cross Section      &   40841    &   1081    \\ \hline
\end{tabular}
\caption{Number of data events for Run-II Negative and Positive polarity }
\label{tab:beamlineDataSelection}
\end{table}


\subsection{Selection of Beamline Events}\label{ch:beamlineDetectorsData}
We leverage the beamline particle identification and momentum measurement before entering the TPC as an input to evaluate the kinetic energy for the hadrons used in the  cross sections measurements. To this end, we select the LArIAT data to keep only events whose wire chamber and time of flight information is registered (line 1 in in Table \ref{tab:beamlineDataSelection}). Additionally, we perform a check of the plausibility of the trajectory inside the beamline detectors: given the position of the hits in the four wire chambers, we make sure the particle's trajectory does not cross any impenetrable material such as the collimator and the magnets steel (line 2 in in Table \ref{tab:beamlineDataSelection}).


\subsection{Particle Identification in the Beamline}
In data, the main tool to establish the identity of the hadron of interest is the LArIAT tertiary beamline, in its function of mass spectrometer. We combine the measurement of the time of flight, $TOF$, and the beamline momentum, $p_{Beam}$, to reconstruct the invariant mass of the particles in the beamline, $m_{Beam}$, as follows
\begin{equation}
m_{Beam} = \frac{p_{Beam}}{c}\sqrt{\biggl(\frac{TOF*c}{l}\biggr)^2 -1},
\label{eq:mass}
\end{equation}
 where $c$ is the speed of light and $l$ is the length of the particle's trajectory between the time of flight paddels. 

Figure \ref{fig:mass} shows the mass distribution for the Run II negative polarity runs on the left and positive polarity runs on the right. We perform the classification of events into the different samples as follows:

\begin{itemize}
\item \underline{$\pi/\mu/e$:}  mass $<$ 350~MeV/c$^2$

\item \underline{kaon:} 350~MeV $<$ mass $<$ 650~MeV/c$^2$

\item \underline{proton:} 650~MeV $<$ mass $<$ 3000~MeV/c$^2$.

\end{itemize}

Lines 3 and 4 in in Table \ref{tab:beamlineDataSelection} show the number of negative $\pi/\mu/e$ and positive $K$ candidates which pass the mass selection for LArIAT Run-II data.

%\begin{comment}     
\begin{figure}
  \centering  
\includegraphics[width=\textwidth]{Chapter-4/Images/massRunII.png}
\caption{Distribution of the beamline mass as calculated according to equation \ref{eq:mass} for the Run-II events reconstructed in the beamline, negative polarity runs on the left and positive polarity runs on the right. The classification of the events into $\pi^\pm/ \mu^\pm/e^\pm$, K$^\pm$, or (anti)proton is based on these distributions, whose selection cut are represented by the vertical colored lines.}
\label{fig:mass}
\end{figure}

\subsection{TPC Selection: Halo Mitigation }\label{ch:pileUp}
The secondary beam impinging on LArIAT secondary target produces a plethora of particles which propagates downstream. The presence of upstream and downstream collimators greatly abates the number of particles traveling down the LArIAT tertiary beamline. However, it is possible that more than one particle sneaks into the LArTPC during its readout time: the TPC readout is triggered by the particle  firing the series of beamline detectors along our tertiary beamline, but particles from the beam halo might also be present in the TPC at the same time. We call ``pile up" the additional traces in the TPC. We adjusted the primary beam intensity between LArIAT Run I and Run II to reduce the presence of events with high pile up particles in the data sample. For the cross section analyses, we remove events with more than 4 tracks in the first 14 cm upstream portion of the TPC from the sample (line 5 in in Table \ref{tab:beamlineDataSelection}).


\subsection{TPC Selection: Shower Removal}\label{ch:electrons}
In the case of the ($\pi^-$,Ar) cross section, the resolution of  beamline mass spectrometer is not sufficient to select a beam of pure pions. In fact,  muons which are close in mass to the pions and relativistic electrons survive the selection on the beamline mass.  It is important to notice that the composition of the negative polarity beam is mostly pions, as will be discussed in section \ref{ch:beamlineComposition}.
Still, we devise a selection on the TPC information to mitigate the presence of electrons in the sample used for the pion cross section. The selection relies on the different topologies of a pion and an electron event when propagating in liquid argon: while the former will trace a track inside the TPC active volume, the latter will tend to ``shower", i.e. interact with the medium, producing bremsstrahlung photons which pair convert into several short tracks. In order to remove the shower topology, we create a region of interest (ROI) around the TPC track corresponding to the beamline particle. We look for short tracks contained in the ROI, as depicted in Figure \ref{fig:showerFilt}:  if more then 5 tracks shorter than 10 cm are in the ROI, we reject the event. Line 7 in  Table \ref{tab:beamlineDataSelection} shows the number of events surviving this selection; that table also shows that this selection is applied after the beamline event is matched to TPC particle (discussed in the next section). This match already lowers the presence of electrons in the sample, which is further reduced by the shower filter.

\begin{figure}
  \centering  
\includegraphics[width=\textwidth]{Chapter-4/Images/Shower.png}
\caption{Visual rendering of the shower filter. The ROI is a cut cone, with a small radius of 4 cm, a big radius of 10 cm and an height of 42 cm (corresponding to 3 radiation lengths for electrons in Argon).}
\label{fig:showerFilt}
\end{figure}



\section{Beamline and TPC Handshake: the Wire Chamber to TPC Match}\label{ch:WC2TPCMatchMethod}
For each event passing the selection on its beamline information, we need to identify the track inside the TPC corresponding to the particle which triggered the beamline detectors, a procedure we refer to as ``WC to TPC match" (WC2TPC for short). In general, the TPC tracking algorithm can reconstruct more than one track in the event, partially due to the fact that hadrons interact in the chamber and partially because of pile up particles during the triggered TPC readout time, as shown in Figure~\ref{fig:kaonInteraction}. 


\begin{figure}
  \centering  
\includegraphics[width=\textwidth]{Chapter-4/Images/KaonExample.png}
\caption{Kaon candidate event: on the right, event display showing raw quantities; on the left, event display showing reconstructed tracks. In the reconstructed event display, different colors represent different track objects. A kink is visible in the kaon ionization, signature of a hadronic interaction: the tracking correctly stops at the kink position and two tracks are formed. An additional pile-up track is so present in the event (top track in red).}
\label{fig:kaonInteraction}
\end{figure}



We attempt to uniquely match one wire chamber track (see Section \ref{sec:MWPCfunc}) to one and only one reconstructed TPC track. 
In order to determine if a match is present, we apply a geometrical selection on the relative position of the wire chamber and TPC tracks. 
We start by considering only TPC tracks whose first point is in the first 2 cm upstream portion of the TPC for the match.  We project the wire chamber track to the TPC front face where we define the coordinates of the projected point as  $x_{FF}$ and $y_{FF}$.  For each considered TPC track, we define $\Delta$X as the difference between the $x$ position of the most upstream point of the TPC track and $x_{FF}$.  $\Delta$Y is defined analogously. We define the radius difference, $\Delta$R, as $ \Delta \text{R} =  \sqrt{ \Delta \text{X}^2 +  \Delta \text{Y}^2}  $. We define  as $\alpha$ the angle between the incident WC track and the TPC track in the plane that contains them.  If  $\Delta \text{R} < 4 $~cm, $\alpha < 8^\circ $,  a match between WC-track and TPC track is found. We describe  how we determine the value for the radius and angular selection in Appendix \ref{ch:AppendixTrack}.
We discard events with multiple WC2TPC matches. We use only those TPC tracks that are matched to WC tracks in the cross section calculation. Line 6 in Table \ref{tab:beamlineDataSelection} shows the number of events where a unique WC2TPC match was found.

In MC, we mimic the matching between the WC and the TPC track by constructing an artificial WC track using truth information at wire chamber four. We then apply the same WC to TPC matching algorithm as in data. 


\begin{figure}
  \centering  
\includegraphics[width=\textwidth]{Chapter-4/Images/WC2TPCMatchTracks.png}
\caption{Visual rendering of the wire chamber to TPC match.}
\label{fig:showerFilt}
\end{figure}

\section{The Thin Slice Method}\label{ch:ThinSliceMethod}



Once we have selected the 40841 beamline pion candidates  and the 1081 beamline kaon candidates, and we have identified the TPC corresponding track, we apply the thin slice method to measure the cross section, as the following sections describe. 
\subsection{Cross Sections on Thin Target}
Cross section measurements on a thin target have been the bread and butter of nuclear and particle experimentalists since the Geiger-Marsden experiments \cite{Geiger1909}. At their core, this type of experiments consists in shooting a beam of particles with a known flux on a thin slab of material and recording the outgoing flux. 


In general even in the case of thin target, the target is not a single particle, but rather a slab of material containing many diffusion centers. The so-called  ``thin target" approximation assumes that the target centers are uniformly distributed in the material and that the target is thin compared to the projectile interaction length, so that no center of interaction sits in front of another. In this approximation, the ratio between the number of particles interacting in the target $N_{\text{Int}}$ and the number of incident particles $N_{\text{Inc}}$ on the target estimates the interaction probability $P_{Interacting}$, which is the complementary to one of the survival probability $P_{Survival}$. 
Equation \ref{eq:thinTargetXS} 
\begin{equation}
P_{Survival} = 1- P_{Interacting} = 1 - \frac{N_{\text{Int}}}{N_{\text{Inc}}} = e^{-\sigma_{TOT}\text{ } n \text{ }\delta X}
\label{eq:thinTargetXS}
\end{equation}
describes the probability for a particle to survive the thin target. This formula relates  the interaction probability to the total hadronic cross section ($\sigma_{TOT}$), the density of the target centers ($n$)\footnote{The scattering center density in the target, {\emph{n}},  relates to the argon density $\rho$, the Avogadro number  $ N_{A} $ and the argon molar mass $m_A$ as $n=\frac{\rho N_{A} }{m_A}$.}    and  the thickness of the target  along the incident hadron direction ($\delta X$). If the target is thin compared to the interaction length of the process considered, we can Taylor expand the exponential function in equation \ref{eq:thinTargetXS} and find a simple proportionality relationship between the cross section and the number of incident and interacting particles, as shown in equation \ref{eq:thinTargetXSTaylor}:
\begin{equation}
1 - \frac{N_{\text{Int}}}{N_{\text{Inc}}} =  1 -\sigma_{TOT} \text{  }n \text{  }\delta X + O(\delta X^2).
\label{eq:thinTargetXSTaylor}
\end{equation}

Solving for the cross section, we find:
\begin{equation}
 \sigma_{TOT}  = \frac{1}{n \text{ }\delta X}\frac{N_{\text{Int}}}{N_{\text{Inc}}}.
\label{eq:thinTargetXSSolved}
\end{equation}

\subsection{Not-so-Thin Target: Slicing the Liquid Argon Volume}\label{ch:XSRaw}
The interaction length of pions and kaons in liquid argon is expected to be of the order of 50 cm for pions and 100 cm for kaons. Thus, the LArIAT TPC, with its 90 cm of length, is not a thin target. However, the granularity of the LArIAT LArTPC detector allows us to treat the argon volume as a sequence of many adjacent thin targets. 

As described in Chapter \ref{sec:experimentDescription}, LArIAT induction and collection planes consist of 240 wires each at 4 mm spacing. The wires are oriented at +/- $60^{\circ}$ from the vertical direction, while the beam direction is oriented 3 degrees off the $z$ axis in the $XZ$ plane.  The collection wires collect signals proportional to the energy deposited by the hadron along its path in a  $\delta${\emph{X}} = 4 mm/(sin($60^{\circ}$)cos($3^{\circ}$)) $\approx$ 4.7~mm slab of liquid argon. Thus, one can think to slice the TPC into many thin targets of $\delta${\emph{X}} = 4.7~mm thickness along the direction of the incident particle, making a measurement at each wire along the path, as sketched in Figure \ref{fig:TPCGran}.

\begin{figure}
  \centering  
\includegraphics[width=0.8\textwidth]{Chapter-4/Images/LArSlice.png}
\caption{Representation of sliced LAr Volume.}
\label{fig:TPCGran}
\end{figure}


Considering each slice {\emph{j}}  a ``thin target",  we can apply the cross section calculation from Equation~\ref{eq:thinTargetXSSolved} iteratively, evaluating the kinetic energy of the hadron as it enters each slice, $E_{j}^{kin}$.  For each WC2TPC matched particle, the energy of the hadron entering the TPC is known thanks to the momentum and mass determination by the tertiary beamline, 

\begin{equation}
 E^{kin}_{Front Face}  = \sqrt{p^2_{Beam} - m^2_{Beam}} - m_{Beam} - E_{loss},
\label{eq:enFF}
\end{equation}
where $E_{loss}$ is a correction for the kinetic energy loss in the uninstrumented material between the beamline and the TPC front face. While propagating through the target,  the kinetic energy of the hadron at each slab is determined by subtracting the energy deposited by the particle in the previous slabs. For example, at the $j^{th}$ slab of a track, the kinetic energy will be

\begin{equation}
 E_{j}^{kin} =  E^{kin}_{Front Face} - \sum_{i < j} E_{\text{Dep},i},
\label{eq:KEj}
\end{equation}
where $E_{\text{Dep},i}$ is the energy deposited at each argon slice before the $j^{th}$ point as measured by the calorimetry associated with the tracking.


If the particle enters a slice, it contributes to the $N_{\text{Inc}}( E^{kin})$ distribution in the energy bin corresponding to its kinetic energy in that slice. While into the slice, a hadron may or may not interact. If it interacts in the slice, it  contributes also to the $N_{\text{Int}}(E^{kin})$ distribution in the appropriate energy bin; this occurrence corresponds to the end of the hadron tracking. If the hadron does not interact, it will enter the next slice and the interaction evaluation starts again.
The process is applied to all the hadrons in the sample; the cross section as a function of kinetic energy, $\sigma_{TOT}( E^{kin})$ is then evaluated to be proportional to the ratio $\frac{N_{\text{Int}}( E^{kin})}{N_{\text{Inc}}( E^{kin})}$ -- bin by bin ratio. 


Our goal is to measure the total interaction cross section, independently  from the topology of the interaction. Thus, we determine that a hadron interacted simply by requiring that the last point of the WC2TPC matched track lies in a slice within the fiducial volume, whose boundaries are defined in Table \ref{tab:FidVol}. If the TPC track ends within the fiducial volume, its last point will be the interaction point; if the track crosses the boundaries of the fiducial volume, the track will be considered ``through going" and no interaction point will be found. The only points of the hadronic candidate track considered to fill the  $N_{\text{Int}}$ and  $N_{\text{Inc}}$ distributions are the ones contained in the fiducial volume. 
 
 A notable background pertinent only to the $N_{\text{Int}}$  distribution are cases in which the hadrons decays inside the TPC. In those cases in fact, the tracking ends inside the TPC but the interaction is not hadronic. The handling of decay background is treated in a slightly different way for the pion and kaon section, details can be found in sections \ref{ch:PionXSBkgSub} and \ref{ch:KaonXSRaw} respectively.



\begin{table}[t]
\centering
\begin{tabular}{|l|r|r|}
\hline
& min   &  max  \\ \hline
$X$ & 1 cm   & 46 cm  \\ \hline
$Y$ & -15 cm   & 15  cm  \\ \hline
$Z$ & 0 cm   & 86 cm  \\ \hline
\end{tabular}
\caption{Fiducial volume boundaries used to determine cross section interaction point. }
\label{tab:FidVol}
\end{table}



\subsection{Corrections to the Raw Cross Section}\label{ch:MCCorrections}
%%%%%%%%%%%%%%%%%%%%%%%%%%
Equation \ref{eq:thinTargetXSSolved}  is a prescription for measuring the cross section in case of a pure beam of the hadron of interest and 100\% efficiency in the determination of the interaction point.  For example, if LArIAT had a beam of pure pions and were 100\% efficient in determining the interaction point within the TPC, the pion cross section as a function of  kinetic energy (estimated at the central value of the energy bin $E_i$) would be given by

\begin{equation}
 \sigma^{\pi^-}_{TOT}(E_{i})  = \frac{1}{n\text{ } \delta X}\frac{N^{\pi^-}_{ \text{Int}} (E_{i})}{N^{\pi^-}_{ \text{Inc}}(E_{i})}.
\label{eq:thinTargetXSSolved2}
\end{equation}

Unfortunately, this is not the case. In fact, the selection used to isolate pions in the LArIAT beam allows for the presence of some muons and electrons as background, while the kaon selection allows for a small contamination of protons (see Section \ref{ch:beamlineComposition}). Also, the LArTPC tracking algorithm is not 100\% efficient in determining the interaction point. This inefficiency occurs in two fashions: i) the tracking algorithm does not stop at the interaction point and continues adding hits from a particle past it (this happens especially in the case of shallow elastic scattering), ii) the tracking stops prematurely. These two cases have different consequences on the population of the interacting and incident distributions. In the first case, the interacting histogram will be underpopulated and the incident histogram might be overpopulated. In case of premature end of tracking, the interacting histogram will be overpopulated at energies greater than the eventual interaction, while the incident histogram will be underpopulated. Given the importance of tracking for the cross section measurements,  we report an optimization to maximize the identification of the interaction point in Appendix \ref{ch:AppendixTrack}.

Therefore, we apply two corrections evaluated on MC in order to extract the final cross section from LArIAT data: i) a background subtraction and ii) a correction for reconstruction effects. 
Still using the pion case as example, we estimate the pion cross section in each energy bin changing  Equation \ref{eq:thinTargetXSSolved2} into
\begin{equation}
 \sigma^{\pi^-}_{TOT}(E_{i})  =\frac{1}{n\text{ } \delta X}\frac{N^{\pi^-}_{ \text{Int}} (E_{i})}{N^{\pi^-}_{ \text{Inc}}(E_{i})} = \frac{1}{n \text{ }\delta X}\frac{ \epsilon^{\text{Inc}}(E_i) [ N^{ \text{TOT}}_{ \text{Int}} (E_{i}) - B_{ \text{Int}} (E_i)] }{   \epsilon^{\text{Int}}(E_i) [N^{ \text{TOT}}_{ \text{Inc}}(E_{i}) - B_{ \text{Inc}} (E_i)]},
\label{eq:True}
\end{equation}


 
where  $N^{\text{TOT}}_{\text{Int}} (E_{i})$ and $N^{\text{TOT}}_{\text{Incident}} (E_{i})$ is the measured content of the interacting and incident histograms for events that pass the event selection, $B_{\text{Int}} (E_i)$ and $B_{\text{Inc}} (E_i)$ represent the contributions from the background to the interacting and incident histograms respectively, and  $\epsilon^{\text{Int}}(E_i)$ and  $\epsilon^{\text{Inc}}(E_i)$ are the corrections for reconstruction effects.

As we will show in Section \ref{ch:PionXSBkgSub}, the background subtraction for the interacting and incident histograms can be translated into corresponding relative pion content factors $C^{\pi MC}_{\text{Int}} (E_{i})$ and $C^{\pi MC}_{\text{Inc}} (E_{i})$ and the cross section re-written as follows

\begin{equation}
      \sigma^{\pi^-}_{TOT}(E_{i})  = \frac{1}{n\text{ } \delta X}\frac{ \epsilon^{\text{Inc}}(E_i)  \hspace{0.2cm} C^{\pi MC}_{\text{Int}} (E_{i}) \hspace{0.2cm} N^{\text{TOT}}_{\text{Int}} (E_{i}) }{   \epsilon^{\text{Int}}(E_i) \hspace{0.2cm} C^{\pi MC}_{\text{Inc}} (E_{i}) \hspace{0.2cm}  N^{\text{TOT}}_{\text{Inc}} (E_{i})}.
\label{eq:C}
\end{equation}



%The following sections describe the procedures used to evaluate  the background subtraction (section \ref{sec:beamCont}) and the efficiency correction (section \ref{sec:EffCorrection}), as well as  their uncertainties. 
%The reader might be concerned about bin-by-bin migration of events in the interacting and incident plots due to the finite resolution of the energy reconstruction. In section \ref{sec:Energy}, we make an argument to why we expect the smearing matrix to be extremely close to diagonal, such that its calculation and relative corrections are left for an improvement of the analysis.


%%%%%%%%%%%%%%%%%%%%%%%%%%%%%%%%%%%%%%%%%%%%%%%%



\section{Procedure testing with MC truth quantities}\label{ch:procedureTesting}
The ($\pi^{-}$,Ar) and (K$^{+}$,Ar) total hadronic cross section implemented in Geant4 can be used as a tool to validate the measurement methodology.  We describe here a closure test done on Monte Carlo to prove that the methodology of slicing the TPC retrieves the underlying cross section distribution implemented in Geant4 within the MC statistical uncertainty. 

For pions and kaons in the considered energy range, the Geant4 inelastic model adopted is ``BertiniCascade"; the pion elastic cross sections are tabulated from Chips, while the kaon elastic cross sections are tabulated on Gheisha and Chips.

For the validation test, we fire a sample of pions and a sample of kaons inside the LArIAT TPC active volume using the Data Driven Monte Carlo, a procedure described in Section \ref{sec:DDMC}. We apply the thin-sliced method using only true quantities to calculate the hadron kinetic energy at each slab in order to decouple reconstruction effects from possible issues with the methodology.  For each slab of 4.7 mm length along the path of the hadron, we integrate the true energy deposition as given by the Geant4 transport model. Then, we recursively subtracted it from the hadron kinetic energy at the TPC front face to evaluate the kinetic energy at each slab until the true interaction point is reached. Since the MC is a pure beam of the hadron of interest and truth information is used to retrieve the interaction point, no background correction or reconstruction effects correction is applied. Doing so, we obtain the true interacting and incident distributions for the considered hadron, whose ratio leads to  the true MC cross section as a function of the hadron kinetic energy. 

Figure \ref{fig:TrueMCXS2} shows the total hadronic cross section for argon implemented in Geant4 10.03.p1 (solid lines) overlaid with the true MC cross section as obtained with the sliced TPC method (markers) for pions on the left and kaons on the right; the total cross section is shown in green. For completeness, we also report the contributions from  the elastic cross section (in blue) and the inelastic cross section (in red), available at the MC level.  The nice agreement with the Geant4 distribution and the cross section  obtained with the sliced TPC method gives us confidence in the  validity of the methodology. 
        
%\begin{comment}     
\begin{figure}
%\captionsetup{justification=raggedright}  
\begin{minipage}[b]{.53\textwidth}  
  \centering  
\includegraphics[width=3in]{Chapter-4/Images/PionTrueXS.png}
\end{minipage}%  
\begin{minipage}[b]{0.53\textwidth}  
  \centering  
\includegraphics[width=3in]{Chapter-4/Images/KaonTrueXS.png}
\end{minipage}
\caption{Hadronic cross sections for ($\pi^-$,Ar) on the left and (K$^+$,Ar) on the right as implemented in Geant4 10.03.p1 (solid lines) overlaid the true MC cross section as obtained with the sliced TPC method (markers). The total cross section is shown in green,  the elastic cross section in blue and the inelastic cross section in red.}
\label{fig:TrueMCXS2}
\end{figure}
%\end{comment}






\chapter{Preparatory Work}\label{ch:samples}
This chapter describes the preparatory work done on the the data and Monte Carlo samples used for the cross section analyses. This entails:

\begin{enumerate}
\item the MC production,
\item the energy calibration of the detector both in data and MC,
\item the optimization of the tracking algorithm for the total cross section analyses.
\end{enumerate}

\section{Construction of a Monte Carlo Simulation for LArIAT}
\subsection{G4Beamline}\label{ch:beamlineComposition}
\subsection{Data Driven MC}\label{sec:DDMC}

\section{Tracking Studies}

\section{Energy Calibration and Studies}

\section{Estimate of Energy Loss before the TPC}

\begin{comment}
\section{Construction of a Monte Carlo Simulation for LArIAT}
For the simulation of LArIAT events and their particle make up, we use a combination of two MC generators: the G4Beamline Monte Carlo and the Data Driven single particle Monte Carlo (DDMC). We use the G4Beamline MC to simulate the particle transportation in the beamline and calculate the particle composition of the beam just past the fourth Wire Chamber (WC4). In order to simulate the beam line particles after WC4 and in the TPC, we use the DDMC.
\subsection{G4Beamline}\label{beamlineComposition}
G4Beamline simulates the beam collision with the LArIAT secondary target, the energy deposited by the particles in the LArIAT beamline detectors and the action of the LArIAT magnets, effectively accounting for particle transportation through the beam line from the LArIAT target until ``Big Disk", a fictional, void detector located just before the cryostat. 
 At the moment of this writing, G4Beamline does not simulated the responses of the beam line detectors. It is possible to interrogate the truth level information of the simulated particles in several points of the geometry. In order to ease the handshake between G4Beamline and the DDMC, we ask for the beam composition just after WC4.
Since LArIAT data are taken under different beam conditions, G4Beamline simulates separately the beam composition according to the magnets' settings and the secondary beam intensity. For the pion cross section analysis the relevant beam conditions are  secondary beam energy of 64 GeV, negative polarity magnet with current of 100 A and 60 A. For the kaon cross section analysis the relevant beam conditions is a secondary beam energy of 64 GeV, positive polarity magnet with current of 100 A. 

\textcolor{red}{DECIDE IF YOU WANT THE BEAM COMPOSITION HERE}
%Figure 16 shows the tertiary beam spectra for the 64 GeV and 100 A. 

%In Table 3, the beam composition is given in terms of percentage of different particle species per spill for positive polarity. The values reported are the weighted average on the two beam conditions considered. The weights are calculated according to the fourth column of Table 2.

\subsection{Data Driven MC}\label{sec:DDMC}
The Data Driven single particle Monte Carlo (DDMC) is a single particle MC gun which simulates the particle transportation from WC4 into the TPC leveraging on the beamline data information. The DDMC uses the data momentum and position at WC4 to derive its initial conditions: a general sketch of the DDMC workflow is shown in Figure \ref{fig:DDMCSketch}.

When producing a DDMC sample, beam line data from a particular running period and/or running condition are selected first. Figure \ref{fig:DDMCQuantities}  schematically shows the data quantities of interest leveraged from data: the momentum ($P_x, P_y, P_z$) and position ($X, Y$) at WC4. For each data event, we obtain the particle position ($X, Y$) at WC4 directly from the data measurement. On the contrary, we calculate the components of the momentum using the beamline measurement of the momentum magnitude (see section \ref{sec:MWPCfunc}) in conjunction with the hits on WC3 and WC4 to determine the direction of the momentum vector, as described in \ref{sec:MWPCfunc}. The momentum and position of the selected data is sampled thousand of times through a 5-dimensional hit-or-miss sampling procedure. This produces MC distributions with the same momentum and position distributions as data, with the additional benefit of accounting for the correlations between the considered variables. A LArSoft simulation module then launches single particle MC from z = -100 cm (the location of the WC4) using the sampled momentum and position distributions as a template. 
As an example, the results of the DDMC generation compared to data for the pion 60A sample are shown in figure \ref{DDMCComparison}; as expected, MC and data agree within the statistical uncertainty by construction. Using this technique ensures the MC and data particles have very similar momentum, position and angular distributions at WC4 and allow us to us the MC sample in several occasions, for example to calibrate the energy loss upstream of the TPC or account for the WC2TPC match inefficiency. A small caveat is in order here: the DDMC is a single particle Monte Carlo, which means that the beam pile-up is not simulated. 
Three sample of \textcolor{red}{NUMBERS} pions, muons and electrons, as well as  a sample of \textcolor{red}{NUMBERS} kaons have been generated with the DDMC and are used for the MC cross section study.

\begin{figure}[hpbt]
\centering
\includegraphics[width=\textwidth]{Chapter-5/Images/DDMCScheme.png}
\caption{Workflow for Data Driven single particle Monte Carlo production.}
\label{fig:DDMCSketch}
\end{figure}


\begin{figure}[hpbt]
\centering
\includegraphics[width=\textwidth]{Chapter-5/Images/DDMCQuantities.png}
\caption{Scheme of the quantities of interest for the DDMC event generation: $P_x, P_y, P_z, X, Y$ at WC4.}
\label{fig:DDMCQuantities}
\end{figure}


\begin{figure}[hpbt]
\centering
\includegraphics[width=\textwidth]{Chapter-5/Images/Pz.png}
\caption{Comparison between generated quantities and data distributions for the 60A pion sample: Z component of the momentum (top left), X position at Wire Chamber 4 (top right), Y position at Wire Chamber 4 (bottom).}
\label{fig:DDMCComparison}
\end{figure}



%%%%%%%%%%%%%%%%%%%%%%%%%%%%%%%%%%%%%%%%%%%
%%%%%%%%%%%%%%         Energy Calibration         %%%%%%%%%%%%%%         
%%%%%%%%%%%%%%%%%%%%%%%%%%%%%%%%%%%%%%%%%%%
\section{Energy Calibration}\label{ch:energyCalibration}
Scope of the energy calibration is to identify the factors which convert the charge collected (dQ) to energy deposited in the chamber(dE). As described in section \ref{sec:SignalProc}, this is a multi-step procedure. In LArIAT, we first correct the raw charge by the electronic noise on the considered wire \cite{technote}, then by the electron lifetime \cite{LArIATLifeTime},  and then by the recombination using the ArgoNeut recombination values. Lastly, we apply overall calibration of the energy, i.e. we determine the ``calorimetry constants" using the procedure described in this section.


We independently determine  the calorimetry constants for Data and Monte Carlo in the LArIAT Run-II Data samples using  a parametrization of the energy deposited per unit length (dE/dX) as a function of momentum. This is done by comparing the stopping power measured on reconstructed quantities against the Bethe-Bloch theoretical prediction for various particle species (see equation \ref{eq:BB}).  We obtain the theoretical expectation for the dE/dX most probable value of pions ($\pi$), muons ($\mu$), kaons ($K$), and protons ($p$) in the momentum range most relevant for LArIAT (Figure \ref{fig:PDGEnergyLossArgon}) using the tables provided by the Particle Data Group \cite{Patrignani:2016xqp} for liquid argon \cite{PDG-Argon}.

The basic idea of this calibration technique is to utilize the most upstream portion of a TPC track which has a well known momentum and particle species to measure its $dE/dX$. Once a sample of particles dE/dX has been measured at various momenta, we then tune to calorimetry constants within the reconstruction software to align these measured values to match the theoretical ones found in Figure \ref{fig:PDGEnergyLossArgon}. 

In data, we start by selecting a sample of beamline positive pion candidates without any restriction on their measured momentum.
We then apply the WC2TPC match and subtract the energy loss upstream to the TPC front face, determining the momentum at the TPC front face. For each surviving pion candidate,  we measure the dE/dx at each of the first 12 spacepoints associated the 3D reconstructed track, corresponding to a $\sim$ 5 cm portion. These dE/dX measurements are then put into a histogram that corresponds to measured momentum of the track. The dE/dX histograms are sampled every 50 MeV in momentum (e.g. 150~MeV/c $< P <$ 200~MeV/c, 200~MeV/c $< P <$ 250/c~MeV, etc...).   This process of selecting, sampling, and recording the dE/dX for various momentum bins is repeated over the entire sample of events, allowing us to collect sufficient statistic in most of the momentum bins between 150~MeV/c and 1100~MeV/c.
Each 50 MeV/c momentum binned dE/dX histogram is now fit with a simple Landau function. The most probable value (MPV) and the associated error on the MPV from the fit are extracted and plotted on Figure \ref{fig:PDGEnergyLossArgon}. Depending on the outcome of the fit, we modify the calorimetry constants and we repeat the procedure until a qualitative agreement is achieved.  We perform this  tuning for the collection and induction plane separately. 
As a cross check to the determined calorimetry constants using the positive pions, we plot the dE/dx vs momentum distribution all the other particle species identifiable in the beamline data ($\pi$, K , p, in both polarities) against the corresponding Beth-Bloch prediction in a similar fashion. On average, pions and muons only lose $\sim$10 MeV in this 5~cm section of the track and protons lose $\sim$20 MeV. Thus choosing 50 MeV size bins for our histograms covers the energy spread within those bins due to energy loss from ionization.  The results of the tuning and cross check for Run-II data on the collection plane is shown in Figure \ref{fig:BBandData} positive polarity data on the left, negative polarity data on the right.

In MC, we simulate the corresponding positive pion sample with the DDMC (see section \ref{sec:DDMC}) and follow the same steps as in data. The calorimetry tuning is explained in all his gory details in \cite{technote}.


\begin{figure}[htb]
\centering
%\includegraphics[width=0.50\textwidth]{images/PDGdEdX.png}
\caption{Mean energy loss in various materials over a range of particle momenta as produced in Reference \cite{PDG}.}
\label{fig:PDGEnergyLoss}
\end{figure}


\begin{figure}[htb]
\centering
%\includegraphics[width=0.50\textwidth]{images/dEdXvsMomentumTemplate}
\caption{Mean energy loss for pions, muons, and protons in liquid argon over the momentum range most relvant for LArIAT.}
\label{fig:PDGEnergyLossArgon}
\end{figure}




\begin{figure}[htb]
\centering
%\includegraphics[width=0.50\textwidth]{images/CalibrationExample.png}
\caption{Illustration of the calibration technique. Here we depict a 325 MeV wire chamber track (shown in green) which enters the TPC (taking into account the energy loss from the upstream material) and we sample the first 12 spacepoints (shown in teal) to extract the dE/dX distribution which is fit with a Landau.}
\label{fig:CalibrationExample}
\end{figure}





%%%%%%%%%%%%%%%%%%%%%%%%%%%%%%%%%%%%%%%%%%%
%%%%%%%%%%%%%%         Energy Calibration         %%%%%%%%%%%%%%         
%%%%%%%%%%%%%%%%%%%%%%%%%%%%%%%%%%%%%%%%%%%




\section{Tracking Studies}
In this section, we describe three studies. The first is a justification of the selection criteria for the beamline handshake with the TPC information.  We perform this study to boost  the correct identification of the particles in the TPC associated with the beamline information, while maintaining sufficient statistics for the cross section measurement. 
The second study is an optimization of the tracking algorithm, with the scope of maximizing the identification of the hadronic interaction point inside the TPC. These two studies are related, since the optimization of the tracking is performed on TPC tracks which have been matched to the wire chamber track; in turn, the tracking algorithm for TPC tracks determine the number of reconstructed tracks in each event used to try the matching with the wire chamber track. Starting with a sensible tracking reconstruction, we perform the WC2TPC matching optimization first, then the tracking optimization. The WC2TPC match purity and efficiency  are then calculated again with the optimized tracking.


%\section{MC sample and WC2TPC match}
We perform the following studies on a MC sample of 191000 kaons and 359000 pions produced with the DDMC technique. DDMC particles are shot from the WC4 location into the TPC following the beam profile.
We mimic the matching between the WC and the TPC track on Monte Carlo by constructing a fake WC track using truth information at wire chamber four. We then apply the same WC to TPC matching algorithm as in data described in \ref{ch:WC2TPCMatchMethod}. 



%In data, we attempt to uniquely match one WC-Track to one and only one reconstructed TPC track. This match is done by using in the $X$ and $Y$ coordinate of the extrapolated WC-Track to the upstream most point of the reconstructed TPC Track and by using the angle between the incoming track angle and the reconstructed TPC. We define $\Delta$X as the difference between the $x$ position of the most upstream point of the TPC track and the $x$ position of the WC track as projected to the TPC front face. $\Delta$Y is defined analogously. We define  $\Delta$R as $ \Delta \text{R} =  \sqrt{ \Delta \text{X}^2 +  \Delta \text{Y}^2}  $. The angle between the incident WC Track and the TPC track in the plane that contains them defines $\alpha$.  

%We define a match between WC-track and TPC reconstructed track if  $\Delta \text{R} < r_{T}$, $\alpha < \alpha_{T}$ and the Z position of the first reconstructed point of the TPC track is within 2 cm from the TPC front face. The determination of the best $r_{T}$ and $\alpha_{T}$ is the scope of the following section.

%In MC, we mimic the matching between the WC and the TPC track on Monte Carlo by constructing a fake WC track using truth information at wire chamber four. We then apply the same WC to TPC matching algorithm as in data. 

\subsection{Selection Study for the Wire Chamber to TPC Match}\label{ch:WC2TPCMatchOptimization}
Plots I want in this section:
\begin{enumerate}
\item WC2TPC MC DeltaX, DeltaY and $\alpha$
\end{enumerate}


Scope of this study is assessing the goodness of the wire chamber to TPC match on Monte Carlo and decide the selection values we will use on data. A word of caution is necessary here. With this study, we want to minimize pathologies associated with the presence of the primary hadron itself, e.g. the incorrect association between the beamline hadron and its decay products inside the TPC.  Assessing the contamination from pile-up\footnote{We remind the reader that the DDMC is a single particle Monte Carlo, where the beam pile up is not simulated.}, albeit related, is beyond the scope of this study.

In MC, we are able to define a correct WC2TPC match using the Geant4 truth information. We are thus able to count how many times the WC tracks is associated with the wrong TPC reconstructed track. 

We define a correct match if the all following conditions are met:
\begin{itemize}
\item[-] the length of the true primary Geant4 track in the TPC is greater than 2 cm,  
\item[-] the length of the reconstructed track length is greater than 2 cm,
\item[-] the Z position of the first reconstructed point is within 2 cm from the TPC front face
\item[-] the distance between the reconstructed track and the true entering point is the minimum compared with all the other reconstructed tracks.
\end{itemize}

In order to count the wrong matches, we consider all the reconstructed tracks whose Z position of the first reconstructed point lies within 2 cm from the TPC front face. Events with true length in TPC $<$ 2 cm are included. 
Since hadrons are shot 100 cm upstream from the TPC front face, the following two scenarios are possible from a truth standpoint: 
\begin{itemize}
\item[[$Ta$]] the primary hadron decays or interact strongly before getting to the TPC,
\item[[$Tb$]] the primary hadron enters the TPC.
\end{itemize}

Once we choose the selection cuts to determine a reconstructed wire chamber-to-TPC match $r_{T}$ and $\alpha_{T}$, the following five scenarios are possible in the truth to reconstruction interplay : 
\begin{itemize}
\item[1)] only the correct track is matched
\item[2)] only one wrong track is matched 
\item[3)] the correct track and one (or more) wrong tracks are matched
\item[4)] multiple wrong tracks  matched.
\item[5)] no reconstructed tracks are matched
\end{itemize}

Since we keep only events with one and only one match, we discard cases 3), 4) and 5) from the events used in the cross section measurement. For each set of $r_{T}$ and $\alpha_{T}$ selection value, we define purity and efficiency of the selection as follows:
\begin{equation}
\text{Efficiency} = \frac{\text{Number of events correctly matched}}{\text{ Number of events with primary in TPC}}
\end{equation}

\begin{equation}
\text{Purity} = \frac{\text{Number of events correctly matched}}{\text{Total number of matched events}}.
\end{equation}

Figure \ref{fig:EffPurityK} shows the efficiency (left) and purity (right) for wire chamber-to-TPC match as a function of the radius, $r_{T}$, and angle, $\alpha_{T}$, selection value. It is apparent how both efficiency and purity are fairly flat as a function of the radius selection value at a given angle. This is not surprising. Since we are studying a single particle gun Monte Carlo sample, the wrong matches can occur only for mis-tracking of the primary or for association with decay products;  decay products will tend to be produced at large angles compared to the primary, but could be fairly close to the in $x$ and $y$ projection of the primary. The radius cut would play a key role in removing pile up events. 

For LArIAT cross section measurements, we generally prefer purity over efficiency, since a sample of particles of a pure species will lead to a better measurement. Obviously, purity should be balanced with a sensible efficiency to avoid rejecting the whole sample. 

We choose $(\alpha_{T}$, $r_{T}) = (8 \text{ deg}, 4 \text{ cm} )$ and get a MC 85\% efficiency and 98\% purity for the kaon sample and a MC \textcolor{red}{BOH}\% efficiency and 98\% purity for the \textcolor{red}{BOH} sample.


\begin{figure}[hpbt]
\centering
\includegraphics[width=15cm]{Chapter-5/Images/KEffPurity.png}
\caption{Efficiency (left) and purity (right) for wire chamber-to-TPC match as a function of the radius and angle selections.}
\label{fig:EffPurityK}
\end{figure}

\subsection{Interaction Point Optimization}\label{ch:TrackingOptimization}
Scheme of this subsection
\subsubsection{Brief Explanation of the reconstruction chain}
\subsubsection{Explanation of clustering parameters}
\subsubsection{Figure of merit and  spanning of cluster}
\subsubsection{Important numbers out of this optimization}


Plots I want in this section:
\begin{enumerate}
\item Delta L, reco - true
\item Delta L, reco - true Elastic, Delta L, reco - true Inelastic, other
\item Length Quality cut
\item Efficiency as a function of true KE and Angle
\end{enumerate}


\subsection{Tracking spatial and angular resolution}
Scope of this study is understanding and comparing the tracking spatial and angular resolution on data and MC.
We start by selecting all the WC2TPC matched tracks. 
We fit a line on all the space points of the track and calculate the $\chi^2$. The $\chi^2$ distribution for data and MC is shown in Figure \ref{fig:Chi2AllPts}.

For the spatial and angular resolution study, we reject tracks with less than 14 space points. For each track, we order the space points according to their Z position and we split them in two sets: the first set counts all the points belonging to the first half of the track and the second set counts all the points belonging to the second half of the track. We remove the last 5 points in the first set and the first 5 points in the second set, so to have a gap in the middle of the original track. We fit the first and the second set of points with a line separately. We reject the event entirely if the  $\chi^2$ for the fit of either of the halves is greater than four.  We define a track middle plane as the plane perpendicular to the original track fit, positioned in the middle of its length. We project the tracks on the middle plane and calculate the impact parameter, $d$, i.e. the distance between the projected points. We also calculate the angle between the original track direction and the fit of the first and second half, called $\alpha_1$ and $\alpha_2$ respectively. The spatial resolution of the track will be $\sigma_S = \frac{d}{\sqrt 2}$ while the angular resolution of the tracks will be  $\sigma_\alpha = \alpha_1 - \alpha_2$. The distributions for data and MC for $\sigma_\alpha$ and $\sigma_S$ are given in \ref{fig:trackingResolution}.




\end{comment}





\chapter{Negative Pion Cross Section Measurement}\label{ch:PionXS}
%{\raggedleft ``\emph{Y ella es flama que se eleva, Y es un p\'ajaro a volar} \par}
%{\raggedleft \emph{En la noche que se incendia, Estrella de oscuridad}"\par}
%{\raggedleft \emph{Que busca entre la tiniebla, La dulce hoguera del beso}"\par}
%{\raggedleft -- Lila Downs, Benediction And Dream,  2002-- \par}


\section{Raw Cross Section}\label{ch:PionXSRaw}
We measure the ($\pi^-$-Ar) cross section as a function of the kinetic energy in the two chosen data sets, the -60A and -100A negative runs. 
As we will clarify in Section \ref{ch:PionXSCorrections},  the corrections to the raw cross section depend on the beam conditions and need to be calculated independently for the two data sets. Thus, we present here the measurements on the two datasets separately.


As stated in section \ref{ch:XSRaw},  the raw cross section is given by the equation \label{eq:thinTargetXSSolved}
\begin{equation}
 \sigma_{TOT} (E_i)  = \frac{1}{n \delta X}\frac{N^{\text{TOT}}_{\text{Int}}(E_i)}{N^{\text{TOT}}_{\text{Inc}}(E_i)}.
\end{equation}

where $N^{\text{TOT}}_{\text{Int}}$  is the number of particles interacting at kinetic energy $E_i$, $N^{\text{TOT}}_{\text{Inc}}$ is number of particles incident  on an argon slice at  kinetic energy $E_i$,  $n$ is the density of the target centers  and $\delta X$ is the thickness of the argon slice.


Figure \ref{fig:InteractingRaw} shows the distribution of  $N^{\text{TOT}}_{\text{Int}}$  as a function of the kinetic energy for the 60A dataset on the left and for the 100A dataset on the right. The data central points are represented by black dots, the statistical uncertainty is shown in black, while the systematic uncertainty is shown in red. Data is displayed over the raw cross section obtained with a MC mixed sample of pions, muon and electrons in the percentage predicted by G4Beamline. The contribution from the simulated pions are shown in blue, the ones from secondaries in red, the ones from muons in yellow and the ones from electrons in gray. 
The simulated pion's and backgrounds' contributions are stacked; the sum of the integrals from each particle species is normalized to the integral of the data.
 
Figure \ref{fig:IncidentRaw} shows the distribution of  $N^{\text{TOT}}_{\text{Inc}}$   for the 60A dataset on the left and for the 100A dataset on the right. Data is displayed over the MC. The same color scheme and normalization procedure is used for both the interacting and incident histograms. 


Figure \ref{fig:XSRaw} shows the raw cross section for the 60A dataset on the left and for the 100A dataset on the right, statistical uncertainty in black and systematic uncertainty in red. The raw data cross section is overlaid to the reconstructed MC cross section (in azure). 

The calculation of the statistical uncertainty for the interacting, incident and cross section measurements is laid out in Section \ref{ch:StatUncertaintyXSRaw}, while the  corresponding systematics  uncertainty on Section \ref{ch:SysUncertaintyXSRaw}.

\begin{figure}[p]
\centering  
\includegraphics[width=0.48\textwidth]{Chapter-6/Images/Plots60A_MCData_Int_StatSyst.pdf}
\includegraphics[width=0.48\textwidth]{Chapter-6/Images/Plots100A_MCData_Int_StatSyst.pdf}
\caption{Raw number of interacting pion candidates as a function of the reconstructed kinetic energy for the 60A runs (left) and for the 100A runs (right). The statistical uncertainties are shown in black, the systematic uncertainties in red.}
\label{fig:InteractingRaw}
\end{figure}


\begin{figure}
\centering  
\includegraphics[width=0.48\textwidth]{Chapter-6/Images/Plots60A_MCData_Inc_StatSyst.pdf}
\includegraphics[width=0.48\textwidth]{Chapter-6/Images/Plots100A_MCData_Inc_StatSyst.pdf}
\caption{Raw number of incident pion candidates as a function of the reconstructed kinetic energy for the 60A runs (left) and for the 100A runs (right). The statistical uncertainties are shown in black, the systematic uncertainties in red.}
\label{fig:IncidentRaw}
\end{figure}

\begin{figure}
\centering  
\includegraphics[width=0.48\textwidth]{Chapter-6/Images/Plots60A_MCData_XS_StatSyst.pdf}
\includegraphics[width=0.48\textwidth]{Chapter-6/Images/Plots100A_MCData_XS_StatSyst.pdf}
\caption{Raw ($\pi^-$-Ar) total hadronic cross section for the 60A runs (left) and for the 100A runs (right). The statistical uncertainties are shown in black, the systematic uncertainties in red. The raw cross section obtained with a MC mixed sample of pions, muon and electrons in the percentage predicted by G4Beamline is shown in azure. }
\label{fig:XSRaw}
\end{figure}


\subsection{Statistical Uncertainty}\label{ch:StatUncertaintyXSRaw}
The statistical uncertainty for each kinetic energy bin of the cross section plot is calculated by error propagation from the statistical uncertainty on $N^{\text{TOT}}_{\text{Inc}}$ and $N^{\text{TOT}}_{\text{Int}}$ correspondent bin.  Since the number of incident hadrons in each energy bin is given by a simple counting, we assume that $N^{\text{TOT}}_{\text{Inc}}$ is distributed as a poissonian with mean and variance equal to $N^{\text{TOT}}_{\text{Inc}}$ in each bin.  
On the other hand, $N^{\text{TOT}}_{\text{Int}}$ follows a binomial distribution: a particle in a given energy bin might or might not interact.  The variance for the binomial is given by  
\begin{equation}
\text{\textsf{Var[}} N^{\text{TOT}}_{\text{Int}} \text{\textsf{]}}
 = \mathcal{N}P_{Interacting}(1-P_{Interacting});
\label{eq:binVar}
\end{equation}

since the interaction probability $P_{Interacting}$ is $\frac{ N^{\text{TOT}}_{\text{Int}}}{N^{\text{TOT}}_{\text{Inc}}}$ and the number of tries $\mathcal{N}$ is $N^{\text{TOT}}_{\text{Inc}}$, equation \ref{eq:binVar} translates into
\begin{equation}
\text{\textsf{Var[}} N^{\text{TOT}}_{\text{Int}} \text{\textsf{]}}
= N^{\text{TOT}}_{\text{Inc}}\frac{ N^{\text{TOT}}_{\text{Int}}}{N^{\text{TOT}}_{\text{Inc}}} (1-\frac{ N^{\text{TOT}}_{\text{Int}}}{N^{\text{TOT}}_{\text{Inc}}}) = N^{\text{TOT}}_{\text{Int}}(1-\frac{ N^{\text{TOT}}_{\text{Int}}}{N^{\text{TOT}}_{\text{Inc}}}). 
\end{equation}

$N^{\text{TOT}}_{\text{Inc}}$ and $N^{\text{TOT}}_{\text{Int}}$ are not independent.
The statistical uncertainty on the cross section is thus calculated as 
\begin{equation}
\delta\sigma_{tot}(E) = \sigma_{tot}(E) \Big(\frac{\delta N^{\text{TOT}}_{\text{Int}}}{N^{\text{TOT}}_{\text{Int}}}+\frac{\delta N^{\text{TOT}}_{\text{Inc}}}{N^{\text{TOT}}_{\text{Inc}}}\Big) 
\end{equation}
where:
\begin{eqnarray}
\delta N^{\text{TOT}}_{\text{Inc}} = \sqrt[]{N^{\text{TOT}}_{\text{Inc}}} \\
\delta N^{\text{TOT}}_{\text{Int}} = \sqrt[]{N^{\text{TOT}}_{\text{Int}}\Big(1-\frac{ N^{\text{TOT}}_{\text{Int}}}{N^{\text{TOT}}_{\text{Inc}}}\Big)}.
\end{eqnarray}



\subsection{Treatment of Systematics} \label{ch:SysUncertaintyXSRaw}
The only systematic effect considered in the measurement of the raw cross section results from the propagation of the uncertainty associate with the measurement of the kinetic energy at each slab.


\section{Corrections to the Raw Cross Section}\label{ch:PionXSCorrections}
As described in section \ref{ch:MCCorrections} as series of corrections are needed to derive the true pion cross section from the raw cross section. 
These corrections are described in equation \ref{eq:C}, 

\begin{equation}
   \sigma^{\pi^-}_{TOT}(E_{i})  = \frac{1}{n \delta X}\frac{ \epsilon^{\text{Inc}}(E_i)  \hspace{0.2cm} C^{\pi MC}_{\text{Int}} (E_{i}) \hspace{0.2cm} N^{\text{TOT}}_{\text{Int}} (E_{i}) }{   \epsilon^{\text{Int}}(E_i) \hspace{0.2cm} C^{\pi MC}_{\text{Inc}} (E_{i}) \hspace{0.2cm}  N^{\text{TOT}}_{\text{Inc}} (E_{i})}.
 \tag{\ref{eq:C}}
\end{equation}



\subsection{Background subtraction}

\begin{figure}[htb]
\centering
\includegraphics[width=\textwidth]{Chapter-6/Images/Bkg60A_inc_int.pdf}
\includegraphics[width=\textwidth]{Chapter-6/Images/Bkg100A_inc_int.pdf}
\caption{.}
\label{fig:BkgCorr}
\end{figure}


\subsubsection{Treatment of Systematics}


\subsection{Efficiency Correction}
\subsubsection{Treatment of Systematics}


\subsection{Final Plots}

\begin{figure}[htb]
\centering
\includegraphics[width=0.48\textwidth]{Chapter-6/Images/TheMoneyPlot60A.pdf}
\includegraphics[width=0.48\textwidth]{Chapter-6/Images/TheMoneyPlot100A.pdf}
\caption{.}
\label{fig:FinalXSPion}
\end{figure}

\begin{figure}[htb]
\centering
\includegraphics[width=0.48\textwidth]{Chapter-6/Images/TheMoneyPlot.pdf}
\caption{.}
\label{fig:FinalXSPion}
\end{figure}

\chapter{Positive Kaon Cross Section Measurement}\label{ch:KaonXS}
\section{Raw Cross Section}\label{ch:KaonXSRaw}
\subsection{Uncertainties }\label{ch:KaonXSRawUnc}


%\chapter{Background subtraction}
\section{Assessing Beamline Contamination}
What is the beamline contamination? We define beamline contamination every TPC track matched to the WC track which is not a primary pion. There are 4 different types of beamline contaminations:
\begin{itemize}
\item[]1) electrons,
\item[]2) muons,
\item[]3) secondaries from pion events,
\item[]4) matched pile up events.
\end{itemize}

So, how do we handle this contamination?

The first step is to estimate what percentage of events used in the cross section calculation is not a primary pion.  
We estimate the percentage of electrons and muons in the beam via the beamline MC\footnote{Since the beamline composition is a function of the magnet settings, we simulate separately events for magnet current of -60A and -100A. 
We calculate the electron to pion and muon to pion ratio on the whole sample as the weighted sum of the corresponding ratio in the two current settings, 
\begin{equation}
\frac{N_e}{N_\pi}_{Data} = w_{60A}\frac{N_e}{N_\pi}_{60A}  + w_{100A}\frac{N_e}{N_\pi}_{100A},
\end{equation}
\begin{equation}
\frac{N_\mu}{N_\pi}_{Data} = w_{60A}\frac{N_\mu}{N_\pi}_{60A}  + w_{100A}\frac{N_\mu}{N_\pi}_{100A},
\end{equation}
where the weights $w_{60A}$ and $w_{100A}$ are the percentage of events in the corresponding magnet configuration passing the mass selection in data. }.
Once the beam composition is know,  we simulate the electrons, muons and pions with the DDMC and we subject the three samples to the same selection chain (WC2TPC match, shower filter, pile up filter, etc...). The percentage of electrons and muons surviving the selection chain is the  electron and muon contamination in the pion cross section sample.
The percentage of secondaries is given in the MC by the number of matched WC2TPC tracks which are not flagged as primary by Geant4.
We estimate the last type of contamination, the ``matched pile up" events, to be a negligible fraction, because of the definition of the WC2TPC match: we deem the probability of a single match with a halo particle in the absence of a beamline particle\footnote{ Events with multiple WC2TPC matches are always rejected.} extremely small.

\section{Subtraction}
Once we estimate the contaminants to primary pion ratio, the next step is subtracting their contribution from data for each type of contaminant independently. The contaminant samples are reconstructed and the corresponding interacting and incident histograms are produced. We then perform a bin by bin subtraction in the data interacting and incident histograms separately. A graphical rendering of this procedure is shown in Fig \ref{fig:backgroundSubtraction}
Once the data is background subtracted, we apply the correction laid out in the previous section.
\textcolor{blue}{How do we account for the error in the contamination subtraction? We change the electron/pion and muon/pion ratio and we see how much difference we get?}

\begin{figure}
\includegraphics[width=\textwidth,height=\textheight,keepaspectratio]{Chapter-9/Images/FakePlot.jpg}
\label{fig:backgroundSubtraction}
\caption{A graphical rendering of the beamline contamination background subtraction. The contribution of the contaminants is shown in green for the secondaries, in orange for the muons and in pink for electrons. The colored plots are coming from the MC and are staggered. The percentages shown in the legend are the percentages of contaminants over the total number of events  passing the selection chain. We actually expect way less contamination.}
\end{figure}




%\chapter{Negative Pion Cross Section Measurement}

\section{Estimate of $E_{loss}$ before the TPC}
\section{Interacting and Incident Distributions}

\section{Total Hadronic Negative Pion-Argon Differential Cross Section}

\chapter{Uncertainty budget}\label{ch:Uncertainty}
Measuring an hadronic cross section  in LArIAT translates into counting how many hadrons impinged on a slab of argon at a given energy and how many of those hadrons interacted at said energy. So, the key questions here are:
\begin{itemize}
\item[]a) how well do we know the kinetic energy at each point of the tracking? %(Incident Kinetic Energy bins)
\item[]b) how well do we know when the tracking stops? %(Interacting Kinetic Energy bin)
\item[]c) are there any systematic shifts?
\end{itemize}

In order to answer this question, will discuss first a simple scenario  were our beam is 100\% made of pions which arrive as primaries in the TPC (no decay in the beam and no inelastic interaction before the TPC front face). We will then add a layer of complexity by discussing how we handle beamline contamination.

\section{Pure beam of pions}
Assuming a beam of pure pions gets to the TPC, let us explicit some of the variables in the kinetic energy equation \ref{eq:KEj}  to point out the important quantities in the uncertainty budget,

\begin{align}
 E_{j}^{kin} &=  E_{Beam}^{kin}  - E_{loss} - \sum_{i < j} \frac {dE_i}{dx_i}*dx_i\\
                  &=  \sqrt{p^2_{Beam} - m^2_{Beam}} - m_{Beam} - E_{loss} - \sum_{i < j} \frac {dE_i}{dx_i}*dx_i.
\end{align}

\subsection{Uncertainty on $E_{Beam}^{kin}$}
Let us start by discussing the uncertainty on $E_{Beam}^{kin}$. Since we are assuming a beam of pions, the uncertainty on the value of mass of the pion ($m_{Beam}$) as given by the pdg is irrelevant compared to the momentum uncertainties, thus $\delta E_{Beam}^{kin} = \delta p_{Beam}^{kin}$. 
We estimate the momentum uncertainty as follows.

\textcolor{blue}{  
We estimate the uncertainty on a 4-point track. In case of 3-points track, we add an additional 2\% coming from Greg's study. 
Uncertainty on a 4-point track:
\begin{itemize}
\item[-]  Alignment surveys. 1mm misalignment translates to 3\% in overall
\item[-] Doug study dp/p = ~2\% based on field map (docdb 1710)
\item[-] Minerva test beam paper
\end{itemize}
}

\subsection{Systematics on $E_{loss}$}




\textbf{Systematics}
Discrepancies between the real TPC geometry and the simulated geometry can lead to a systematic in the $E_{loss}$ calculation. In particular, we found a difference in the depth of the un-instrumented argon upstream to the TPC front face, the MC geometry reporting $~\sim 3.3$ cm more un-instrumented argon than the TPC survey. For a pion MIP, this depth corresponds to 7.4 MeV which we account for as a double sided systematic in the determination of the pion kinetic energy.

%\textcolor{blue}{ TO DO HERE: make sure we have the geometry right, cause otherwise this correction is meaningless.  With this method, so far we get a mean ~40 MeV, but uncertainty ~7MeV. 
%The trajectory method does not improve uncertainty, why? It's a mystery I don't think we should solve before June :) .
%Back of the envelope material budget calculation:}
%\begin{table}[h!]
%\centering
%\caption{Back of the envelope calculation}
%\label{my-label}
%\begin{tabular}{|l|l|l|l|}
%\hline
%dEdx for MIP, MPV {[}MeV cm$^2$/gr{]} & density {[}g/cm$^3${]} & width {[}cm{]} & E$_{loss}$ {[}MeV{]} \\ \hline
%1.6                                                 & 1.7 (G10)                            & 1.3            & 3.5                   \\
%1.6                                                 & 1.4 (LAr)                            & 1.77           & 4.0                   \\
%1.6                                                 & 7.7 (S.S.)                           & 0.23           & 2.8                   \\
%1.6                                                 & 4.5 (Ti)                             & 0.04           & 0.3                   \\ 
%1.6                                                 & 1.03 (Plastic Sci)                   & 1.1            & 1.8                   \\ \hline
%Total                                               &                                      &                & 12.4                  \\ \hline
%\end{tabular}
%\end{table}



\subsection{Uncertainty on dE/dx and pitch}
We obtain the uncertainty on dE/dx and track pitch by comparing the dE/dx and pitch distributions in data and MC.
\textcolor{blue}{ Currently, MPV MC = 1.70 and MPV DATA = 1.72 MeV/cm (~3\% higher).
TO DO HERE: calculate Argon density from mid-RTD temperature. Compare this  density with MC Argon density. 
Density change  affects dE/dx (in MeV/cm!). Try changing MC density up to ``real one" and see if dEdX agrees between DATA and MC}


\subsection{Uncertainty on track end, aka efficiency correction}
From the MC, we obtain an efficiency correction on the interacting and incident distributions separately. This is done by comparing the MC reconstructed with the true MC deposition on an event by event basis.
This correction is applied bin by bin on the data interacting and incident distributions.
The better our tracking, the smaller this efficiency correction will be. So, step number one is improving the tracking.
\textcolor{blue}{Need to talk to Bruce about this.}
\textcolor{blue}{ I don't understand the angle cut that Dave Schmitz and Jon Paley were so vocal about.}

Now, the key question remains: does the tracking behave in the same way in data and MC? 
We can compare some key plots between reconstructed data and MC which gives us confidence this is true: the track pitch, the tracks straightness and the goodness of fit in data and MC. \textcolor{blue}{ Does such a variable as ``goodness of fit" exists in the tracking? We should ask Bruce.}



% Only call appendix once, here.
\appendix
%%\chapter{Kaon Analysis}
\section{Data Sample}
\section{Beamline Contamination}
\section{WC2TPC match}
\section{Cross Section}
\section{Future developments}

\chapter{Tracking Studies}\label{ch:AppendixTrack}
In this section, we describe three studies. The first is a justification of the selection criteria for the beamline handshake with the TPC information.  We perform this study to boost  the correct identification of the particles in the TPC associated with the beamline information, while maintaining sufficient statistics for the cross section measurement.  The second study is an optimization of the tracking algorithm, with the scope of maximizing the identification of the hadronic interaction point inside the TPC.  These two studies are related, since the optimization of the tracking is performed on TPC tracks which have been matched to the wire chamber track; in turn, the tracking algorithm for TPC tracks determines the number of reconstructed tracks in each event used to try the matching with the wire chamber track. Starting with a sensible tracking reconstruction, we perform the WC2TPC matching optimization first, then the tracking optimization. The WC2TPC match purity and efficiency  are then calculated again with the optimized tracking.


\subsection{Study of WC to TPC Match}\label{ch:WC2TPCMatchOptimization}

Plots I want in this section:
\begin{enumerate}
\item WC2TPC MC DeltaX, DeltaY and $\alpha$
\end{enumerate}


Scope of this study is assessing the goodness of the wire chamber to TPC match on Monte Carlo and decide the selection values we will use on data. A word of caution is necessary here. With this study, we want to minimize pathologies associated with the presence of the primary hadron itself, e.g. the incorrect association between the beamline hadron and its decay products inside the TPC.  Assessing the contamination from pile-up\footnote{We remind the reader that the DDMC is a single particle Monte Carlo, where the beam pile up is not simulated.}, albeit related, is beyond the scope of this study.

In MC, we are able to define a correct WC2TPC match using the Geant4 truth information. We are thus able to count how many times the WC tracks is associated with the wrong TPC reconstructed track. 

We define a correct match if the all following conditions are met:
\begin{itemize}
\item[-] the length of the true primary Geant4 track in the TPC is greater than 2 cm,  
\item[-] the length of the reconstructed track length is greater than 2 cm,
\item[-] the Z position of the first reconstructed point is within 2 cm from the TPC front face
\item[-] the distance between the reconstructed track and the true entering point is the minimum compared with all the other reconstructed tracks.
\end{itemize}

In order to count the wrong matches, we consider all the reconstructed tracks whose Z position of the first reconstructed point lies within 2 cm from the TPC front face. Events with true length in TPC $<$ 2 cm are included. 
Since hadrons are shot 100 cm upstream from the TPC front face, the following two scenarios are possible from a truth standpoint: 
\begin{itemize}
\item[[$Ta$]] the primary hadron decays or interact strongly before getting to the TPC,
\item[[$Tb$]] the primary hadron enters the TPC.
\end{itemize}

As described in Section \ref{ch:WC2TPCMatchMethod}, we define a WC2TPC match according to the relative position of the WC and TPC track parametrized with $\Delta R$ and the angle between them, parametrized with $\alpha$. Once we choose the selection values $r_{T}$ and $\alpha_{T}$ to determine a reconstructed WC2TPC match, the following five scenarios are possible in the truth to reconstruction interplay : 
\begin{itemize}
\item[1)] only the correct track is matched
\item[2)] only one wrong track is matched 
\item[3)] the correct track and one (or more) wrong tracks are matched
\item[4)] multiple wrong tracks  matched.
\item[5)] no reconstructed tracks are matched
\end{itemize}

Since we keep only events with one and only one match, we discard cases 3), 4) and 5) from the events used in the cross section measurement. For each set of $r_{T}$ and $\alpha_{T}$ selection value, we define purity and efficiency of the selection as follows:
\begin{equation}
\text{Efficiency} = \frac{\text{Number of events correctly matched}}{\text{ Number of events with primary in TPC}},
\end{equation}

\begin{equation}
\text{Purity} = \frac{\text{Number of events correctly matched}}{\text{Total number of matched events}}.
\end{equation}

Figure \ref{fig:EffPurityK} shows the efficiency (left) and purity (right) for WC2TPC match as a function of the radius, $r_{T}$, and angle, $\alpha_{T}$, selection value. It is apparent how both efficiency and purity are fairly flat as a function of the radius selection value at a given angle. This is not surprising. Since we are studying a single particle gun Monte Carlo sample, the wrong matches can occur only for mis-tracking of the primary or for association with decay products;  decay products will tend to be produced at large angles compared to the primary, but could be fairly close to the in $x$ and $y$ projection of the primary. The radius cut would play a key role in removing pile up events. 

For LArIAT cross section measurements, we generally prefer purity over efficiency, since a sample of particles of a pure species will lead to a better measurement. Obviously, purity should be balanced with a sensible efficiency to avoid rejecting the whole sample. 

We choose $(\alpha_{T}$, $r_{T}) = (8 \text{ deg}, 4 \text{ cm} )$ and get a MC 85\% efficiency and 98\% purity for the kaon sample and a MC 95\% efficiency and 90\% purity for the pion sample.


\begin{figure}[hpbt]
\centering
\includegraphics[width=15cm]{Chapter-5/Images/KEffPurity.png}
\caption{Efficiency (left) and purity (right) for WC2TPC match as a function of the radius and angle selections for the kaon sample.}
\label{fig:EffPurityK}
\end{figure}




\subsection{Tracking Optimization}\label{ch:TrackOptimization}

\chapter{Energy Calibration}\label{ch:energyCalibration}

Scope of the energy calibration is to identify the factors which convert the charge collected (dQ) to energy deposited in the chamber (dE). As described in section \ref{sec:SignalProc}, this is a multi-step procedure. In LArIAT, we first correct the raw charge by the electronic noise on the considered wire \cite{LArIATdqdx}, then by the electron lifetime \cite{LArIATLifeTime},  and then by the recombination using the ArgoNeut recombination values. Lastly, we apply overall calibration of the energy, i.e. we determine the ``calorimetry constants" using the procedure described in this section.


We independently determine  the calorimetry constants for Data and Monte Carlo in the LArIAT Run-II Data samples using  a parametrization of the stopping power (a.k.a. energy deposited per unit length, $dE/dX$)  as a function of momentum. This is done by comparing the stopping power measured on reconstructed quantities against the Bethe-Bloch theoretical prediction for various particle species (see Equation \ref{eq:BB}).  We obtain the theoretical expectation for the $dE/dX$ most probable value of pions ($\pi$), muons ($\mu$), kaons ($K$), and protons ($p$) in the momentum range most relevant for LArIAT (Figure \ref{fig:PDGEnergyLossArgon}) using the tables provided by the Particle Data Group \cite{Patrignani:2016xqp} for liquid argon \cite{PDG-Argon}.

The basic idea of this calibration technique is to utilize a sample of beamline events with known particle species and momentum to measure the $dE/dX$ of the corresponding tracks in the TPC. In particular, we decided to use positive pions as calibration sample and samples from all the other particle species as cross check. Once the $dE/dX$ of the positive pion sample  has been measured at various momenta, we tune to calorimetry constants within the reconstruction software to align the measured values to match the theoretical ones found in Figure \ref{fig:PDGEnergyLossArgon}. 

In data, we start by selecting a sample of beamline positive pion beamline candidates without any restriction on their measured momentum\footnote{it should be noted that some muon and position contamination is present in the $\pi^+$ sample}.
We then apply the WC2TPC match and subtract the energy loss upstream to the TPC front face, determining the momentum at the TPC front face. For each surviving pion candidate,  we measure the $dE/dx$ at each of the first 12 spacepoints associated the 3D reconstructed track, corresponding to a $\sim$ 5 cm portion. These $dE/dX$ measurements are then put into a histogram that corresponds to measured momentum of the track. The $dE/dX$ histograms are sampled every 50 MeV/c in momentum (e.g. 150~MeV/c $< P <$ 200~MeV/c, 200~MeV/c $< P <$ 250/c~MeV, etc...).   This process of selecting, sampling, and recording the $dE/dX$ for various momentum bins is repeated over the entire sample of events, allowing us to collect sufficient statistic in most of the momentum bins between 150~MeV/c and 1100~MeV/c. On average, pions and muons only lose $\sim$10 MeV in this 5~cm section of the track and protons lose $\sim$20 MeV. Thus choosing 50 MeV/c size bins for our histograms covers the energy spread within those bins due to energy loss from ionization for all the particle species identifiable in the beamline. 
Each 50 MeV/c momentum binned $dE/dX$ histogram is now fit with a simple Landau function. The most probable value (MPV) and the associated error on the MPV from the fit are extracted and plotted against the theoretical prediction Figure \ref{fig:PDGEnergyLossArgon}. Depending on the outcome of the data-prediction comparison, we modify the calorimetry constants and we repeat the procedure until a qualitative agreement is achieved.  We perform this  tuning for the collection and induction plane separately. 
As a cross check to the calorimetry constants determined using the positive pions, we lock the constants and  plot the $dE/dx$ versus momentum distribution of all the other particle species identifiable in the beamline data ($\pi/\mu/e$, K , p, in both polarities) against the corresponding Beth-Bloch prediction. The agreement between data from the other particle species and the predictions is the expected result of this cross check.
The results of the tuning and cross check for Run-II data on the collection plane is shown in Figure \ref{fig:BBandData}  negative polarity data on top, positive polarity data on the bottom.

In MC, we simulate the corresponding positive pion sample with the DDMC (see section \ref{sec:DDMC}) and follow the same steps as in data. More details on the calorimetry tuning can be found in \cite{LArIATCalo}.

After the calibration is done separately on data and MC, we can compare the resulting dE/dX distributions; this is done for the set of pion beamline candidates and pion MC on the left side of Figure \ref{fig:dedx}. On the right side of the same figure, we report the data-MC comparison for the kaon sample used in the cross section. The distributions are fitted with simple Landau functions. As expected, the Landau MPV for data  is consistent with the MC MPV for both pions and kaons. For both the pion and the kaon case, the width of the Landau is a bit wider in data than in MC; this difference might be due to an underestimate of the electronic noise in the MC simulation.


\begin{figure}[htb]
\centering
\includegraphics[width=0.50\textwidth]{Chapter-5/Images/dEdXvsMomentumTemplate.png}
\caption{Stopping power for pions, muons, kaons, and protons in liquid argon over the momentum range most relvant for LArIAT according to the Beth-Bloch equation. The solid lines represent the prediction for the mean energy $dE/dX$, while the dashed lines are the predictions for the MPV.}
\label{fig:PDGEnergyLossArgon}
\end{figure}


\begin{figure}[htb]
\centering
\includegraphics[width=0.48\textwidth]{Chapter-5/Images/RunIINegTotaldEdXvsMomentum.png}
\includegraphics[width=0.48\textwidth]{Chapter-5/Images/RunIIPosTotaldEdXvsMomentum.png}
\caption{Stopping power versus Momentum for Run-II negative (top) and positive (bottom) polarity data. We achieve the agreement between the Bethe-Bloch predictions and the distribution obtained with of the positive pions (top plot, red dots) by tuning the calorimetry constants. Once the calorimetry constants are locked in, the agreement between the other particle species and the Bethe-Bloch predictions follows naturally.}
\label{fig:BBandData}
\end{figure}


\begin{figure}[htb]
\centering
\includegraphics[width=0.48\textwidth]{AppendixC-EnergyCalibration/dEdXPions.pdf}
\includegraphics[width=0.48\textwidth]{AppendixC-EnergyCalibration/dEdXKaons.pdf}
\caption{\emph{Left:} dE/dx distribution for $\pi^-/\mu^-/e^-$ data (black) and Pion MC (blue). The Landau fit for data is shown in red, the one for MC in teal.\emph{Right:} dE/dx distribution for $K^+$ data (black) and Kaon MC (blue). The Landau fit for data is shown in red, the one for MC in teal. All the distributions are area normalized.}
\label{fig:dedx}
\end{figure}


%\begin{figure}[htb]
%\centering
%\includegraphics[width=0.50\textwidth]{images/CalibrationExample.png}
%\caption{Illustration of the calibration technique. Here we depict a 325 MeV wire chamber track (shown in green) which enters the TPC (taking into account the energy loss from the upstream material) and we sample the first 12 spacepoints (shown in teal) to extract the dE/dX distribution which is fit with a Landau.}
%\label{fig:CalibrationExample}
%\end{figure}




%\chapter{Measurement of LArIAT Electric Field}\label{ch:AppendixB}
The electric field of a LArTPC in the drift volume is a fundamental quantity for the proper functionality of this technology, as it affects almost every reconstructed quantity such as the position of hits or their collected charge. Given its importance, we calculate the electric field for LArIAT with a single line diagram from our HV circuit and we cross check the obtained value with a measurement relying only on TPC data. 

Before getting into the details of the measurement procedures, it is important to explicit the relationship between some  quantities in play. The electric field and the drift velocity ($v_{drift}$) are related as follows 
\begin{equation} v_{drift} = \mu(E_{field},T) E_{field}, \label{eq:vd}
\end{equation}
where $\mu$ is the electron mobility, which depends on the electric field and on the temperature (T). The empirical formula for this dependency is described in ~\cite{WWW} and shown in Figure \ref{fig:EV} for several argon temperatures.

\begin{figure}[htb]
\centering
\includegraphics[scale=0.45]{./AppendixB-EField/Images/Walkowiak.png}\\
\caption{Drift velocity dependence on electric field for several temperatures. The slope of the line at any one point represents the electron mobility for that given temperature and electric field.}
\label{fig:EV}
\end{figure}



The relationship between the drift time ($t_{drift}$) and the drift velocity is trivially given by
\begin{equation}
t_{drift} = \Delta x/v_{drift}, \label{eq:drifttime}
\end{equation}
where $\Delta x$ is the distance between the edges of the drift region.
Table \ref{tab:Efields} reports the values of the electric field, drift velocity, and drift times for the smaller drift volumes. 

\begin{table}[]
\centering
\caption{Electric field and drift velocities in LArIAT smaller drift volumes}
\label{tab:Efields}
\begin{tabular}{|l|l|l|}
\hline
& Shield-Induction & Induction-Collection \\ \hline
E$_{field}$ &                 700.63 V/cm        &                892.5  V/cm             \\ \hline
v$_{drift}$ &                   1.73  mm/$\mu$s   &                  1.90 mm/$\mu$s        \\ \hline
t$_{drift}$ &                   2.31  $\mu$s      &                   2.11 $\mu$s          \\ \hline

\end{tabular}
\end{table}

With these basic parameters established, we can now move on to calculating the electric field in the main drift region (between the cathode and the shield plane).

\subsection*{Single line diagram method}
The electric field strength in the LArIAT main drift volume can be determined knowing the voltage applied to the cathode, the voltage applied at the shield plane, and the distance between them. We assume the distance between the cathode and the shield plane to be 470 mm and any length contraction due to the liquid argon is negligibly small ($\sim$2~mm).

The voltage applied to the cathode can be calculated using Ohm's law and the single line diagram shown in Figure \ref{fig:circuit}.  A set of two of filter pots for emergency power dissipation are positioned between the Glassman power supply and the cathode, one at each end of the feeder cable, each with an internal resistance of 40~M$\Omega$. 


Given the TPC resistor chain, the total TPC impedance is ~6 G$\Omega$. Since the total resistance on the circuit is driven by the TPC impedance, we expect the resulting current to be 
\begin{equation}
I=V_{PS} /R_{tot} = -23.5\text{ kV}/6 \text{ G}\Omega  \sim 4 \text{ $\mu$A}, 
\end{equation}

which we measure with the Glassman power supply, shown in  Figure \ref{fig:currentMeasurement}.  

\begin{figure}[h]
\centering
\begin{minipage}{0.45\textwidth}
\centering
\includegraphics[width=3in]{AppendixB-EField/Images/CircuitLArIAT.png}
\caption{LArIAT HV simple schematics.}
\label{fig:circuit}
\end{minipage}\hfill
\begin{minipage}{0.45\textwidth}
\centering
\includegraphics[width=3in]{AppendixB-EField/Images/glassman_current_20160525-30.png}
\caption{Current reading from the Glassman between May 25th and May 30th, 2016 (typical Run-II conditions).}
\label{fig:currentMeasurement}
\end{minipage}
\end{figure}

Using this current, the voltage at the cathode is calculated as
\begin{equation} \label{eq:VBC}
V_{BC}=V_{PS} - (I \times R_{eq}) = -23.5\text{ kV} + ( 0.00417\text{ mA} \times 80\text{ M}\Omega ) = -23.17\text{ kV}, 
\end{equation}
where $I$ is the current and $R_{eq}$ is the equivalent resistor representing the two filter pots. The electric field is then calculated to be
\begin{equation}E_{\text{field}} = \frac{V_{BC} - V_{\text{shield}}}{\Delta x} = 486.54\text{ V/cm}.
\end{equation}
%\begin{equation}v_{drift} = \mu E_{field} = 1.5097 \textit{ mm/$\mu$s}
%\end{equation}
%\begin{equation}t_{drift} = \frac{\Delta x}{v_{drift}} = 311.316 \textit{ $\mu$s.}
%\end{equation}



\subsection*{E field using cathode-anode piercing tracks}
%%%%%%%%%%%%%%%%%%%%%%%%%%%%%%%%%%%%%%%%%%%%%%%%%%%%%%%%%%%%%%%%%%%%%%%%%%
\begin{figure}[b]
\centering
\begin{minipage}{0.45\textwidth}
\centering
\includegraphics[width=3in]{AppendixB-EField/Images/TPCCrossSectionView.png}
\caption{Pictorial representation of the YX view of the TPC. The distance within the anode planes and between the shield plane and the cathode is purposely out of proportion to illustrate the time difference between hits on collection and induction. An ACP track is shown as an example.}
\label{fig:Scheme}
\end{minipage}\hfill
\begin{minipage}{0.45\textwidth}
\centering
\includegraphics[width=3in]{AppendixB-EField/Images/AngleDef.png}
\caption{Angle definition in the context of LArIAT coordinate system.}
\label{fig:AngleDef}
\end{minipage}
\end{figure}
We devise an independent method to measure the drift time (and consequently drift velocity and electric field) using TPC cathode to anode piercing tracks. We use this method as a cross check to the single line method.
The basic idea is simple:
\begin{itemize}
\item[0.] Select cosmic ray events with only 1 reconstructed track 
\item[1.] Reduce the events to the one containing tracks that cross both anode and cathode
\item[2.] Identify the first and last hit of the track
\item[3.] Measure the time difference between these two hits ($\Delta t$).
\end{itemize}
This method works under the assumptions that the time it takes for a cosmic particle to cross the chamber ($\sim$ns) is small compared to the charge drift time ($\sim$ hundreds of $\mu$s).

We choose cosmic events to allow for a high number of anode to cathode piercing tracks (ACP tracks), rejecting  beam  events where the particles travel almost perpendicularly to drift direction. We select events with only one reconstructed track to  maximize the chance of selecting a single crossing muon (no-michel electron). We utilize ACP tracks because their hits span the full drift length of the TPC, see figure \ref{fig:Scheme}, allowing us to define where the first and last hit of the tracks are located in space regardless of our assumption of the electric field. %The definition of the last hit is easy: it is the hit closest to the cathode. The definition of the first hit is a bit more complicated. A track that crosses the anode planes will deposit charge in the small drift volumes (S-I and I-C). The drift time in S-I and I-C region was already calculated in Table \ref{tab:Efields}. This is to say that the position of the first hit matters when calculating the drift time at the microsecond precision. Single hits on the collection plan will not form a 3D object. This means that we can safely exclude that the reconstruction of ACP tracks starts at the cathode plane (black point in figure \ref{fig:Scheme}). %Understanding if the first hit of a track in on the induction or on the shield plan is more complicated. For now, we'll take an uncertainty hit of 2.3 $\mu$s.


One of the main features of this method is that it doesn't rely on the measurement of the trigger time. Since $\Delta t$ is the time difference between the first and last hit of a track and we assume the charge started drifting at the same time for both hits, the measurement of the absolute beginning of drift time $t_0$ is unnecessary. We boost the presence of ACP tracks in the cosmic sample by imposing the following requirements on tracks:

\begin{itemize}
\item vertical position (Y) of first and last hits within $\pm$ 18 cm from TPC center (avoid Top-Bottom tracks) 
\item horizontal position (Z) of first and last hits within 2 and 86 cm from TPC front face (avoid through going tracks) 
\item track length greater than 48 cm (more likely to be crossing)
\item angle from the drift direction (phi in figure \ref{fig:AngleDef}) smaller than 50 deg (more reliable tracking)
\item angle from the beam direction (theta in figure \ref{fig:AngleDef}) greater than 50 deg (more reliable tracking)
\end{itemize}


Tracks passing all these selection requirements are used for the $\Delta t$ calculation.

For each track passing our selection, we loop through the associated hits to retrieve the timing information. The analysis is performed separately on hits on the collection plane and induction plane, but lead to consistent results. As an example of the time difference, figures \ref{fig:Run2PosColFit} and \ref{fig:Run2PosIndFit} represent the difference in time between the last and first hit of the selected tracks for Run-II Positive Polarity sample on the collection and induction plane respectively.  We fit with a Gaussian to the peak of the $\Delta t$ distributions to extract the mean drift time and the uncertainty associated with it. The long tail at low $\Delta t$ represents contamination of non-ACP tracks in the track selection.  We apply the same procedure to Run-I and Run-II, positive and negative polarity alike.

   
\begin{figure}[h!]
\begin{minipage}{0.40\textwidth}
\centering
\includegraphics[width=3in]{AppendixB-EField/Images/RunIIPosCol.png}
\caption{Collection plane $\Delta$t fit for Run II positive polarity ACP data  selected tracks.}
\label{fig:Run2PosColFit}
\end{minipage}\hfill
\begin{minipage}{0.40\textwidth}
\centering
\includegraphics[width=3in]{AppendixB-EField/Images/RunIIPosInd.png}
\caption{Induction plane $\Delta$t fit for Run II positive polarity ACP data  selected tracks.}
\label{fig:Run2PosIndFit}
\end{minipage}
\end{figure}

To convert $\Delta t$ recorded for the hits on the induction plane to the drift time we employ the formula
\begin{equation}
t_{drift} = \Delta t - t_{S-I}
\end{equation}
where $t_{drift}$ is the time the charge takes to drift in the main volume between the cathode and the shield plane and $t_{S-I}$ is the time it takes for the charge to drift from the shield plane to the induction plane. In Table \ref{tab:Efields} we calculated the drift velocity in the S-I region, thus we can calculate $t_{S-I}$ as 
\begin{equation}
t_{S-I} = \frac{l_{S-I}}{v_{S-I}} = \frac{4 mm}{1.73 mm/ \mu s}
\end{equation}
where $\l_{S-I}$ is the distance between the shield and induction plane and $v_{S-I}$ is the drift velocity in the same region. A completely analogous procedure is followed for the hits on the collection plane, taking into account the time the charge spent in drifting from shield to induction as well as between the induction and collection plane
The value for $\Delta t_{drift}$ , the calculated drift velocity ($v_{drift}$), and corresponding drift electric field for the various run periods is given in Table \ref{tab:deltaTACP} and are consistent with the electric field value calculated with the single line diagram method.

\begin{center}
\begin{table}[htb]
  \begin{center}
    %\resizebox{0.45\textwidth}{!}{%
    \begin{tabular}{|c|c|c|c|}
      \multicolumn{4}{c}{\textbf{Delta t$_{drift}$, drift v and E field with ACP tracks}} \\
      \hline \hline
       Data Period  & $\Delta t_{Drift}$ [$\mu s$] & Drift velocity [mm/$\mu$s] & E field [V/cm] \\
       \hline
       RunI Positive Polarity Induction &  311.1 $\pm$ 2.4   &1.51 $\pm$ 0.01  & 486.6 $\pm$ 21\\
       \hline
       RunI Positive Polarity Collection &  310.9 $\pm$ 2.6 & 1.51 $\pm$ 0.01  &  487.2 $\pm$ 21\\
       \hline
       RunII Positive Polarity Induction &   315.7 $\pm$ 2.8 & 1.49 $\pm$ 0.01 &  467.9 $\pm$ 21\\
       \hline
       RunII Positive Polarity Collection &  315.7 $\pm$ 2.7 & 1.49 $\pm$ 0.01 &  467.9 $\pm$ 21\\
       \hline
       RunII Negative Polarity Induction &   315.9 $\pm$ 2.6 & 1.49 $\pm$ 0.01  & 467.1 $\pm$ 21 \\
       \hline
       RunII Negative Polarity Collection &  315.1 $\pm$ 2.8 & 1.49 $\pm$ 0.01  & 470.3 $\pm$ 21  \\
       \hline
       \hline
       Average Values & 314.1 & 1.50 $\pm$ 0.01 & 474.3 $\pm$ 21 \\
       \hline
       \end{tabular}
    \caption{$\Delta t$ for the different data samples used for the Anode-Cathode Piercing tracks study. }
    \label{tab:deltaTACP}
    \end{center}
\end{table}
\end{center}

%%%%%%%%%%%%%%%%%%%%%%%%%%%%%%%%%%%%%%%%%%%%%%%%%%%%%%%%%%%%




%%\input{Chapters/C-MicroBooNECRT}


% Old chapters configuration %
%\chapter{Data Collection}
Your first chapter is probably an introduction. But who knows.
%\chapter{LArIAT Monte Carlo}
\section{Beamline}\label{beamlineComposition}
\subsection{G4Beamline}
\subsection{Data Driven MC}\label{ch:DDMC}
\label{sec:DDMC}
\section{TPC MC}

%\chapter{Energy Calibration}
Your first chapter is probably an introduction. But who knows.
%\input{Chapter-8/8-TrackingOptimization}


% Add additional \chapter{}s as necessary.

% use \cite{} to cite a reference in your bibliography file.
% use \ref{} to reference a \label{} from an equation, figure, or table.

% for sets of equations use align:
%\begin{align}
%\end{align}

% for figures:
%\begin{figure}[ht]
%\centering
%\includegraphics[width=.45\textwidth]{name_of_figure.eps}
%\caption{A caption! \label{a_figure}}
%\end{figure}

% for tables:
%\begin{table}
%\begin{tabular}{c|c|c}
% 1 & 2 & 3 \\
%\hline
%\end{tabular}
%\caption{Another caption! \label{a_table}}
%\end{table}

% Any chapters such as End Notes go after this.
\backmatter

\bibliography{bib.bib}
%\bibliographystyle{plain}
% for your own sake, use a bibtex file, so all of the numbering of references will be done
% automatically.

\end{document}
