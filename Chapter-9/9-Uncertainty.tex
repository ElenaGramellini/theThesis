\chapter{Uncertainty budget}\label{ch:Uncertainty}
Measuring an hadronic cross section  in LArIAT translates into counting how many hadrons impinged on a slab of argon at a given energy and how many of those hadrons interacted at said energy. So, the key questions here are:
\begin{itemize}
\item[]a) how well do we know the kinetic energy at each point of the tracking? %(Incident Kinetic Energy bins)
\item[]b) how well do we know when the tracking stops? %(Interacting Kinetic Energy bin)
\item[]c) are there any systematic shifts?
\end{itemize}

In order to answer this question, will discuss first a simple scenario  were our beam is 100\% made of pions which arrive as primaries in the TPC (no decay in the beam and no inelastic interaction before the TPC front face). We will then add a layer of complexity by discussing how we handle beamline contamination.

\section{Pure beam of pions}
Assuming a beam of pure pions gets to the TPC, let us explicit some of the variables in the kinetic energy equation \ref{eq:KEj}  to point out the important quantities in the uncertainty budget,

\begin{align}
 E_{j}^{kin} &=  E_{Beam}^{kin}  - E_{loss} - \sum_{i < j} \frac {dE_i}{dx_i}*dx_i\\
                  &=  \sqrt{p^2_{Beam} - m^2_{Beam}} - m_{Beam} - E_{loss} - \sum_{i < j} \frac {dE_i}{dx_i}*dx_i.
\end{align}

\subsection{Uncertainty on $E_{Beam}^{kin}$}
Let us start by discussing the uncertainty on $E_{Beam}^{kin}$. Since we are assuming a beam of pions, the uncertainty on the value of mass of the pion ($m_{Beam}$) as given by the pdg is irrelevant compared to the momentum uncertainties, thus $\delta E_{Beam}^{kin} = \delta p_{Beam}^{kin}$. 
We estimate the momentum uncertainty as follows.

\textcolor{blue}{  
We estimate the uncertainty on a 4-point track. In case of 3-points track, we add an additional 2\% coming from Greg's study. 
Uncertainty on a 4-point track:
\begin{itemize}
\item[-]  Alignment surveys. 1mm misalignment translates to 3\% in overall
\item[-] Doug study dp/p = ~2\% based on field map (docdb 1710)
\item[-] Minerva test beam paper
\end{itemize}
}

\subsection{Systematics on $E_{loss}$}




\textbf{Systematics}
Discrepancies between the real TPC geometry and the simulated geometry can lead to a systematic in the $E_{loss}$ calculation. In particular, we found a difference in the depth of the un-instrumented argon upstream to the TPC front face, the MC geometry reporting $~\sim 3.3$ cm more un-instrumented argon than the TPC survey. For a pion MIP, this depth corresponds to 7.4 MeV which we account for as a double sided systematic in the determination of the pion kinetic energy.

%\textcolor{blue}{ TO DO HERE: make sure we have the geometry right, cause otherwise this correction is meaningless.  With this method, so far we get a mean ~40 MeV, but uncertainty ~7MeV. 
%The trajectory method does not improve uncertainty, why? It's a mystery I don't think we should solve before June :) .
%Back of the envelope material budget calculation:}
%\begin{table}[h!]
%\centering
%\caption{Back of the envelope calculation}
%\label{my-label}
%\begin{tabular}{|l|l|l|l|}
%\hline
%dEdx for MIP, MPV {[}MeV cm$^2$/gr{]} & density {[}g/cm$^3${]} & width {[}cm{]} & E$_{loss}$ {[}MeV{]} \\ \hline
%1.6                                                 & 1.7 (G10)                            & 1.3            & 3.5                   \\
%1.6                                                 & 1.4 (LAr)                            & 1.77           & 4.0                   \\
%1.6                                                 & 7.7 (S.S.)                           & 0.23           & 2.8                   \\
%1.6                                                 & 4.5 (Ti)                             & 0.04           & 0.3                   \\ 
%1.6                                                 & 1.03 (Plastic Sci)                   & 1.1            & 1.8                   \\ \hline
%Total                                               &                                      &                & 12.4                  \\ \hline
%\end{tabular}
%\end{table}



\subsection{Uncertainty on dE/dx and pitch}
We obtain the uncertainty on dE/dx and track pitch by comparing the dE/dx and pitch distributions in data and MC.
\textcolor{blue}{ Currently, MPV MC = 1.70 and MPV DATA = 1.72 MeV/cm (~3\% higher).
TO DO HERE: calculate Argon density from mid-RTD temperature. Compare this  density with MC Argon density. 
Density change  affects dE/dx (in MeV/cm!). Try changing MC density up to ``real one" and see if dEdX agrees between DATA and MC}


\subsection{Uncertainty on track end, aka efficiency correction}
From the MC, we obtain an efficiency correction on the interacting and incident distributions separately. This is done by comparing the MC reconstructed with the true MC deposition on an event by event basis.
This correction is applied bin by bin on the data interacting and incident distributions.
The better our tracking, the smaller this efficiency correction will be. So, step number one is improving the tracking.
\textcolor{blue}{Need to talk to Bruce about this.}
\textcolor{blue}{ I don't understand the angle cut that Dave Schmitz and Jon Paley were so vocal about.}

Now, the key question remains: does the tracking behave in the same way in data and MC? 
We can compare some key plots between reconstructed data and MC which gives us confidence this is true: the track pitch, the tracks straightness and the goodness of fit in data and MC. \textcolor{blue}{ Does such a variable as ``goodness of fit" exists in the tracking? We should ask Bruce.}
