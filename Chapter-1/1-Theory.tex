\chapter{The theoretical framework}

\section{The Standard Model}
The Standard Model (SM) of particle physics the most accurate theoretical description of the subatomic world and, more generically, one of the most precisely tested theories in the history of physics.  The SM describes the strong, electromagnetic and weak interactions among  elementary particles in the framework of quantum field theory, accounting for the unification of electromagnetic and weak interactions for energies above the  vacuum expectation value of the Higgs field. The SM does not describe gravity or general relativity.

The Standard Model is a gauge theory based on the local group of symmetry
\begin{equation}
G_{SM} = SU(3)_C  \otimes SU(2)_T \otimes U(1)_Y
\label{eq:SMGroup}
\end{equation}

where the subscripts indicate the conserved charges: the strong charge, or color C, the weak isospin T (or rather its third component T3) and the hypercharge Y. These quantities can be related to the electric charge Q through the Gell-Mann-Nishijima relation:
\begin{equation}
Q = \frac{Y}{2} + T_3.
\end{equation}

In the quantum field framework, the elementary particles correspond to the irreducible representations of the G$_{SM}$ symmetry group. In particular, the particles are divided in two categories, fermions and bosons, according to their spin-statistics. Described by the Fermi-Dirac statistics, Fermions have half-integer spin and are sometimes called ``matter-particles". Bosons or ``force carriers" have integer spin, follow the Bose-Einstein statistics and mediate the interaction between fermions. The fundamental fermions and their quantum numbers are listed in Tab \ref{tab:SMParticles}.

\begin{table}[]
\centering
\begin{tabular}{|cccc|c|c|c|}\hline
Generation               & I                   & II                  & III                 & T                       & Y                   & Q                   \\\hline

\multirow{6}{*}{Leptons}                          &                     &                     &                     &                         &                     &                     \\
& $\begin{pmatrix}\ \nu_e\\ e \end{pmatrix}_L$ & $\begin{pmatrix}\ \nu_\mu\\ \mu \end{pmatrix}_L$ & $\begin{pmatrix}\ \nu_\tau\\ \tau \end{pmatrix}_L$ & $\begin{matrix}\ 1/2\\ -1/2 \end{matrix}$ & $\begin{matrix}\ -1\\ -1 \end{matrix}$ & $\begin{matrix}\ 0\\ -1 \end{matrix}$ \\
                         &                     &                     &                     &                         &                     &                     \\
                         & $e_R$         & $\mu_R$      & $\tau_R$                  & 0                      & -2                  & 1    \\
                         &                     &                     &                     &                         &                     &                     \\\hline
\multirow{7}{*}{Quarks} 
                         &                     &                     &                     &                         &                     &                     \\
& $\begin{pmatrix}\ u\\ d' \end{pmatrix}_L$ & $\begin{pmatrix}\ c\\ s' \end{pmatrix}_L$ & $\begin{pmatrix}\ t\\ b' \end{pmatrix}_L$ & $\begin{matrix}\ 1/2\\ -1/2 \end{matrix}$ & $\begin{matrix}\ 1/3\\ 1/3 \end{matrix}$ & $\begin{matrix}\ 2/3\\ -1/3 \end{matrix}$ \\
                         &                     &                     &                     &                         &                     &                     \\
& $\begin{matrix}\ u_R\\ d'_R \end{matrix}$ & $\begin{matrix}\ c_R\\ s'_R \end{matrix}$ & $\begin{matrix}\ t_R\\ b'_R \end{matrix}$ & $\begin{matrix}\ 0\\ 0 \end{matrix}$ & $\begin{matrix}\ 4/3\\ -2/3 \end{matrix}$ & $\begin{matrix}\ 2/3\\ -1/3 \end{matrix}$ \\
                         &                     &                     &                     &                         &                     &                    \\\hline
\end{tabular}
\caption{SM elementary fermions. The subscripts L and R indicate respectively the negative helicity (left-handed) and the positive helicity (right-handed).}
\label{tab:SMParticles}
\end{table}

Quarks can interact via all three the fundamental forces; they are triplets of SU(3)$_C$, that is they can exist in three different colors: C = R, G, B. If one chooses a base where $u$, $c$ and $t$ quarks are simultaneously eigenstates of both the strong and the weak interactions, the remaining eigenstates are usually written as $d$, $s$ and $b$ for the strong interaction and $d'$, $s'$ and $b'$ for the weak interaction, because the latter ones are the result of a Cabibbo rotation on the first ones.
Charged leptons interact via the weak and the electromagnetic forces, while neutrinos only interact via the weak force. 
The gauge group univocally determines the number of gauge bosons that carry the interaction; the gauge bosons correspond to the generators of the group: eight gluons (g) for the strong interaction, one photon ($\gamma$) and three bosons (W$^\pm$, Z$^0$) for the electroweak interaction.
A gauge theory by itself can not provide a description of massive particles, but it is experimentally well know that most of the elementary particles have non-zero masses. The introduction of massive fields in the Standard Model lagrangian would make the theory non-renormalizable, and - so far - mathematically impossible to handle. This problem is solved in the Standard Model by the introduction of a scalar iso-doublet $\Phi(x)$, the Higgs field, which gives mass to W$^\pm$ and Z$^0$ gauge bosons through the electroweak symmetry breaking and to the fermions through Yukawa coupling \cite{Higgs1964,Higgs19642}.

\section{Neutrinos: in and beyond the Standard Model}
\subsection{Neutrinos in the SM}
Neutrino were introduced in the SM as left-handed massless Weyl spinors.
The Dirac equation of motion
\begin{equation}
(i\gamma^ \mu \partial_\mu - m) \psi = 0
\end{equation}
for a fermionic field 
\begin{equation}
 \psi =  \psi_L +  \psi_R
\end{equation}
is equivalent to the equaitons
\begin{equation}
i\gamma^ \mu \partial_\mu  \psi_L = m \psi_R
\end{equation}
\begin{equation}
i\gamma^ \mu \partial_\mu  \psi_R = m \psi_L
\end{equation}

for the chiral fields $\psi_R$ and $\psi_L$, whose evolution in space and time is coupled through the mass $m$.
If the fermion is massless, the chiral fields decouple and the fermion can be described by a single Weyl spinor with two independent components. Pauli initially rejected the description of a physical particle through a single Wyle spinor because of its implication of parity violation. In fact, since the spatial inversion operator throws $\psi_R \leftrightarrow \psi_L$, parity is conserved only if the both the chiral components exist at the same time. 
\textcolor{red}{ADD CITATIONS}.  For the neutrino introduction in the SM, experiments came in help of the theoretical description:  the constraint of parity conservation weakened after Wu's experiment \textcolor{red}{ADD CITATIONS AND DATES}, there was no experimental indication for massive neutrinos and neutrinos likely interacted only via the left-handed component. 

The symmetry group $SU(2)_T \otimes U(1)_Y$ is the only group relevant for neutrino interactions. The SM electroweak lagrangian is the most general renormalizable lagrangian invariant under the local symmetry group $SU(2)_T \otimes U(1)_Y$. The lagrangian couples the weak isotopic spin doublets and singlets described in \ref{tab:SMParticles} with the gauge bosons  $A^{\mu}_{a}$ ($a$ $=$ 1,2,3) and $B^{\mu}$, and Higgs doublet $\Phi(x)$:

\begin{eqnarray}
\lefteqn{\mathcal{L} = i\sum_{\alpha=e,\mu,\tau} \bar{L}'_{\alpha L}  \slashed D L'_{\alpha L} + 
 i\sum_{\alpha=1,2,3} \bar{Q}'_{\alpha L}  \slashed D Q'_{\alpha L} {}}
 \nonumber\\
 & & {} + i\sum_{\alpha=e,\mu,\tau} \bar{l}'_{\alpha R}  \slashed D l'_{\alpha R} + i\sum_{\alpha=d,s,b} \bar{q}'^D_{\alpha R}  \slashed D q'^D_{\alpha R} + i\sum_{\alpha=u,c,t} \bar{q}'^U_{\alpha R}  \slashed D q'^U_{\alpha R}
 \nonumber\\
 & & {} -\frac{1}{4}A_{\mu \nu}A^{\mu \nu} - \frac{1}{4}B_{\mu \nu}B^{\mu \nu}
 \nonumber\\
 & & {} +(D_{\rho}\Phi)^\dagger(D^{\rho}\Phi) - \mu^2\Phi^\dagger\Phi - \lambda(\Phi^\dagger\Phi)^2 
 \nonumber\\
 & & {} -\sum_{\alpha,\beta=e,\mu,\tau} \Big(Y'^l_{\alpha\beta}\bar{L}'_{\alpha L}  \Phi l'_{\beta R} + Y'^{l*}_{\alpha\beta}\bar{l}'_{\beta R}  \Phi^\dagger L'_{\alpha L}\Big)
  \nonumber\\
 & & {} -\sum_{\alpha=1,2,3} \sum_{\beta=d,s,b} \Big(Y'^D_{\alpha\beta}\bar{Q}'_{\alpha L}  \Phi q'^D_{\beta R} + Y'^{D*}_{\alpha\beta}\bar{q}'^D_{\beta R}  \Phi^\dagger Q'_{\alpha L}\Big)
  \nonumber\\
 & & {} -\sum_{\alpha=1,2,3} \sum_{\beta=u,c,t} \Big(Y'^U_{\alpha\beta}\bar{Q}'_{\alpha L}   \widetilde{\Phi} q'^U_{\beta R} + Y'^{U*}_{\alpha\beta}\bar{q}'^U_{\beta R} \widetilde{\Phi}^\dagger Q'_{\alpha L}\Big).
\end{eqnarray}

The first two lines of the lagrangian summarize the kinetic terms for the fermionic fields and their coupling to the gauge bosons $A^{\mu\nu}_a$, $B^{\mu\nu}$ \footnote{In gauge theories the ordinary derivative $\partial_\mu$  is substitued with �the covariant derivative $D_\mu$. Here $D_\mu = \partial_\mu + igA_\mu \cdot I + ig'B_\mu\frac{Y}{2}$, where I and Y are the SU(2)$_L$ and U(1)$_Y$ generators, respectively.}.
The third line describes the kinetic terms and the self-coupling terms of the gauge bosons. The forth line is the Higgs lagrangian, which results in the spontaneous symmetry breaking. The last three lines describe the Yukawa coupling between fermions and the Higgs field, origin of the fermion's mass.

The coupling between left-handed and right-handed field generates the mass term for fermions. The SM assumes only left-handed components for neutrinos, thus implying zero neutrino mass. Since any linear combination of massless fields results in a massless field, the flavor eigenstates are identical to the mass eigenstates in the SM.

\subsection{Neutrino Oscillations}
The determination of the flavor of a neutrino dynamically arises from the corresponding charged lepton associated in a CC interaction; for example, a $\nu_e$ is a neutrino which produces an $e^-$, a $\bar\nu_\mu$ is a neutrino which produces a $\mu^+$, $etc$. 
The neutrino flavor eigenstates $\ket{\nu_\alpha}$,  with $\alpha = e,\mu,\tau$, are orthogonal to each other and form a base for the the weak interaction matrix.

Overwhelming experimental data show neutrinos change flavor during their propagation. This phenomenon, predicted by Bruno Pontecorvo in 1957, is  called ``neutrino oscillations" and it is possible only if the flavor eigenstate are not identical to the mass eigenstates. A minimal extension of the Standard Model introduces three mass eigenstates, $\ket{\nu_i}$ ($i = 1,2, 3$), whose mass $m_i$ is well defined. 
The unitary Pontecorvo-Maki-Nakagawa-Sakata matrix transforms the spinor wave functions ($\psi$) of each component  between flavor and mass bases as follows

\begin{equation} 
\sum \psi_\alpha \ket{\nu_\alpha} =  \sum \psi_i \ket{\nu_i}, \rightarrow \psi_\alpha =  U_{PMNS} \psi_i, 
\end{equation}

with

\begin{equation}
U_{PMNS}=\\
\left[ 
\begin{array}{ccc}
c_{12} & s_{12} & 0\\
-s_{12} & c_{12} & 0\\
0 & 0 & 1\\
\end{array}
\right]
\left[ 
\begin{array}{ccc}
c_{13} & 0 &s_{13}e^{-i\delta} \\
0 & 1 & 0\\
-s_{13}e^{-i\delta}& 0 & c_{13}\\
\end{array}
\right]
\left[
\begin{array}{ccc}
1 & 0 & 0\\
0 & c_{23} & s_{23} \\
0& -s_{23} & c_{23} \\
\end{array}
\right]
\left[ 
\begin{array}{ccc}
e^{i\alpha_1} & 0 & 0\\
0& e^{i\alpha_2} & 0\\
0 & 0 & 1\\
\end{array}
\right]
\end{equation}





where $c$ e $s$ stand respectively for cosine and sine of the corresponding  mixing angles ($\theta_{12}$, $\theta_{23}$ and $\theta_{13}$), $\delta$ � is the Dirac CP violation phase, $\alpha_1$ and $\alpha_{2}$   � is the eventual Majorana CP violation phases.

\textcolor{red}{Experimental results summary}

\section{Beyond the Standard Model}
The discovery of neutrino oscillation and its implication of non-zero neutrino mass mark  the beginning of a new, exciting era in neutrino physics: the era of physics Beyond the Standard Model (BSM) in the neutrino sector.
We are currently searching for new, deeper theories that can accommodate neutrinos with non-zero mass, while remaining consistent with the rest of the Standard Model. %More excitingly, we need to probe these theories experimentally. 

\subsubsection{Open Questions in Neutrino Physics}
On one hand, the last X decades of experiments in neutrino oscillations brought spectacular advancements in the understanding of the oscillations pattern, measuring the neutrino mixing angles and mass splitting with a precision of less than 10\%.  On the other, it opened the field for a series of questions needing experimental answers. 
Critical challenges in the next decade will entail an even deeper understanding of the neutrino mixing pattern, investigation of the neutrino mass origin, neutrino number and nature, and assessing CP violation in the lepton sector. 


Following the recommendation of the latest Particle Physics Project Prioritization Panel  \cite{ParticlePhysicsProjectPrioritizationPanel(P5):2014pwa}, the US  is dedicating substantial resources to the development of a short- and long- baseline neutrino program to address many of these fundamental questions.  This program pivots on the Liquid Argon Time Projection Chamber (LArTPC) detector technology which will be described in \ref{ch:1}.  

The main goals of these research programs include:
\begin{itemize}
\item[-] Assessment of the existence of right-handed sterile neutrinos. 
\item[-] Determination of the sign of $\Delta m^2_{13}$ (or $\Delta m^2_{23}$), i.e., the ``neutrino mass ordering".
\item[-] Determination of the octant, i.e.  whether $\theta_{23}$ is maximal.
\item[-] Determination the status of CP symmetry in the lepton sector.
\end{itemize}

\textcolor{red}{Smallness of neutrino masses, link to GUTs}

\subsubsection{Towards a more fundamental theory}
Despite its highly predictive power, a number of conceptual issues arise in the SM which disfavor it to be a good candidate for a fundamental theory.

The SM rather complex group structure, where a gauge group is formed with the direct product of other three groups as shown in eq. \ref{eq:SMGroup},  is unexplained. Also, the SM fails to include a suitable dark matter candidate and a mechanisms that accounts for the baryon asymmetry of the universe. Within the SM, a total of 25 parameters remain seemingly arbitrary and need to be fitted to data: 3 gauge couplings, 9 charged fermion masses, 3 mixing angles and one CP phase in the CKM matrix, the Higgs mass and quartic coupling, $\theta_{QCD}$, 3 neutrino masses, 3 neutrino mixing angles, 1 Dirac phase and, eventually,  2 Majorana phases.





