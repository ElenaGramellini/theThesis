\chapter{Tracking Optimization}
Understanding how kaons and pions are tracked inside the TPC and optimizing the reconstruction algorithms to maximize the correct identification of the interaction point is a fundamental step of the analysis. 


\section{MC sample and WC2TPC match}
The optimization is performed on a MC sample of \textcolor{red}{HOW MANY?} kaons and 359000 pions produced with the DDMC technique. DDMC particles are shot from the WC4 location into the TPC following the beam profile.
We mimic the matching between the WC and the TPC track on Monte Carlo by constructing a fake WC track using truth information at wire chamber four. We then apply the same WC to TPC matching algorithm as in data. 

Plots I want in this section:
\begin{enumerate}
\item WC2TPC MC DeltaX, DeltaY and $\alpha$
\item Delta L, reco - true
\item Delta L, reco - true Elastic, Delta L, reco - true Inelastic, other
\item Length Quality cut
\item Efficiency as a function of true KE and Angle
\end{enumerate}





\section{Wire chamber-to-TPC match}
\subsection{Wire chamber-to-TPC match: importance and definition}
\textcolor{red}{something something about why this match is important}

In data, we attempt to uniquely match one WC-Track to one and only one reconstructed TPC track. This match is done by using in the $X$ and $Y$ coordinate of the extrapolated WC-Track to the upstream most point of the reconstructed TPC Track and by using the angle between the incoming track angle and the reconstructed TPC. We define $\Delta$X as the difference between the $x$ position of the most upstream point of the TPC track and the $x$ position of the WC track as projected to the TPC front face. $\Delta$Y is defined analogously. We define  $\Delta$R as $ \Delta \text{R} =  \sqrt{ \Delta \text{X}^2 +  \Delta \text{Y}^2}  $. The angle between the incident WC Track and the TPC track in the plane that contains them defines $\alpha$.  

We define a match between WC-track and TPC reconstructed track if  $\Delta \text{R} < r_{T}$, $\alpha < \alpha_{T}$ and the Z position of the first reconstructed point of the TPC track is within 2 cm from the TPC front face. The determination of the best $r_{T}$ and $\alpha_{T}$ is the scope of the following section.

In MC, we mimic the matching between the WC and the TPC track on Monte Carlo by constructing a fake WC track using truth information at wire chamber four. We then apply the same WC to TPC matching algorithm as in data. 

\subsection{Matching optimization}
Scope of this optimization study is assessing the goodness of the wire chamber-to-TPC match on Monte Carlo and porting the optimized selection cuts to data. A word of caution is necessary here. With this study, we want to minimize pathologies associated with the presence of the primary kaon itself, like the incorrect association between the beamline kaon its decay products inside the TPC.  Assessing the contamination from pile-up\footnote{We remind the reader that the DDMC is a single particle Monte Carlo, where the beam pile up is not simulated.}, albeit related, is beyond the scope of this study .



In MC, we are able to define a correct WC2TPC match using the Geant4 truth information. We are thus able to count how many times the WC tracks get associated incorrectly with the wrong TPC reconstructed track. 

%We define two populations: TPC tracks correctly matched and one of all the other match.
We define a correct match if the all following conditions are met:
\begin{itemize}
\item[-] the length of the true primary Geant4 track in the TPC is greater than 2 cm,  
\item[-] the length of the reconstructed track length is greater than 2 cm,
\item[-] the Z position of the first reconstructed point is within 2 cm from the TPC front face
\item[-] the distance between the reconstructed track and the true entering point is the minimum compared with all the other reconstructed tracks.
\end{itemize}

In order to count the wrong matches, we consider all the reconstructed tracks whose Z position of the first reconstructed point lies within 2 cm from the TPC front face. Events with true length in TPC < 2 cm included. 
Since kaons are shot 100 cm upstream from the TPC front face, the following two scenarios are possible from a truth standpoint: 
\begin{itemize}
\item[[$Ta$]] the primary kaon decays or interact strongly before getting to the TPC,
\item[[$Tb$]] the primary kaon enters the TPC.
\end{itemize}

Once we choose the selection cuts to determine a reconstructed wire chamber-to-TPC match, the following four scenarios are possible : 
\begin{itemize}
\item[1)] no reconstructed tracks are matched
\item[2)] the correct track and one (or more) wrong tracks are matched
\item[3)] only the correct track is matched
\item[4)] one (or more) wrong track is matched.
\end{itemize}

If we keep only events with one and only one match, we discard cases 1) and 2) from the cross section measurement. For each set of $\alpha$ and $\Delta R$ selection value, we define purity and efficiency of the selection as follows:
\begin{equation}
\text{Efficiency} = \frac{\text{events with correct track matched}}{\text{events with primary in TPC}}
\end{equation}

\begin{equation}
\text{Purity} = \frac{\text{Events with correct track matched}}{\text{Events with correct track matched + Events with wrong track matched}}
\end{equation}
