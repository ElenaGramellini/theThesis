\chapter{Liquid Argon Detectors at the Intensity Frontier}
Following the recommendation of the latest Particle Physics Project Prioritization Panel  \cite{ParticlePhysicsProjectPrioritizationPanel(P5):2014pwa}, the US is dedicating substantial resources to the development of a short- and long- baseline neutrino program to address the fundamental questions in neutrino physics today. A large portion of this program pivots on the Liquid Argon Time Projection Chamber (LArTPC) detector technology. In the next few years, LArTPC experiments -- such as the Short-Baseline Neutrino program (SBN) and DUNE -- will be major players in the intensity frontier field. 

\section{The SBN Program}
\subsection{SBN Goals}
\subsection{Neutrino Interactions and Detection }

\section{DUNE}
\subsection{DUNE Non-Accelerator Physics Program}
\subsection{Rare Decay Searches: Experimental Limit}
\subsection{Nucleon Decay Detection in LAr}

\section{Liquid Argon Time Projection Chambers at the Intensity Frontier}

%Bubble-chamber experiments played a key role in probing the properties of ?-interactions. The Liquid Argon Time Projection Chambers (LArTPC) technology,  first proposed by C.Rubbia in 1977 with ICARUS project [14], is considered the modern evolution of bubble-camber concept, with the additional features of three-dimensional event reconstruction, high-resolution calorimetry, active mass coincident with detector sensitive mass and can intrinsically supply a trigger signal (self-triggering) by means of the scintillation light produced in the liquid noble gas. This technology is ideal to perform $\nu$-studies in a broad energy range, from MeV up to few GeV, with high event reconstruction efficiency, thanks to the capability of particle identifcation and detailed reconstruction of different interaction topologies. In Figure 1.4 is shown a neutrino interaction event, producing a proton, a pion and a muon, as seen in a bubble chamber and in a LArTPC.


\subsection{Time Projection Chamber}
\subsection{Ionization Detectors with Noble Liquids}
\subsection{LArTPC: Principles of Operation}
\subsection{Liquid Argon Ionization Charge Detection}
\subsubsection{Electron Life Time \& purity}
\subsubsection{Space Charge Effect}
\subsubsection{Recombination Effect}
\subsection{Liquid Argon scintillation Light Detection}
\subsubsection{LAr Scintillation Process}
\subsubsection{Wavelength Shifting of LAr Scintillation Light}