\chapter{Introduction}

\section{The Standard Model}
The Standard Model of particle physics the most accurate theoretical description of the subatomic world.  The Standard Model describes the strong, electromagnetic and weak interactions among  elementary particles in the framework of quantum field theory. The weak and the electromagnetic interactions are unified into the electroweak interaction. The Standard Model does not describe gravity or general relativity.
The Standard Model is a gauge theory based on the local group of symmetry
\begin{equation}
G_{SM} = SU(3)_C  \otimes SU(2)_T \otimes U(1)_Y
\end{equation}

where the subscripts indicate the conserved charges: the strong charge, or color C, the weak isospin T (or rather its third component T3) and the hypercharge Y. These quantities can be related to the electric charge Q through the Gell-Mann-Nishijima relation:
\begin{equation}
Q = \frac{Y}{2} + T_3
\end{equation}

In the quantum field framework, the elementary particles correspond to the irreducible representations of the G$_{SM}$ symmetry group. In particular, the particles are divided in two categories, fermions and bosons, according to their spin-statistics. Described by the Fermi-Dirac statistics, Fermions have half-integer spin and are sometimes called ``matter-particles". Bosons or ``force carriers" have integer spin, follow the Bose-Einstein statistics and mediate the interaction between fermions. The fundamental fermions and their quantum numbers are listed in Tab \ref{tab:SMParticles}\textcolor{red}{INSERT TABLE}
\begin{table}[!htbp]

\begin{tabular}{|l ccc|c|c|c|}
\hline
Generation: & I & II & III & T$_3$ & Y & Q \\
\hline
Leptonic doubles: & e$_L$ & e$_L$ & e$_L$ & 0 & -2 & -2 \\
Leptonic singlets: & e$_R$ & e$_R$ & e$_R$ & 0 & -2 & -2 \\
\hline
Quark doubles: & I & II & III & T$_3$ & Y & Q \\
Quark singlets: & I & II & III & T$_3$ & Y & Q \\
\hline
\end{tabular}
\caption{SM elementary fermions. The subscripts L and R indicate respectively the negative helicity (left-handed) and the positive helicity (right- handed).}
\label{tab:SMParticles}
\end{table}




Quarks can interact via all three the fundamental forces; they are triplets of SU(3)$_C$, that is they can exist in three different colors: C = R, G, B. If one chooses a base where u, c and t quarks are simultaneously eigenstates of both the strong and the weak interactions, the remaining eigenstates are usually written as d, s and b for the strong interaction and d', s' and b' for the weak interaction, because the latter ones are the result of a Cabibbo rotation on the first ones.
Charged leptons interact via the weak and the electromagnetic forces, while neutrinos only interact via the weak force. 
The gauge group univocally determines the number of gauge bosons that carry the interaction; the gauge bosons correspond to the generators of the group: eight gluons (g) for the strong interaction, one photon ($\gamma$) and three bosons (W$^\pm$, Z$^0$) for the electroweak interaction.
A gauge theory by itself can not provide a description of massive particles, but it is experimentally well know that most of the elementary particles have non-zero masses. The introduction of massive fields in the Standard Model lagrangian would make the theory non-renormalizable, and - so far - mathematically impossible to handle. This problem is solved in the Standard Model by the introduction of a scalar iso-doublet $\phi$(x), the Higgs field, see Tab \textcolor{red}{INSERT TABLE}, which gives masse to W$^\pm$ and Z$^0$ gauge bosons and to the fermions [1, 2].


\section{Neutrinos in the Standard Model}
\section{Beyond the Standard Model}
The discovery of neutrino oscillation marks  the beginning of a new, exciting era in neutrino physics: the era of physics Beyond the Standard Model (BSM) in the neutrino sector.
We are currently searching for new, deeper theories that can accommodate neutrinos with non-zero mass, while remaining consistent with the rest of the Standard Model. More excitingly, we need to probe these theories experimentally. 
\subsection{Neutrino Oscillations}
\subsection{Proton Decay}
