\chapter{Negative Pion Cross Section Measurement}\label{ch:PionXS}
%{\raggedleft ``\emph{Y ella es flama que se eleva, Y es un p\'ajaro a volar.} \par}
%{\raggedleft \emph{En la noche que se incendia, estrella de oscuridad}\par}
%{\raggedleft \emph{que busca entre la tiniebla, la dulce hoguera del beso.}"\par}
%{\raggedleft -- Lila Downs, Benediction And Dream,  2002 -- \par}
%\vspace{0.5cm}

In this chapter, we show the result of the thin slice method to measure 
the ($\pi^-$-Ar) total hadronic cross section. In Section \ref{ch:PionXSRaw}, we start by measuring the raw cross section, i.e. the cross section obtained exclusively using data reconstruction, without any additional corrections. In Section \ref{ch:PionXSCorrections}, we apply a statistical subtraction of the background contributions based on simulation and a correction for detection inefficiency. The final results are presented in Section \ref{ch:FinalPion}.


\section{Raw Cross Section}\label{ch:PionXSRaw}
We measure the raw ($\pi^-$-Ar) total hadronic cross section as a function of the kinetic energy in the two chosen data sets, the -60A and -100A negative runs. 
As we will clarify in Section \ref{ch:PionXSCorrections},  the corrections to the raw cross section depend on the beam conditions and need to be calculated independently for the two datasets. Thus, we present here the measurement of the raw cross section on the two datasets separately.


As stated in section \ref{ch:XSRaw},  the raw cross section is given by the equation \label{eq:thinTargetXSSolved}
\begin{equation}
 \sigma_{TOT} (E_i)  = \frac{1}{n \delta X}\frac{N^{\text{TOT}}_{\text{Int}}(E_i)}{N^{\text{TOT}}_{\text{Inc}}(E_i)},
\end{equation}

where $N^{\text{TOT}}_{\text{Int}}$  is the measured number of particles interacting at kinetic energy $E_i$, $N^{\text{TOT}}_{\text{Inc}}$ is the  measured  number of particles incident  on an argon slice at  kinetic energy $E_i$,  $n$ is the density of the target centers  and $\delta X$ is the thickness of the argon slice. The density of the target centers and the slab thickness are $n = 0.021\cdot10^{24} \text{ cm}^{-3} $ and  $\delta X=0.47\text{ cm}$, respectively.


Figure \ref{fig:InteractingRaw} shows the distribution of  $N^{\text{TOT}}_{\text{Int}}$  as a function of the kinetic energy for the 60A dataset on the left and for the 100A dataset on the right. The data central points are represented by black dots, the statistical uncertainty is shown in black, while the systematic uncertainty is shown in red. Data is displayed over the $N^{\text{TOT}}_{\text{Int}}$  distribution obtained with a MC mixed sample of pions, muon and electrons (additional details on the composition will be provided in Section \ref{ch:BKGsubXS}). The contribution from the simulated pions is shown in blue, the one from secondaries in red, the one from muons in yellow and the ones from electrons in gray. 
The simulated pion's and backgrounds' contributions are stacked; the sum of the integrals from each particle species is normalized to the integral of the data.
 
Figure \ref{fig:IncidentRaw} shows the distribution of  $N^{\text{TOT}}_{\text{Inc}}$   for the 60A dataset on the left and for the 100A dataset on the right. Data is displayed over the MC. The same color scheme and normalization procedure is used for both the interacting and incident histograms. 


Figure \ref{fig:XSRaw} shows the raw cross section for the 60A dataset on the left and for the 100A dataset on the right, statistical uncertainty in black and systematic uncertainty in red. The raw data cross section is overlaid to the reconstructed cross section for the MC mixed sample, displayed in azure. Since the background contributions and the detector effects for the 60A and 100A sample are different, it is premature to compare the raw cross sections obtained from the two samples at this point.

We describe the calculation of the statistical uncertainty for the interacting, incident and cross section distributions in Section \ref{ch:StatUncertaintyXSRaw}; we describe the procedure to calculate the corresponding systematics uncertainty on Section \ref{ch:SysUncertaintyXSRaw}.

\begin{figure}[p]
\centering  
\includegraphics[width=0.48\textwidth]{Chapter-6/Images/Plots60A_MCData_Int_StatSyst.pdf}
\includegraphics[width=0.48\textwidth]{Chapter-6/Images/Plots100A_MCData_Int_StatSyst.pdf}
\caption{Raw number of interacting pion candidates as a function of the reconstructed kinetic energy for the 60A runs (left) and for the 100A runs (right). The statistical uncertainties are shown in black, the systematic uncertainties in red.}
\label{fig:InteractingRaw}
\end{figure}


\begin{figure}
\centering  
\includegraphics[width=0.48\textwidth]{Chapter-6/Images/Plots60A_MCData_Inc_StatSyst.pdf}
\includegraphics[width=0.48\textwidth]{Chapter-6/Images/Plots100A_MCData_Inc_StatSyst.pdf}
\caption{Raw number of incident pion candidates as a function of the reconstructed kinetic energy for the 60A runs (left) and for the 100A runs (right). The statistical uncertainty is shown in black, the systematic uncertainties in red.}
\label{fig:IncidentRaw}
\end{figure}

\begin{figure}
\centering  
\includegraphics[width=0.48\textwidth]{Chapter-6/Images/Plots60A_MCData_XS_StatSyst.pdf}
\includegraphics[width=0.48\textwidth]{Chapter-6/Images/Plots100A_MCData_XS_StatSyst.pdf}
\caption{Raw ($\pi^-$-Ar) total hadronic cross section for the 60A runs (left) and for the 100A runs (right). The statistical uncertainty is shown in black, the systematic uncertainties in red. The raw cross section obtained with a MC mixed sample of pions, muon and electrons in the percentage predicted by G4Beamline is shown in azure. }
\label{fig:XSRaw}
\end{figure}


\subsection{Statistical Uncertainty}\label{ch:StatUncertaintyXSRaw}
The statistical uncertainty for a given kinetic energy bin of the cross section  is calculated by error propagation from the statistical uncertainty on $N^{\text{TOT}}_{\text{Inc}}$ and $N^{\text{TOT}}_{\text{Int}}$ correspondent bin.  Since the number of incident particles in each energy bin is given by a simple counting, we assume that $N^{\text{TOT}}_{\text{Inc}}$ is distributed as a poissonian with mean and variance equal to $N^{\text{TOT}}_{\text{Inc}}$ in each bin.  
On the other hand, $N^{\text{TOT}}_{\text{Int}}$ follows a binomial distribution: a particle in a given energy bin might or might not interact.  The variance for the binomial is given by  
\begin{equation}
\text{\textsf{Var[}} N^{\text{TOT}}_{\text{Int}} \text{\textsf{]}}
 = \mathcal{N}P_{Interacting}(1-P_{Interacting}).
\label{eq:binVar}
\end{equation}

Since the interaction probability $P_{Interacting}$ is $\frac{ N^{\text{TOT}}_{\text{Int}}}{N^{\text{TOT}}_{\text{Inc}}}$ and the number of tries $\mathcal{N}$ is $N^{\text{TOT}}_{\text{Inc}}$, equation \ref{eq:binVar} translates into
\begin{equation}
\text{\textsf{Var[}} N^{\text{TOT}}_{\text{Int}} \text{\textsf{]}}
= N^{\text{TOT}}_{\text{Inc}}\frac{ N^{\text{TOT}}_{\text{Int}}}{N^{\text{TOT}}_{\text{Inc}}} (1-\frac{ N^{\text{TOT}}_{\text{Int}}}{N^{\text{TOT}}_{\text{Inc}}}) = N^{\text{TOT}}_{\text{Int}}(1-\frac{ N^{\text{TOT}}_{\text{Int}}}{N^{\text{TOT}}_{\text{Inc}}}). 
\end{equation}

$N^{\text{TOT}}_{\text{Inc}}$ and $N^{\text{TOT}}_{\text{Int}}$ are not independent.
The statistical uncertainty on the cross section is thus calculated as 
\begin{equation}
\delta\sigma_{TOT}(E) = \sigma_{TOT}(E) \Big(\frac{\delta N^{\text{TOT}}_{\text{Int}}}{N^{\text{TOT}}_{\text{Int}}}+\frac{\delta N^{\text{TOT}}_{\text{Inc}}}{N^{\text{TOT}}_{\text{Inc}}}\Big) 
\end{equation}
where:
\begin{eqnarray}
\delta N^{\text{TOT}}_{\text{Inc}} = \sqrt[]{N^{\text{TOT}}_{\text{Inc}}} \\
\delta N^{\text{TOT}}_{\text{Int}} = \sqrt[]{N^{\text{TOT}}_{\text{Int}}\Big(1-\frac{ N^{\text{TOT}}_{\text{Int}}}{N^{\text{TOT}}_{\text{Inc}}}\Big)}.
\end{eqnarray}



\subsection{Treatment of Systematics} \label{ch:SysUncertaintyXSRaw}
The only systematic effect considered in the measurement of the raw cross section results from the propagation of the uncertainty associate with the measurement of the kinetic energy at each argon slab. As shown in Section \ref{ch:kinEn}, the uncertainty on the kinetic energy of a pion candidate at the j$^{th}$ slab of argon  is given by

\begin{eqnarray}
\delta KE_{j} &=& \sqrt{\delta p_{Beam}^2 + \delta E_{Loss}^2 +  \delta  E_{\text{dep FF-j}}^2}\\
&=& \sqrt{(2\% \text{ }p_{Beam})^2 +  (\sim 6 \text{ [MeV]})^2 +  (j-1)^2 (\sim0.08\text{ [MeV]})^2}.
\end{eqnarray}

We propagate this uncertainty  by varying the energy measurement $KE_{j}$ at each argon slab. We measure $N^{\text{TOT}}_{\text{Inc}}$,  $N^{\text{TOT}}_{\text{Int}}$ and the cross section  in three cases: first assigning the measured $KE_{j}$ at each kinetic energy sampling, then assigning $KE_{j} + \delta KE_{j}$, and finally assigning $KE_{j} - \delta KE_{j}$. The difference between the values obtained using the $KE_{j}$ sampling and the maximum and minimum values in each kinetic energy bin determines the systematic uncertainty.

%We propagate this uncertainty  by calculating the $N^{\text{TOT}}_{\text{Inc}}$,  $N^{\text{TOT}}_{\text{Int}}$ and cross section plots twice: first assigning $KE_{j} + \delta KE_{j}$ at each kinetic energy sampling, then assigning $KE_{j} - \delta KE_{j}$. The difference between the central value and the maximum and minimum value in each kinetic energy bin gives the systematic uncertainty.

\section{Corrections to the Raw Cross Section}\label{ch:PionXSCorrections}
As described in section \ref{ch:MCCorrections} as series of corrections are needed to derive the true pion cross section from the raw cross section. 
The corrected cross section is given in equation \ref{eq:C}, 

\begin{equation}
   \sigma^{\pi^-}_{TOT}(E_{i})  = \frac{1}{n \delta X}\frac{ \epsilon^{\text{Inc}}(E_i)  \hspace{0.2cm} C^{\pi MC}_{\text{Int}} (E_{i}) \hspace{0.2cm} N^{\text{TOT}}_{\text{Int}} (E_{i}) }{   \epsilon^{\text{Int}}(E_i) \hspace{0.2cm} C^{\pi MC}_{\text{Inc}} (E_{i}) \hspace{0.2cm}  N^{\text{TOT}}_{\text{Inc}} (E_{i})}.
 \tag{\ref{eq:C}}
\end{equation}

Section \label{ch:BKGsubXS} describes the procedure employed to obtain  
$C^{\pi MC}_{\text{Int}} (E_{i})$  and  $C^{\pi MC}_{\text{Inc}} (E_{i})$ and the propagation to the cross section measurement of the relative uncertainties.

Section \label{ch:ch:EFFXS} describes the procedure employed to obtain  
$\epsilon^{\text{Int}}(E_i)$  and $\epsilon^{\text{Inc}}(E_i)$ and the propagation to the cross section measurement of the relative uncertainties.


\subsection{Background subtraction}\label{ch:BKGsubXS}
Even if pions are by far the biggest component of the beam in negative polarity runs, the LArIAT beam is not a pure pion beam. While useful to discriminate  pions/muons/electrons from kaons, and protons, the beamline detectors are not sensitive enough to  discriminate among the lighter particles in the beam: electrons, muons and pions fall under the same mass hypothesis. Thus, we need to assess the background from beamline particles other than pions in the event selections used for the pion cross section analysis and correct for its effects.

\subsubsection{Beam Composition}\label{sec:BeamAtWC4}
We define beamline background every TPC track matched to the WC track which is not a primary pion. Potentially, there are 4 different types of beamline background:
\begin{itemize}
\item[]1) electrons,
\item[]2) muons,
\item[]3) secondaries from pion events,
\item[]4) matched pile up events.
\end{itemize}

The first step is to estimate what percentage of events used in the cross section calculation is not a primary pion.  The next two sections will illustrate this estimate for the electrons, muons and secondaries from pion event.
We estimate the last type of background, the ``matched pile up" events, to be a negligible fraction, because of the definition of the WC2TPC match: we deem the probability of a single match with a halo particle in the absence of a beamline particle\footnote{ Events with multiple WC2TPC matches are always rejected.} negligibly small. \textcolor{red}{SHOW VTX distribution in WC2TPC match}


\subsubsection{Background from Beamline Electrons and Muons}\label{stionEMu}
\begin{figure}[b]
%\includegraphics[width=0.5\textwidth,height=\textheight,keepaspectratio]{Studies/Figures//Beam60A.png}
%\includegraphics[width=0.5\textwidth,height=\textheight,keepaspectratio]{Studies/Figures//Beam100A.png}
\caption{Beam composition for the -60A runs (left) and -100A runs (right). The solid blue plot represents the simulated pion content, the yellow plot represents the simulated muon content and the grey plot represents the simulated electron content. The plots are area normalized to the number of data events, shown in red. }
\label{fig:BeamComposition}
\end{figure}

We estimate the percentage of electrons and muons in the beam via the G4Beamline MC. 
Since the beamline composition is a function of the magnet settings, we simulate separately events for magnet current of -60A and -100A. 
Table \ref{tab:beamline} shows the beam composition per magnet setting after the mass selection according to the G4Beamline simulation.
\begin{table}[p]
\centering
\begin{tabular}{|l|c|c|}
\hline
                     & I = -60 A           & I = -100 A \\ \hline
G4Pions       &   68.8 \%           &      87.4 \%        \\ \hline
G4Muons     &     4.6 \%           &        3.7 \%         \\ \hline
G4Electrons &   26.6 \%           &        8.9 \%        \\ \hline
\end{tabular}
\caption{Simulated beamline composition per magnet settings}
\label{tab:beamline}
\end{table}


Figure \ref{fig:BeamComposition} shows the momentum predictions from G4Beamline overlaid with data for the 60A runs (left) and for the 100A runs (right). The predictions for electrons, muons and pions have been staggered and their sum is area normalized to data, which is shown in red. Albeit not perfect, these plots show a reasonable agreement between the momentum shapes in data and MC. We attribute  the difference in shape to the lack of simulation of the WC efficiency in the MC which is momentum dependent and leads to enhance the number events in the center of the momentum distribution.

Once the beam composition at WC4 is know,  we simulate the electrons, muons and pions with the DDMC and we subject the three samples to the same selection chain (WC2TPC match, shower filter, pile up filter). The percentage of electrons and muons surviving the selection chain weighted by the beam composition is the  electron and muon background in the pion cross section sample, as shown in Table \ref{tab:MCafterCutContaminants}.

\subsubsection{Background from secondaries at TPC Front Face}
Pions can travel the length of the LArIAT beamline and interact hadronically in the steel or in the non-instrumented argon upstream to the TPC front face. Or, they could decay in flight between WC4 and the TPC. One of the interaction products can leak into the TPC and be matched with the WC track, contributing to the pool of events used for the cross section calculation. We call this type of particles ``secondaries" from pion events, with a terminology inspired by Geant4. 
We estimate the number of secondaries using the DDMC pion sample.  The percentage of secondaries is given by the number of matched WC2TPC tracks whose corresponding particle is not flagged as primary by Geant4.  The secondary to pion ratio is 4.9\% in the 60A sample and $Y$\% in the 100A sample.

\subsection{Background Contribution to the Cross Section}\label{sec:Correction}

Figures \ref{fig:CorrectionsBeam} show $C^{\pi MC}_{Interacting} (E_{i})$ and $C^{\pi MC}_{Incident} (E_{i})$ as a function of the kinetic energy for the 60A runs and their systematic uncertainty.  We take a 100\% systematic uncertainty on the muon and electron content: we calculate the extreme values of  $C^{\pi MC}_{Interacting} (E_{i})$ and $C^{\pi MC}_{Incident} (E_{i})$ in each bin changing the beam composition for the configurations listed in Table \ref{tab:beamlineSys}.



\begin{table}[p]
\centering
\begin{tabular}{| l | l | l | l | l | l | l | l | }
\hline
 &  \multicolumn{3}{|c|}{Magnet Current -60A} & \multicolumn{3}{|c|}{Magnet Current -100 A}\\

                                                  & MC $\pi^-$   & MC  $ \mu^-$ & MC  $e^-$ & MC  $\pi^-$ & MC  $\mu^-$ & MC  $e^-$  \\
\hline
Composition 2x muons          & 64.2	\%&9.2 \%&	26.6 \%&	83.7 \%&	7.4	\%&8.9 \% \\
Composition 0.5x muons       &71.1	\%&2.3 \%&	26.6 \%&	89.2 \%&	1.9	\%&8.9 \% \\
Composition 2x Electrons      &42.2	\%&4.6 \%&	53.2 \%&	78.5	\%&  3.7	\%&17.8 \%\\
Composition 0.5x Electrons   &82.1	\%&4.6 \%&	13.3 \%&	91.9 \%&	3.7	\%&4.4 \% \\
\hline
\end{tabular}
\caption{Beam composition variation for the study of systematics due to beam contamination.}
\label{tab:beamlineSys}
\end{table}

\begin{figure}[p]
\centering
%\includegraphics[width=\textwidth]{Studies/Figures/c60A_Multiple.png}
\caption{Left: relative pion content for interacting histogram a function of kinetic energy for the 60A runs, statistics uncertainty in azure and systematic uncertainty in blue. Right: relative pion content  for incident histogram a function of kinetic energy for the 60A runs, statistics uncertainty in azure and systematic uncertainty in blue.}
\label{fig:CorrectionsBeam}
\end{figure}

\begin{figure}[htb]
\centering
\includegraphics[width=\textwidth]{Chapter-6/Images/Bkg60A_inc_int.pdf}
\includegraphics[width=\textwidth]{Chapter-6/Images/Bkg100A_inc_int.pdf}
\caption{.}
\label{fig:BkgCorr}
\end{figure}



\subsubsection{Treatment of Systematics}


\subsection{Efficiency Correction}\label{ch:EFFXS}
The interaction point for a track used in the total hadronic cross section analysis is defined to be the last point of the WC2TPC matched track which lies inside the fiducial volume. This definition is independent from the topology of the interaction. If the TPC track stops within the fiducial volume, its last point will be the interaction point, no matter what the products of the interaction look like; if the track crosses the boundaries of the fiducial volume, the track will be considered ``through going" and no interaction point will be found.  Given this definition, it is evident that we rely on the tracking algorithm to discern where the interaction occurred in the TPC  and correctly stop the tracking. The tracking algorithm has an intrinsic angle resolution as shown in section \ref{sec:angleRes}, which limits its efficiency, especially in the case of elastic scattering occurring a low angles. 
Thus, we need to apply an efficiency correction to data in order to retrieve the true cross section.  The efficiency correction is evaluated separately for the interacting and incident histograms, namely $\epsilon^{int}_i$ and  $\epsilon^{inc}_i$, and propagated to the cross section as shown in  equation \ref{eq:True}. 

\subsubsection{Efficiency Correction: Procedure}\label{sec:EffCorrection}
We describe here the procedure to calculate the efficiency correction taking the interacting histogram as example and noting that the procedure is identical for the  incident histogram. 

We derive the correction on a set of pure pion MC, calculating its value bin by bin as the ratio between the true bin content and the correspondent reconstructed bin content. The correction is then applied to the relevant bin in data. In formulae, the efficiency correction is calculated to be

\begin{equation}
\epsilon^{\text{int}}_i  =  \frac{N^{\text{ $\pi$ Reco MC}}_{\text{Interacting}} (E_{i})}{ N^{\text{ $\pi$ True MC}}_{\text{Interacting}} (E_{i})  },
\end{equation}
 
where $N^{\text{ $\pi$ True MC}}_{\text{Interacting}} (E_{i}) $ is the content of the $i$-th bin in for the true interacting histogram, and $N^{\text{ $\pi$ Reco MC}}_{\text{Interacting}} (E_{i}) $ is the content of the $i$-th bin in for the reconstructed interacting histogram. The correction is applied to data as follows

\begin{equation}
N^{\text{ $\pi$ True Data}}_{\text{Interacting}} (E_{i})  =  \frac{N^{\text{$\pi$ Reco Data}}_{\text{Interacting}} (E_{i})}{\epsilon^{\text{int}}_i} = N^{\text{$\pi$ Reco Data}}_{\text{Interacting}} (E_{i}) \frac{N^{\text{ $\pi$ True MC}}_{\text{Interacting}} (E_{i})}{ N^{\text{ $\pi$ Reco MC}}_{\text{Interacting}} (E_{i})}.
\end{equation}

where $N^{\text{$\pi$ Reco Data}}_{\text{Interacting}} (E_{i})$ is the background subtracted bin content of the $i$-th bin in for the reconstructed interacting histogram for data, i.e. 
\begin{equation}
N^{\text{$\pi$ Reco Data}}_{\text{Interacting}} (E_{i}) =  N^{\text{TOT Data}}_{\text{Interacting}} (E_{i}) - B^{\text{Data}}_{\text{Interacting}} (E_i)  =  C^{\text{$\pi$ MC}}_{\text{Interacting}} (E_{i}) N^{\text{TOT Data}}_{\text{Interacting}} (E_{i}).
\end{equation}

Figures \ref{fig:EffCorr} show $\epsilon^{\text{int}}_i(E_{i})$ and $\epsilon^{\text{int}}_i (E_{i})$ as a function of the kinetic energy for the 60A runs and their systematic uncertainty. 

%############################# Systematics 

In section \ref{sec:angleRes}, we estimated the angular resolution for data and MC to be $\bar\alpha_{Data} = (5.0 \pm 4.5) \text{ deg}$  and 
$\bar\alpha_{MC} = (4.5 \pm 3.9) \text{ deg}$, respectively. Interaction angles smaller than the angular resolution are indistinguishable for the reconstruction. Thus, we claim we are able to  measure the cross section for interaction angles greater than 5.0 deg. Geant4 simulates interactions at all angles, as shown in figure \ref{trueScatteringAngle}. In order to calculate the efficiency correction,  we select events which have an interaction angle greater than a given $\alpha_res$ to construct the true interacting and incident histograms (the denominator of the efficiency correction). 

The systematics on the efficiency correction is estimated by varying the value of $\alpha_{res}$ to be 

\subsubsection{Treatment of Systematics}

\clearpage
\section{Results}\label{ch:FinalPion}
Figure \ref{fig:FinalXSPion} show the measurement of the ($\pi^-$-Ar) total hadronic cross section for  scattering angles greater than 5$^\circ$, as the result of the background subtraction and efficiency correction to the raw cross section. The top left plot is the measurement obtained on the 60A data, statistical uncertainty in black and systematic uncertainty in red. The top right plot is the measurement obtained on the 100A data, statistical uncertainty in black and systematic uncertainty in blue. The bottom plot shows the two measurements overlaid. In all three plot, the Geant4 prediction for the total hadronic cross section for angle scattering greater than 5$^\circ$ is displayed in green.

\begin{figure}[htb]
\centering
\includegraphics[width=0.48\textwidth]{Chapter-6/Images/TheMoneyPlot60A.pdf}
\includegraphics[width=0.48\textwidth]{Chapter-6/Images/TheMoneyPlot100A.pdf}
\includegraphics[width=0.48\textwidth]{Chapter-6/Images/TheMoneyPlot.pdf}
\caption{ \emph{Top Left:} ($\pi^-$-Ar) total hadronic cross section for  scattering angles greater than 5$^\circ$ measured in the 60A sample, statistical uncertainty in black and systematic uncertainty in red. The Geant4 prediction for the total hadronic cross section for angle scattering greater than 5$^\circ$ is displayed in green. \\ 
\emph{Top Right:} ($\pi^-$-Ar) total hadronic cross section for  scattering angles greater than 5$^\circ$ measured in the 100A sample, statistical uncertainty in black and systematic uncertainty in blue. The Geant4 prediction for the total hadronic cross section for angle scattering greater than 5$^\circ$ is displayed in green.\\
\emph{Bottom:} ($\pi^-$-Ar) total hadronic cross section measurements in the 60A and 100A samples overlaid with the  Geant4 prediction (green).}
\label{fig:FinalXSPion}
\end{figure}

